\documentclass[times, utf8, diplomski]{fer}
\usepackage{booktabs}
\usepackage[ruled]{algorithm2e}
\usepackage{float}
\usepackage{mathtools}
\usepackage{amsmath}
\newcommand\tab[1][1cm]{\hspace*{#1}}

\begin{document}
%\setcitestyle{numbers}

\thesisnumber{1286}

\title{Detectability of Patient Zero Depending on its Position in the Network}

\author{Iva Miholić}

\maketitle

% Ispis stranice s napomenom o umetanju izvornika rada. Uklonite naredbu \izvornik ako želite izbaciti tu stranicu.
\izvornik

% Dodavanje zahvale ili prazne stranice. Ako ne želite dodati zahvalu, naredbu ostavite radi prazne stranice.
\zahvala{}

\tableofcontents
\listoffigures
\listofalgorithms

\chapter{Introduction}

A \emph{network} is a set of items with connections between them. The Internet, the World Wide Web, social networks like genealogical trees, networks of friends or co-workers, biological networks like epidemiological networks, networks of citations between papers, distribution systems like postal delivery routes: they all take a form of networks. Most social, biological and technological networks have specific structural properties. Such networks are referred to as \emph{complex networks}.  An example of a complex network is represented on Figure \ref{net}.

\begin{figure}[htp]
\centering
\includegraphics[scale = 0.3]{/home/iva/dipl/img/alters2.png}
\caption{A network graph of Paul Erd\~os and his collaborators, courtesy of \citet{krebs}. The nodes represent mathematicians and the edges represent the relationship "wrote a paper with".}
\label{net}
\end{figure}
A network structure or a topology can be mathematically modelled as a graph with set of vertices (or nodes) representing the items of the network. The network structure can then be analysed using graph theory. An edge between two nodes represents a connection between the two corresponding items. Edges can be directed or undirected, depending  on the nature of the connection.  To better mimic the real-world (complex) network structure, it is common to add attributes to nodes and/or edges or to have both directed and undirected edges on the same graph.

For large-scaled complex networks that have millions or billions of vertices, the study in the form of traditional graph theory is not sufficient or sometimes possible. When this is the case, the statistical methods  for quantifying large complex networks are used. 

The ultimate goal of the study of complex network structure is to understand and explain the workings of systems built upon network such as spreading of disease or information propagation.

After statistical properties analysis, the model of the system or a process  is created. The model can help us understand the meaning of statistical properties - how they came to be as they are and how they relate to the behaviour of a networked system. Based on the statistical properties and using the right model, the behaviour of networked systems can be determined and predicted.

The basis of the complex network theory -- the structure  analysis and the process modelling -- can be found in \citet{Newman03thestructure}.

\section{Epidemic processes in complex networks}

The models for stochastic processes such as disease spreading are categorized as homogeneous or heterogeneous mixing frameworks. The former assume that all individuals in a population have an equal probability of contact and different equations can be applied to understand epidemic dynamics. Since such models fail to describe the realistic scenario of disease spreading,  heterogeneity is introduced by using a network structure.


There is an extremely close relationship between epidemiology and network theory since the connections between individuals (or group of individuals) allowing an infectious disease to propagate naturally define a contact network. Simplest epidemic dynamics consider a system with fixed total population consisting of $N$ individuals modelled with undirected contacting network. We define the contact network as an undirected and non-weighted graph $G(N, L)$ with fixed set of nodes $N$ and fixed set of links $L$. A link $(u, v)$ between two nodes exists if the two corresponding members were in contact during the epidemic time.

The structure of the network has profound impact on the contagnion dynamics but in order to understand the evolution of the epidemic over time we have to define the basic individual-level processes that govern the epidemic spreading. Complementary to the network, epidemic modelling describes the dynamical evolution of the contagion process within a population. The state of the art results on epidemic modelling in complex networks can be found in \citet{revmod}.

 Classic epidemic models generally  assume the network is static during epidemic process while the population can be divided into different classes or compartments depending on the stage of the disease, such as susceptible (those who can contract the infection), infectious (those who contracted the infection and are contagious), recovered,  removed or immune. The model defines the basic processes that govern the transition of individuals from one compartment to another.
 Each member of population can be a part of exactly one compartment at once. 
 
 Understanding the structure of the transmission network along with choosing the right epidemic model allows us to predict the distribution of infection and to simulate the full dynamics in order to control disease or plan immunization. In this thesis we will focus on SIR model for epidemic spreading and its modification, the ISS model for modelling rumour diffusion. 

\section{Finding patient zero}  
The inverse problem of estimating the initial epidemic conditions like localizing the source of an epidemic, commonly known as the patient zero problem, has only recently been formulated.

In the patient zero problem the source(s) of an epidemic or information diffusion propagation are determined based on limited knowledge of network structure or partial history of the propagation. The survey of methods for identifying the propagation source in networks can be found in \citet{soa_source}.

In the case of the SIR model there are three different approaches. \citet{Zhu} proposed a simple path counting approach and prove that the source node minimizes the maximum distance (Jordan centrality) to the infected nodes on infinite trees. \citet{Lohkov} used a dynamic message-passing algorithm and estimate the probability that a given node produces the observed snapshot using a mean-field approch and an assumption of a tree-like contact network.

\citet{Nino} introduce analytical combinatoric, as well as Monte-Carlo based methods for source detection problem. These methods produce exact and approximate source probability distribution for any network topology based on a snapshot of the epidemic at known discrete time $T$. The provided benchmark results show Monte-Carlo based MAP estimators outperforming previous results on a grid network for the SIR model. Additionally, they are applicable to many heterogeneous mixing models (SIR, IS, ISS) and are able to introduce uncertainty in the epidemic starting time, as well as uncertainty of temporal ordering of interactions. 
Even though the introduced Monte Carlo methods assume the epidemic started from a single source, one can also discriminate such hypothesis using Kolmogorov-Smirnov test.

\section{Effects of network topology on epidemic spreading and detectability of patient zero}
Complex networks show various levels of correlations in their topology which can have an impact on dynamical processes running on top of them. 
Real-world network of relevance for epidemic spreading are different from regular lattices. Networks are hierarchically organized with a few nodes that may act as hubs and where the vast majority of nodes have few interactions. Although randomness in the connection process of nodes is always present, organizing principles and correlations in the connectivity patterns define network structures that are deeply affecting the evolution and behavior of epidemic and contagion process. These network's complex features often find their signature in statistical distributions which are generally heavy tailed and skewed. 

\citet{Nino} have also introduced a metric for source detectability based on the entropy of estimated source probability distribution. The detectability of source node differs based on models parameters concerning the rate of disease spreading. Since topological properties of the network  have profound impact on epidemic dynamic, the detectability of source node depending on its topological properties is in the focus of this thesis.

\section{Structure of the thesis}
%TODO give overview of the chapters
In Chapter \ref{structure} ... 

% [x] Implement Monte Carlo and Soft margin algorithms for the identification of patient zero for SIR (susceptible-infected-recovered) and ISS (ignorant-spreader-stifler) dynamic models in networks. 
In Chapter \ref{EPM} the SIR epidemic model as well as ISS model for rumour diffusion are introduced and the algorithms for their simulation in discrete time are presented. 

Chapter \ref{PZ} introduces Direct Monte Carlo and Soft Margin Monte Carlo algorithms \cite{Nino} for indetification of patient zero. 

% [x] Check whether it is possible to implement importance sampling in the Monte Carlo algorithm. 
In Chapter \ref{IS} the technique of importance sampling is  used to create the Sequential Importance Sampling (SIS) algorithm - a new algorithm for single source detection problem. A few reduction based optimization techniques are also introduced. 

% [x] The solution should be appropriated for parallel architectures and implemented in c++. % [] The code is to be documented using comments and should follow the Google C++ Style Guide. 
% [x]The complete application should be hosted on Github under an OSI-approved licence.\textbf{q}
Chapter \ref{Bench} uses benchmark data from \cite{Nino} to assert accuracy of Direct Monte Carlo and Soft Margin implementations. Detailed analysis of SIS algorithm performance is also given.

% [x] For several networks explore the detectability of the source node when source nodes are selected using different centrality measures such as degree, k-core, betweennes centraliy and eigenvector centrality. 
In Chapter \ref{Det} detectability of source node based on parameters of SIR and ISS model is revisited. The detectability analysis for $1-$Barabasi and Erdos Renyi sample graphs is given, as well as the detectability breakdown based on nodes attributes: degree, k-core, betweenness centrality and eigenvector centrality.

\chapter{Complex network structure}
\label{structure}
Most of real networks in social and biological systems are characterized by the similar topological properties: small average path length, high clustering coefficients, fat tailed scale-free degree distributions, degree correlations and  local network structure observable as the presence of communities.

The small world network property is considered to be present when average shortest path length is comparable to logarithm of the network size. 

The \textbf{degree distribution} $P(k)$ defines the probability of choosing the vertex with degree $k$ uniformly from the set of all vertices. The power-law degree distribution of the form $P(k) = Ak^{-\gamma}$ where $2<\gamma<3$ can be fitted to degree distributions of most real networks. The networks with such property are refered to as \emph{scale-free networks}. 


\section{Measures and metrics}
The adjacency matrix $A$ of network with $N$ nodes is a matrix of size $N \times N$ which contains non-zero element $A_{ij}$ if there exist an edge between vertices $i$ and $j$. For the unweighted network all non-zero elements are equal to one.  Note the adjacency matrix is symmetric for undirected graphs and generally asymmetric for  directed graphs.

\subsection{Degree}
\subsection{K-core}



A path in a network is defined as an arbitrary sequence of vertices in which each pair of adjacent vertices are  directly connected in the graph. Number of different paths between two vertices $i$ and $j$ can be computed from the $0-1$ adjacency matrix as $A^k_{ij}$. The number of different cycles of length $k$ can thus be computed as the sum $\sum_{m \in V} A^k_{mm}$ - exactly the trace of the matrix $A^k$. 

A geodesic path is the shortest path between two vertices. Let $d_{ij}$ denote the length of geodesic path from vertex $i$ to vertex $j$. 



 The \textbf{closeness centrality} $C_i$ of vertex $v_i$ is the harmonic mean between the distances of geodesic paths from the vertex $v_i$ to all others:
$$ C_i = \frac{1}{n - 1} \sum_{j \neq i} \frac{1}{d_{ij}}. $$

Let $\sigma_{st}$ be the number of geodesic paths between pairs of vertices $v_s$ and $v_t$ and let $\sigma_{st}(v_i)$ be the number of the geodesic paths $\sigma_{st}$  which pass through the vertex $v_i$. The \textbf{betweenness centrality} is than defined as $$C(v_i) = \sum_{st} \frac{\sigma_{st}(v_i)}{\sigma_{st}} $$.

\textbf{Eigenvector centrality} \cite{bonacich1987power}  captures a concept the vertex is more important if it has more important neighbours. For the given vertex $v$, the eigenvector centrality is defined recursively as $$x_v = \sum_{k} A_{vk} x_k$$  where $A$ is the adjacency matrix. Consequently, the only $x_k$ we are interested in are the neighbours of $x_v$. 

Centralities $x_v$ can be calculated iteratively, starting from arbitrary vector $\vec x(0) = \{x_1(0), x_2(0), \ldots, x_N(0) \}$ using equality $\vec x(t) = A^t \vec x(0)$. 
%TODO continue

\section{Modelling global network structure}
\subsection{Erd\~{o}s-Renyi graph model}
%TODO

\subsection{Barabasi-Albert graph model}
Barabasi-Albert model is the model of evolving a scale-free network, which uses a preferential attachment \cite{Dorogovtsev-2000} property. Starting from $m_0$ isolated vertices, at each time step new vertices with $m$ edges are added to the netowrk $m < m_0$. The new vertex will create an edge to the existing node $v_i$  with the probability proportional to its degree $k_i$. The Barabasi-Albert graph model produces power law distribution $P(k) \approx k ^{-3}$  in the limit of time. The average geodesic path  increases logarithmically with the size of the network.

\chapter{Epidemic process modelling}
\label{EPM}
In the focus of this thesis are heterogeneous epidemic models on the contact network formed by connections between single contacting individuals with transitions of individuals between compartments happening in discrete time steps.

\section{SIR model}
Wide range of diseases that provide immunity to the host can be successfully modelled on a network whose members take 	one of three possible roles at a time: susceptible $(S)$, infected $(I)$ or recovered $(R)$. The diffusion of disease takes place between infected nodes and their susceptible neighbours. An infectious node may also recover from the disease. The recovery grants permanent immunity effectively erasing the member from the contacting network.  The possible events can be represented as 
\begin{equation}
S + I \xrightarrow{p} 2I,\hspace{5mm}  I \xrightarrow{q} R.
\label{transitionsSIR}
\end{equation}

In the SIR model, the infection and recovery process completely determine the epidemic evolution. The transitions (\ref{transitionsSIR}) occur spontaneously and independently in each time step. In discrete-time formulation an infected individual when meeting susceptible will infect the neighbouring susceptible with probability $p$ at each time step. The recovery probability $q$ is the probability infected individual will recover at any time step. The transition probabilities $p$ and $q$ are often assumed constant and equal for all nodes in the same epidemic process. 

\subsection{Simulating the discrete SIR epidemic}

For the contacting network represented by $G(N, L)$ and SIR parameters $p$ and $q$, we are able to simulate one time step of discrete SIR process. Let $s_t$, $i_t$ and $r_t$ denote sets of nodes that are respectively susceptible, infected and recovered after time step $t$. At time step $t$ all previously infected nodes $i_{t - 1}$ will try to infected their susceptible neighbours independently of each other and at the same time. Afterwards the passive recovery process will try to turn them to recovering nodes, each with probability $q$.

This process can be simulated with NaiveSIR algorithm  \cite{NaiveSIR} by putting all the initially infected nodes in the queue. While traversing the nodes, we try to infect each neighbouring node. If the node gets infected, it gets pushed to the queue. SIR simulation of one time step $t$ is described by algorithm \ref{SIRStep}.
\vspace{10mm}

%TODO complexity of NaiveSIR simulation
\begin{algorithm}[H]
% \SetAlgoLined
 \KwData{$\mathbf{G}$ - network, $(p, q)$ -parameters of the SIR model, 
  \textbf{$Iq$} - queue of infected nodes,    
  \textbf{$I$} - bitset of infected nodes, 
  \textbf{$S$} - bitset of susceptible nodes,
  \textbf{$R$} - bitset of recovered nodes}
% \KwResult{New \textbf{$S$}, \textbf{$I$}, \textbf{$R$} -- a result of one time step of the SIR epidemic process}
 infected\_size = $Iq$.size()\;
 \While{(!$Iq$.empty() and infected\_size > 0)}{
  infected\_size $=$ infected\_size $-1$\;
  $u = Iq$.front()\; $Iq$.pop()\;
  \ForEach {$v \in \mathbf{G}$.adj\_list($u$)}{
    \uIf{$v \in S$}{
    let transmission $u \rightarrow v$ occur with probability  $p$\;
    \If{$u \rightarrow v$ \textbf{occured}}{ 
     \textbf{update}($I(v)$ and $S(v)$)\;
     }}
  }
  let transmission $u \rightarrow v$ occur with probability  $q$\;
             \uIf{$u \rightarrow v$ \textbf{occured}}{ 
      \textbf{update}($I(u)$ and $R(u)$)\;
                }\Else{ 
  Iq.insert(u)\;
  }
  }
  \Return \{S, I, R\}
 \label{SIRStep}
 \caption{One time step NaiveSIR simulation on graph $\mathbf{G}$.}
\end{algorithm}

\subsection{Probability of one time step transition}
Probability of one time step transition can be easily evaluated. Let $nei(v)$ indicate a set of all neighbours of node $v$,  $nei(V)$ a set of all neighbours of all nodes in set $V$ and  $nei_V(v) = nei(v) \cap V$ ,  a set of all neighbours of $v$ that are also in $V$. After $k$-th time step of the SIR process, the resulting $i_k$ and $r_k$ were given. At time step $k$, only initially active nodes $i_{k - 1}$ and their neighbours $nei(i_{k - 1})$ actively participate in the epidemic process. For each node $v$ in $i_{k - 1} \cup nei(i_{k - 1})$, one of four independent events may have happened during time step $k$:
\let\labelitemi\labelitemii
\begin{itemize}
\item{$E_1:$ $\mathbf{if}$ $v \not\in i_{k - 1}$ and $v \in i_{k}$\\ \tab node $v$ was infected with probability $1 - (1 - p) ^ {nei_{i_{k-1}}(v)}$}
\item{$E_2:$ $\mathbf{if}$ $v \not\in i_{k - 1}$ and $v \not \in i_{k}$\\ \tab node $v$ was not infected with probability $(1 - p)^{nei_{i_{k-1}}(v)}$}
\item{$E_3:$ $\mathbf{if}$ $v \in i_{k - 1}$ and $v \not\in r_{k}$\\ \tab  node $v$ was recovered with probability $q$}
\item{$E_4:$ $\mathbf{if}$ $v \in i_{k - 1}$ and $v \in r_{k}$ \\ \tab node $v$ was not recovered with probability $1 - q$}
\end{itemize}

Since all the events are independent and the sets corresponding to events $E_{1} - E_{4}$ are disjoint while completely covering the set of active nodes $i_{k - 1} \cup nei(i_{k - 1})$, the  conditional probability of the one time step SIR transition $P(i_k, r_k | i_{k - 1}, r_{k - 1})$ can be calculated as
\begin{equation}
\begin{split}
P(i_k, r_k | i_{k - 1}, r_{k - 1}) =  &\big[\Pi_{v \in E_1} (1 - (1 - p) ^ {nei_{i_{k-1}}(v)})\big]\big[\Pi_{v \in E_2} (1 - p)^{nei_{i_{k-1}}(v)} \big] \\& \cdot
\big[\Pi_{v \in E_3} q\big]\big[ \Pi_{v \in E_4} (1 - q)\big]
\end{split}
\end{equation}
where $nei_{i_{k-1}}(v)$ is the set of all neighbours of nodes $i_{k - 1}$ -- the ones infected at the beginning of time step $k$.

\section{Epidemic models as social contagion processes}
Even though infectious diseases represent the central focus of epidemic modelling, the model where an individual is strongly influenced by the interaction with peers is present in several other domains, especially in social context in the diffusion of information, the propagation of rumour and adoption of innovation or behaviours.  Since the social contacts can in these domains generate epidemic-like outbreaks, simple models for information diffusion are epidemic models modified to specific features of social contagion. The crucial difference to pathogen spreading is that transmission of information involves intentional acts by both the sender and the receiver and it is often beneficial for both participants.

\subsection{Rumour spreading with the ISS model}
The need to study rumour spreading presents itself in a number of important technological and commercial applications where it is desirable to spread the "epidemic" as fast and as efficient as possible.  In examples such as rumour based protocols for resource discovery and marketing campaigns that use rumour like strategies (viral marketing)  the problem translates to design of an epidemic algorithm in such a way that the given information reaches as much nodes as possible, similarly to a rumour.

Models for rumour spreading are variants of the SIR model in which the recovery process does not occur spontaneously, but rather is a consequence of interactions. The modification mimics the idea it is worth spreading the rumour as long as it is novel for the recipient. This process can be formalized as a model where each of  $N$ members of the contacting network can be a part of one of three compartments: \textbf{ignorant (S), spreader (I) and stifler (R)}. Ignorants have not heard the rumour and are susceptible to being informed. Spreaders are actively spreading the rumour, while stiflers know about the rumour but they're not  spreading it.
 
The spreading process evolves by direct contacts of spreaders with others in the population. When a spreader meets an ignorant, the latter turns into a new spreader with probability $a$. When a spreader meets another spreader or a stifler, the former spreader turns into stifler with probability $b$ and the latter remains unchanged. This model is known as the ISS model (Ignorant-Spreader-Stifler) \cite{Moreno2004}. The possible events can be represented as 
\begin{equation*}
S + I \xrightarrow{\alpha} 2I,\hspace{5mm}  R + I \xrightarrow{\beta} 2R,  \hspace{5mm} 2I \xrightarrow{\beta} R + I.
\end{equation*}

Since we examine the spreading process in discrete time, at each time step, the current spreaders try to interact with their neighbours. A modification of the NaiveSIR algorithm for rumour spreading simulation of one time step $t$ is described by algorithm \ref{ISSStep}.

\begin{algorithm}[H]
 \KwData{$\mathbf{G}$ - network, $(a, b)$ - parameters of the ISS model, 
  \textbf{$Iq$} - priority queue of spreader nodes,    
  \textbf{$I$} - bitset of spreader nodes, 
  \textbf{$S$} - bitset of ignorant nodes,
  \textbf{$R$} - bitset of stifler nodes }
% \KwResult{New \textbf{$S$}, \textbf{$I$}, \textbf{$R$} --  a result of one time step in the rumour spreading process}
 stifler\_size = $Iq$.size()\;
 \While{(!$Iq$.empty() and stifler\_size > 0)}{
  stifler\_size $=$ stifler\_size $-1$\;
  $u = Iq$.front()\; $Iq$.pop()\;
  %\If{$u \not\in I$}{\textbf{continue}\;}
  %\textbf{shuffle}($\mathbf{G}$.adj\_list($u$)\;
  \ForEach {$v \in \mathbf{G}$.adj\_list($u$)}{
    \uIf{$v \in S$}{
    let transmission $u \rightarrow v$ occur with probability  $a$\;
    \If{$u \rightarrow v$ \textbf{occured}}{ 
     \textbf{update}($I(v)$ and $S(v)$)\;
     }}\uElse{
         let transmission $u \rightarrow v$ occur with probability  $b$\;
             \If{$u \rightarrow v$ \textbf{occured}}{ 
      \textbf{update}($I(u)$ and $R(u)$)\;
      \textbf{break}\;
                }}
  }
  \If{$u \in$ I}{
  Iq.insert(u)\;
  }
  }
  \Return \{S, I, R\}
 \label{ISSStep}
 \caption{One time step simulation of rumour spreading under ISS model on graph $\mathbf{G}$.}
\end{algorithm}

\chapter{Patient zero -- single source epidemic detection}
\label{PZ}

In accordance with \cite{Nino}, we will focus on a patient zero problem given snapshot of population at time $T$ and complete knowledge of underlying contacting network modelled by $G$ with the assumption the epidemic has started from a single source node and that it is governed by the SIR process with known $p$ and $q$.  The estimators proposed in \cite{Nino} will be presented in this chapter, while the newly proposed estimators based on importance sampling  technique will be presented in the next chapter.

Let random vector $\vec S = (S(1), \ldots, S(N))$ indicate the nodes that got infected up to a predefined temporal threshold $T$ with SIR$(p, q)$ epidemic process on network $G$ with $N$ nodes. $S(i)$ is a Bernoulli random variable with the value $1$ if the node $i$ got infected before time $T$ from the start of the epidemic process. We observe one realization $\vec s_*$ of $\vec S$ and we want to infer which nodes from the set of infected or recovered nodes $\Theta = \{\theta_1, \theta_2, \ldots, \theta_m \}$ are the most likely to be the source of the epidemic process.  The finite set of possible source nodes $\Theta$ is determined by realization $\vec s_*$.

A maximum aposteriori probability estimate (MAP) is the node with the highest probability for being the source of the epidemic spread for a given realization $\vec s_*$: 
\begin{equation}
\hat{\theta}_{MAP} = \text{arg max}_{\theta_i \in \Theta} P(\Theta = \theta_i | \vec{S} = \vec s_*)
\label{MAP}
\end{equation}
By applying the Bayes theorem with equal apriori probabilities $P(\Theta = \theta_i)$, probability in (\ref{MAP}) can be expressed as 
\begin{equation}
\begin{aligned}
P(\Theta = \theta_i | \vec{S} = \vec s_*) &= \frac{P(\vec S = \vec s_* | \Theta = \theta_i) P(\Theta = \theta_i)}{\sum_{\theta_k \in \Theta} P(\vec S = \vec s_* | \Theta = \theta_k) P(\Theta = \theta_k)} \\ &= \frac{P(\vec S = \vec s_* | \Theta = \theta_i)}{\sum_{\theta_k \in \Theta} P(\vec S = \vec s_* | \Theta = \theta_k)}.
\end{aligned}
\label{MAP_pravi}
\end{equation}

\section{Direct Monte Carlo estimator}

\textbf{The integration problem}
\begin{equation}
\mathbf{E_f[\textit{h(X)}]} = \int_{X} h(x) f(x) dx
\label{exp}
\end{equation} can be estimated using Monte Carlo technique with $n$ samples $X_1, \ldots, X_n$ generated from the density $f$ as the empirical average 
\begin{equation}
h_n = \frac{1}{n} \sum_{j = 1}^{n} h(X_j).
\end{equation}
The convergence of $h_n$ towards $\mathbf{E_f[\textit{h(X)}]}$ is assured by the Strong Law of Large Numbers.

Inferring the probability  $P(\vec R = \vec r_* | \Theta = \theta_i)$ up to multiplicative constant is an integration problem equivalent to expectation of Kronecker delta function
 $\delta(\vec R) = 1\{\vec R = \vec r_{*}\}$ where $\vec R$ is a random variable governed by probability $P(\vec R  | \Theta = \theta_i)$. 
 Let $m_i$ denote estimation of expected number of hits for a fixed source $\theta_i$ estimated using Monte Carlo technique:
\begin{equation}
 m_i = \sum_{j = 1}^{n} 1\{\vec R_i = r_{*}\}
\label{mi}
\end{equation}
where $\vec R_i$ are drawn from $P(\vec R | \Theta = \theta_i)$. With Direct Monte Carlo estimator $m_i$ is obtained by simulating epidemic process up to time $T$ starting from a single infected node $\theta_i$ and checking whether the generated realization $\vec R_i$ coincides with $\vec r_*$. Since $m_i$ is estimation of $P(\vec R = \vec r_* | \Theta = \theta_i)$ up to multiplicative constant $1/n$ for all $\theta_i \in S$, we derive Direct Monte Carlo MAP estimator based on the estimation of probability $P(\Theta = \theta_i | \vec R = \vec r_*)$ combining (\ref{mi}) with (\ref{MAP_pravi}): 
\begin{equation}
\hat{P_i}^n = \hat{P}(\vec R = \vec r_* | \Theta = \theta_i) = \frac{m_i}{m}
\end{equation}
where $m = \sum_{j = 1}^{n} m_j$ .

If the size of realization $\vec r_*$ is big, the number of simulations required to obtain reliable estimations can be large. Since the estimations for different source node candidates are independent, the computations can be parallelised.  

Additionally for the SIR model, a prunning mechanism can be incorporated. If a sampling simulation infects a node that was not infected during the target epidemic  represented by the realization $r_*$, it is safe to stop the sampling simulation prior to ending time $T$ and call a sample unequal to target realization $r_*$.

The accuracy of direct Monte Carlo approximations are controlled by convergence conditions. Upon estimating two source PDF's, $\hat{P}_i^n$ and $\hat{P}_i^{2n}$ with $n$ and $2n$ independent simulations, respectively. The PDF's are said to converge when the following conditions are satisfied:
\begin{equation}
| \hat{P}_i^{2n} - \hat{P}_i^{n} | / \hat{P}_{2n} \leq c, \hspace{1mm} | \hat{P}_i^{2n} - \hat{P}_i^{n} | \leq c \hspace{5mm} \forall \theta_i \in S.
\label{DMCConv}
\end{equation}

%TODO cite alg somewhere
\vspace{5mm}
\begin{algorithm}[H]
\ \KwData{$\mathbf{G}$ - network, $(p, q)$ - parameters of the SIR process, 
  $\vec r_*$ - target realization,    
  $T$ - temporal threshold, 
  $\theta_i$ - proposed initial source node,
  $n$ - number of simulations}
 \KwResult{ $m_i$ - estimated expected number of realizations started from $\theta_i$ and completely corresponding to $r_*$ }
 $m_i = 0$\; 
 \For{$s = 1$ to $n$} {
   \For{$t = 1$ to $n$} {
    Run one SIR simulation $(p, q, \theta_i)$ for time step $t$ and obtain  
    $r_i^t$.\; 
    \If{$\hspace{3mm} \exists j \in N: (r_i^t(j) == 1$ \textbf{and} $r_*(j) == 0)$} {
   \textbf{break}\;
    }
   }
    \If{$r_i^T$ \textbf{equals} $r_*$} {
  $m_i = m_i + 1$\;
 }
 }
  \Return $m_i$
 \label{DMC_lag}
 \caption{Direct Monte Carlo estimation $m_i$ of expected number of realization hits completely corresponding to $r_*$  for a fixed source $\theta_i$.}
\end{algorithm}
  \vspace{5mm}


\section{The Soft Margin estimator}

Let $\vec r^{(j)}_{\theta_i}$ denote $j$-th sample (outcome) obtained by Monte Carlo simulation of contagion process with source node $\theta_i$ and duration of $T$ time steps. $\vec r^{(j)}_{\theta_i}$ is one realization of random binary vector $\vec R_{\theta_i}$ that describes the outcome of epidemic process. A similarity measure $\varphi : (\vec R_{\theta_i} \times \vec R_{\theta_i}) \rightarrow [0, 1 ]$  can  be defined between any two  realizations of $\vec R_{\theta_i}$, for example,  as the Jaccard similarity function:
\begin{equation}
\varphi(\vec r_1, \vec r_2) = \frac{\vec r_1 \cap \vec r_2}{\vec r_1 \cup \vec r_2 } = \frac{\sum_{j = 1}^{N} (r_1(j) = 1 \hspace{2mm} \text{and} \hspace{2mm} r_2(j) = 1) }{\sum_{j = 1}^{N} (r_1(j) = 1 \hspace{2mm} \text{or} \hspace{2mm} r_2(j) = 1) }.
\label{JaccardPHI}
\end{equation}

Moreover, we  can define a discrete random variable $\varphi(\vec r_*, \vec R_{\theta})$ that measures the similarity between a fixed realization $r_*$ and a random realization from $\vec R_{\theta}$. Let PDF of that random variable be $f_\theta(x)$ where $x = \varphi(\vec r_*, \vec r_{\theta})$. Unbiased estimator for PDF can be obtained with Monte Carlo method from $n$  samples as 
\begin{equation}
f_{\theta}(x) = \int_0^1 p_k \delta(x - x_k)dx \approx \frac{1}{n} \sum_{i = 1}^{n} \delta(x - \varphi(\vec r_*, \vec r_{\theta} ^{(i)}))
\label{Fapprox}
\end{equation}
where $\delta(x)$ denotes the Dirac delta function. In the integral we observe a series of probabilities $p_1, p_2, \ldots, p_d$ corresponding to each realization of $\varphi(\vec r_*, \vec R_{\theta})$. With Monte Carlo method, we observe the PDF definition as an integration problem and sample from this discrete distribution $\{p_1, \ldots, p_d\}$ to obtain an estimate.

\textbf{The Soft Margin estimator} is defined as 
\begin{equation}
\hat{P}(\vec R = \vec r_* | \Theta = \theta) = \int_0^1 w_a(x)\hat{f}_\theta(x)dx
\label{SMF1}
\end{equation}
where $w_a(x)$ is a weighting function and $f_\theta(x)$ is the PDF function of the random variable $\varphi(\vec r_*, \vec R_\theta)$. For $w_a(x)$, \cite{Nino} proposed a Gaussian weighting form $w_a(x) = e^{-(x -1)^2 / a ^2}$. In this way, the problem definition was altered to estimating the number of relizations with similarity in the interval defined by Gaussian function $w_a(x)$ around $\varphi = 1$ , whereas estimating the number of realizations with similarity strictly equal to $\varphi = 1$ as with Direct Monte Carlo method. In the limit where $a \rightarrow 0$ unbiased direct Monte Carlo estimate is obtained.

The Soft Margin formula (\ref{SMF1}) can be further simplified combining with (\ref{Fapprox}):
\begin{equation}
\begin{aligned}
\hat{P}(\vec R = \vec r_* | \Theta = \theta) &= \int_0^1 w_a(x)\hat{f}_\theta(x)dx \\ &=
\int_0^1 w_a(x) \frac{1}{n} \sum_{i = 1}^{n}\delta(x - \varphi(\vec r_*, \vec r_{\theta}^{(i)})) dx,
\end{aligned}
\end{equation}
and further by using the property of delta distribution $\int_{-\infty}^{\infty} f(x)\delta(x - b)dx = f(b)$:
\begin{equation}
\begin{aligned}
\hat{P}(\vec R = \vec r_* | \Theta = \theta) &= \frac{1}{n} \sum_{i = 1}^{n} \int_0^1 w_a(x) \delta(x - \varphi(\vec r_*, \vec r_{\theta}^{(i)})) dx  \\ &= \frac{1}{n} \sum_{i = 1}^{n}  w_a(\varphi(\vec  r_*, \vec r_\theta^{(i)})) \\ &= \frac{1}{n} \sum_{i = 1}^{n} e ^{\frac{(\varphi_i -1)^2}{a ^2}}.
\end{aligned}
\end{equation}

Note that it's not needed to determine constant $a$ in advance. The parameter $a$ can be chosen as the infinum of the set of parameters for which the source probability distribution estimates $\hat{P_a}(\Theta = \theta_i | \vec R = \vec r_*)$ have converged under the convergence property from (\ref{DMCConv}). 

%TODO cite alg somewhere
\vspace{5mm}
\begin{algorithm}[H]
 \KwData{$\mathbf{G}$ - network, $(p, q)$ - parameters of the SIR process, 
  $\vec r_*$ - target realization,    
  $T$ - temporal threshold, 
  $\theta_i$ - proposed initial source node,
  $n$ - number of simulations,
  $a$}
 \KwResult{ $\hat{P_a}(\vec R = \vec r_* | \Theta = \theta_i)$}
 \For{$s = 1$ to $n$} {
    Run SIR simulation $(p, q, \theta_i)$ for $T$ time steps and obtain  
    $\vec r_i^T$\;
    Calculate and save $\varphi_i = \varphi(\vec r_*, \vec r_i^T)$\;
 }
  Calculate $\hat{P}(\vec R = \vec r_* | \Theta = \theta_i) = \frac{1}{n} \sum_{i = 1}^{n} e ^ {\frac{-(\varphi_i -1)^2}{a^2}}$\;
  \Return $\hat{P}(\vec R = \vec r_* | \Theta = \theta_i)$
 \label{DMC_lag}
 \caption{Soft Margin estimation of $P(\vec R = \vec r_* | \Theta = \theta_i)$ for a fixed source $\theta_i$.}
\end{algorithm}
 \AlgoDontDisplayBlockMarkers%

\section{Time complexity of estimators}
%TODO 

\chapter{Sequential Importance Sampling estimator}
\label{IS}
%TODO [] results for SIS with prunning, 
%     [] add another version of prunning, 
%     [] time accuracy correlation, 
%     [] experimental complexity / duration results, a
%     [] add soft margin to SIS, 
%     [] Metropolis - Hastings
%     [] Incorporte vc2 and/or ESS in the result analysis
\section{Importance sampling}

\textbf{Importance sampling} is a technique for estimating properties of a particular distribution with samples generated from a different distribution than the one of interest. The technique is used with Monte Carlo method as a variance reduction technique since we usually choose to sample from the distribution that is biased towards the realizations that have more impact on the parameters being estimated.

In other words, the method of importance sampling is estimation of integration problem (\ref{exp}) based on generating a sample $X_1, \ldots, X_m$ from a given biased distribution $\textit{g}$ when in fact the samples $X_i$ come from the target distribution $\textit{f}$:
\begin{equation}
\mathbf{E_f[\textit{h(X)}]} = \int_{X} h(x) f(x) dx = \int_{X} h(x) \frac{f(x)}{g(x)} g(x) dx \approx \frac{1}{m} \sum_{j = 1}^{m} \frac{f(X_j)}{g(X_j)} h(X_j).
\end{equation}
By choosing to sample from the biased distribution $g$, we are left with the extra weight $w^{(i)} = \frac{f(X_j)}{g(X_j)}$ from the integral that corrects the bias of the estimation. The new estimator converges whatever the choice of distribution $g$, as long as $supp(g) \supset supp(f)$\footnote{$supp(g) = \{x | g(x) \neq 0\}$}. %TODO cite

Note the estimation can be done with unbiased estimate,
\begin{equation}
\frac{1}{m}\sum_{i=1}^{m}w^{(i)}h(\mathbf{x}^{(i)}),
\end{equation}
or with a weighted estimate
\begin{equation}
\frac{\sum_{i=1}^{m}w^{(i)}h(\mathbf{x}^{(i)})}{ \sum_{i=1}^{m}w^{(i)}}.
\label{wei_est}
\end{equation}
When using the weighted estimate, we only need to know the ratio $f(\mathbf{x})/g(\mathbf{x})$ up to a multiplicative constant. Although inducing a small bias, the weighted estimate often has a smaller mean squared error than the unbiased one.

\subsection{Measuring the quality of importance distribution}
By properly choosing $g(\cdot)$, one can reduce the variance of the estimate substantially. In order to make the estimation error small, one wants to choose $g(\mathbf{x})$ as close in shape to $f(\mathbf{x})h(\mathbf{x})$ as possible. The efficiency of such method is difficult to measure. Effective sample size (ESS) is commonly used to measure how different the importance distribution is from the target  distribution. %TODO Explain in more detail.

Suppose we have $m$ independent samples generated from $g(\mathbf{x})$. The ESS of this method is defined as 
\begin{equation}
\text{ESS}(m) = \frac{m}{1 + var_g[w(\mathbf{x})]}.
\end{equation}  
The variance here is estimated as a square of the coefficient of variation of the weights:
\begin{equation*}
cv^2 = \frac{\sum_{j=1}^{m} (w^{(j)} - \bar{w})^2}{(m - 1)\bar{w}^2}
\end{equation*}
where $\bar{w}$ is sample average of the $w^{(j)}$. The ESS measure of efficiency can be partially justified by the delta method \cite{Liu}. %TODO more

\subsection{Rejection control and weighting}
When applying importance sampling, one often produces random samples with very small importance weights because of a less than ideal importance density. The following technique for combining rejection and importance weighting can be used. 

Suppose we have drawn samples $\mathbf{x}^{(1)}, \ldots, \mathbf{x}^{(m)}$ from $g(\mathbf{x})$. Let $w^{(j)} =  \frac{f(\mathbf{x}^{(j)})}{g(\mathbf{x}^{(j)})}$. We can conduct the following operation for any given threshold value $c > 0$:
\vspace{5mm}

\begin{algorithm}[H]
\textbf{Rejection Control (RC)}\\
\For{$j = 1, \ldots, m,$}{
  accept $\mathbf{x}^{(j)}$ with probabilty 
$r^{(j)} =  \min  \Big \{ {1, \frac{w^{(j)}}{c}} \Big \}$\\
 \If{$\mathbf{x}^{(j)}$ is accepted}{ weight $w^{(j)}$ is updated to $w^{(*j)} = q_c w^{(j)} / r^{(j)}$, where 
\begin{equation}
q_c = \int \min  \Big \{ {1, \frac{w^{(j)}}{c}} \Big \} g(\mathbf{x}) d\mathbf{x}
\label{qc}
\end{equation}}
}
 \caption{\textbf{Rejection Control (RC)}}
 \label{RC}
\end{algorithm}
\vspace{5mm}

Since the constant $q_c$ is the same for all accepted samples, it is not needed for the  evaluation of the weighted estimate in (\ref{wei_est}). Nevertheless, it can be unbiasedly estimated \cite{Liu} from the sample as 
\begin{equation}
\hat{p}_{c} = \frac{1}{m} \sum_{j = 1}^{m} \min  \Big \{ {1, \frac{w^{(j)}}{c}} \Big \}.
\end{equation}

With this technique we are adjusting the importance density $g$  in light of current importance weights. The new importance density $g^{*}(\mathbf{x})$ is expected to be close to the target distribution $f(\mathbf{x})$.

After applying rejection control, we will typically have fewer than $N$ samples. More samples can be drawn from either $g(x)$ or $g^{*}(x)$ (via rejection control) to make up for the rejected samples. 

\section{Sequential importance sampling}
Since it is not trivial to design a good importance sampling distribution, especially for high dimensional problems, one may build up the importance density sequentially. Suppose we can decompose $\mathbf{x}$ as $\mathbf{x} = (x_1, \ldots, x_d)$ where each of the $x_j$ may be multidimensional. Then our importance distribution can be constructed as 
\begin{equation}
g(\mathbf{x}) = g_1(x_1) g_2(x_2 | x_1) g_3(x_3 | x_1, x_2) \ldots g_d(x_d | x_1, \ldots, x_{d - 1})
\end{equation}
by which we hope to obtain some guidance from the target density while building up the  importance density. We can then rewrite the target density as 
\begin{equation}
f(\mathbf{x}) = f_1(x_1) f_2(x_2 | x_1) f_3(x_3 | x_1, x_2) \ldots f_d(x_d | x_1, \ldots, x_{d - 1})
\end{equation}
and the weights as 
\begin{equation}
w(\mathbf{x}) = \frac{f_1(x_1) f_2(x_2 | x_1) f_3(x_3 | x_1, x_2) \ldots f_d(x_d | x_1, \ldots, x_{d - 1})}{g_1(x_1) g_2(x_2 | x_1) g_3(x_3 | x_1, x_2) \ldots g_d(x_d | x_1, \ldots, x_{d - 1})}
\end{equation}
which suggests a recursive monitoring and computing of importance weight:
\begin{equation}
w_t(\mathbf{x}_t) = w_{t - 1}(\mathbf{x}_{t - 1})\frac{f(x_t | \mathbf{x}_{t - 1})}{g(x_t | \mathbf{x}_{t - 1})}.    
\end{equation}
At the end, $w_d$ is equal to $w(\mathbf{x})$. By using the recursive process we can stop generating further components of $\mathbf{x}$ if the partial weight derived from the sequentially generated partial sample is too small and we can take advantage of $f(x_t | \mathbf{x}_{t - 1})$ in designing $g_t(x_t | \mathbf{x}_{t - 1})$.

The sequential importance sampling method can then be defined as follows:

\vspace{5mm}
\begin{algorithm}[H]
\caption{SIS Step}
$1.$ Draw $X_t=x_t$ from $g_t(x_t | \mathbf{x_{t-1}})$, and let $\mathbf{x}_t = (\mathbf{x}_{t - 1}, x_t)$.\\
$2.$ Compute $w_t(\mathbf{x}_t) = w_{t - 1}(\mathbf{x}_{t - 1})\frac{f(x_t | \mathbf{x}_{t - 1})}{g(x_t | \mathbf{x})_{t - 1})}$.\\
\label{SIS_step}
\end{algorithm}
\vspace{5mm}
When we observe that $w_t$ is getting too small, we can choose to reject the sample halfway and restart again.

\subsection{Improving the SIS procedure - rejection control}
\label{SIS_RC}
The rejection control method \textbf{RC} can be applied dynamically to improve the  \textbf{SIS} scheme. Suppose a sequence of "check points," $0 < t_1 < t_2 < \ldots < t_k \leq d$ and a sequence of threshold values $c_1, \ldots, c_k$, are given in advance. 
\begin{itemize}
\item{At each check point $t_j$ start \textbf{RC}$(t_k)$ with the threshold value $c = c_j$ . If the partial sample $(x_1, \ldots, x_{t_j})$ has a weight $w_{t_j}$, we accept it with probability $\min \{ 1, w_{t_j}/c_j \}$. If accepted, replace its weight by $w^{*}_{t_j} = \max \{ w_{t_j}, c_j \}$}
\item{For each rejected partial sample, restart from the beginning again and let it pass through all the check points.}
\end{itemize}

\subsection{Improving the SIS procedure - resampling}
When the system grows, the variance of the importance weights $w_t$ increases. After a certain number of steps, many of the weights become very small and a few very large. In that situation one may use a \textbf{resampling} strategy. 

Suppose at step $t$ we have a collection of $m$ partial samples of length $t, S_t = \{ \mathbf{x}_{t}^{(j)}, j = 1, \ldots, m \}$ which are properly weighted by the collection of weights $W_t = \{w_t^{(j)}, j = 1, \ldots, m\}$ with respect to the density $g$.

The resampling step is done on the existing partial sample set before expanding with the \textbf{SIS step}.

\subsubsection{Simple random sampling}
\begin{itemize}
\item{Sample a new set of partial samples, $S'_t$ from $S_t$ according to the weights $w_t^{(j)}$.}
\item{Assign equal weights, $W_t / m$, to the samples in $S_t'$ where $W_t = w_t^{(1)} + \ldots + w_t^{(m)}$}
\end{itemize}

\subsubsection{Residual resampling}
\begin{itemize}
\item{Retain $k_j = [mw_t^{(*j)}]$ copies of $\mathbf{x}_t^{(j)}$ where $w_t^{(*j)} = w_t^{(j)} / W_t$ and $j = 1, \ldots, m$. Let $m_r = m - k_1 - k_2 - \ldots - k_m$.}
\item{Obtain $m_r$ draws from $S_t$ with probabilities proportional to $mw_t^{(*j)} - k_j$, $j = 1, \ldots m$.}
\item{Reset all the weights to $W_t / m$.}
\end{itemize}

Residual sampling dominates the simple random sampling in having smaller Monte Carlo variance. 

\subsubsection{General resampling strategy}
\begin{itemize}
\item{For $j = 1, \ldots m$:
\subitem{Draw $\mathbf{\tilde{x}}_t^{(j')}$ independently from the current sample $\{ \mathbf{x}_t^{(j)}, j = 1, \ldots, m   \}$ according to the probability vector $(a^{(1)}, \ldots, a^{(m)})$. Suppose we obtain $\mathbf{\tilde{x}}^{(j')} = \mathbf{x}_t^{(j)}$.
}
\subitem{A new weight $\tilde{w}_t^{(j')} = w_t^{(j)} / a^{(j)}$ is assigned to this sample.}
}
\end{itemize}
The new set thus formed is also properly weighted by new weights with respect to $g$. Because the role of resampling is to prune away "bad" samples and to split the good ones, we should choose $a^{(j)}$ as a monotone function of $w_t^{(j)}$. We can choose $a^{(j)}$ to reflect certain future trend, balance between the need of diversity and the of focus (giving more presence to the samples with large weights) etc.  A generic choice is $a^{(j)} = [w_t^{(j)}]^{\alpha}$ with $0 < \alpha \leq 1$ that can vary according to the variance of $w_t$.

\vspace{10mm}

\subsubsection{Resampling schedule}
The resampling step tends to result in a better group of anecestors so  as to produce better descendants. The success of resampling, however, relies heavily on the Markovian structure among the state variables $x_1, x_2, \ldots$. Given the realization of $x_t$, the next variable $x_{t + 1}$ is statistically independent of all the previous states $\mathbf{x}_{t - 1}$. If the resampling from set $\{ \mathbf{x}_{t - 1}^{(j)}, j = 1, \ldots m\}$ is not equivalent to resampling from $\{ x_{t - 1}^{(j)}, j = 1, \ldots, m\}$, the set of the "current state" frequent resampling will rapidly impoverish diversity of the partial samples produced earlier. When no simple Markovian structure is present, frequent resampling generally gives bad results.

For this reason, it is desirable to prescribe a schedule for the resampling to take place. The resampling schedule can be either deterministic or dynamic. When the schedule is dynamic, some small bias may be introduced.

With a deterministic schedule, we conduct resampling at time $t_0, 2t_0, \ldots,$ where $t_0$ is given in advance. In a dynamic schedule, a sequence of thresholds $c_1, c_2, \ldots,$ are given in advance. We monitor the coefficient of variation of the weights $cv_t^2$ and invoke the resampling step when event $cv_t^2 > c_t$ occurs. A typical sequence of $c_t$ can be $c_t = a + bt^\alpha$.

Increasing $c_t$ after each SIS step makes sense since it can be shown that as the system evolves, $cv_t^2$ increases stochastically \cite{Kong94}.

\vspace{10mm}
\textbf{Resampling scheme}
\begin{itemize}
\item{Check the weight distribution by performing one of the methods at time $t$. Resample if needed.}
\item{Invoke an SIS step. Set $t = t + 1$.}
\end{itemize}


\subsection{Partial rejection control}
As $t$ increases, $cv_t^2$ increases stochastically and the weights $w_t$ typically become skewed. As a consequence, many samples  will have minimal impact on the final estimation. It is thus desirable to prune them away at an earlier stage. In \ref{SIS_RC} we have seen how the rejection control method can be used to achieve pruning without creating bias or correlations. However, the implementation of full rejection control requires that we make up the lost samples by restarting from stage $1$ and passing through all the intermediate rejection steps. Instead of employing the full rejection control, we can follow a more pratical, partial rejection method that combines both rejection and resampling. 


\textbf{Partial rejection control}
\begin{itemize}
\item{At each check point $t_k$, start \textbf{RC}$(t_k)$ from \ref{RC} with threshold value $c=c_k$. If the stream $\mathbf{x}_{t_k}^{(j)}$ passes this check point, we proceed with standard SIS replacing the old weight with $w_{t_k}^{(*j)} = \max \{w_{t_k}^{(j)}, c_k\}$.}
\item{When rejected, go back to check point $t_{k - 1}$ to draw a sample $x_{t_{k - 1}}^{(j)}$ from the sample pool $S_{t_{k - 1}}$, with probability proportional to $w_{t_{k - 1}}^{(j)}$. Reset its weight to $\bar{w}_{t_{k - 1}}$ and make a SIS step. If the new sample formed in this way pass the check point $t_k$, then its weight is set as $w_{t_k}^{(*j)} = \max \{w_{t_k}^{(j)}, c_k \} $}
\item{Reset all the weights to $w_{t_k}^{(j)} = \hat{p}_c w_{t_k}^{(*j)}$.}
\end{itemize}

\section{Sequential Importance sampling estimator}
%TODO continue
Given snapshot $r_*$ that holds all infected nodes up to time $T$, we want to determine the probability  of epidemic starting in node $\theta_i$, $P(\theta_i | \vec S = \vec s_*)$ where $\vec R$ is a random variable whose one realization is $\vec s_*$.   Since all the apriori probabilities $P(\theta_i)$ are the same, we can approximate aposteriori probabilities $P(\vec S = \vec s_* | \theta_i)$ and use then to determine $P(\theta_i | \vec S = \vec s_*)$:
\begin{equation*}
\hat{P}(\theta_i | \vec S = \vec s_*) = \frac{\hat{P}(\vec S = \vec s_* | \theta_j)}{\sum_j \hat{P}(\vec S = \vec s_* | \theta_j)}
\end{equation*}

The aposteriori probabilities are estimated with Direct Monte Carlo and Soft Margin method up to a multiplicative constant. This can also be done using Sequential Importance Sampling.

First note the \textbf{SIS step} as defined in \ref{SIS_step} is based on the densities of a complete history of the process, or at time $t$, all the process steps up to time $t$. The target density is thus the join probability of all the steps taken in the process. Since we are only interested  in the final realization, it makes sense to use target (and importance) density of the form:
\begin{equation*}
f(s_t) = f_1(i_1, r_1) f_2(i_2, r_2 | i_1, r_1)  f_3(i_3, r_3 | i_2, r_2)  \ldots  f_t(i_t, r_t | i_{t - 1}, r_{t - 1})
\end{equation*}
\begin{equation*}
g(s_t) = g_1(i_1, r_1) g_2(i_2, r_2 | i_1, r_1) g_3(i_3, r_3 | i_2, r_2) \ldots  g_t(i_t, r_t | i_{t - 1}, r_{t - 1}))
\end{equation*}
where $i_t$ denotes a vector of infected nodes at time $t$, and $r_t$ denotes a vector of recovered nodes at time $t$. Note that $(i_t, r_t) = s_t$. The sequence $(i_1, r_1), (i_2, r_2), (i_3, r_3), \ldots, (i_t, r_t)$ is connected with a SIR step.

\subsection{Modelling the target distribution}
We can evaluate the partial target density $f_k(i_k, r_k | i_{k - 1}, r_{k - 1})$ in closed form. Denote as $nei(v)$ a set of all neighbours of node $v$, as $nei(I)$ a set of all neighbours of all nodes in $I$ and as $nei_I(v) = nei(v) \cap I$ a set of all neighbours of $v$ that are in $I$. One step SIR simulation was conducted and the resulting $i_k$ and $r_k$ were given. For each node $v$ in $i_{k - 1} \cup nei(i_{k - 1})$, $1$ of $4$ independent events may have happened:
\begin{itemize}
\item{if $v \in i_{k - 1}$ and $v \in r_{k}$ node $v$ was recovered with probability $q$.}
\item{if $v \in i_{k - 1}$ and $v \not\in r{k}$ node $v$ was not recovered with probability $1 - q$.}
\item{if $v \not\in i_{k - 1}$ and $v \in i_{k}$ node $v$ was infected with probability $1 - (1 - p) ^ {nei_{i_{k-1}}(v)}$}
\item{if $v \not\in i_{k - 1}$ and $v \not \in i_{k}$ node $v$ was not infected with probability $(1 - p)^{nei_{i_{k-1}}(v)}$}
\end{itemize}

Since all the events are independent, the target density is of the form
\begin{equation*}
f_k(i_k, r_k | i_{k - 1}, r_{k - 1}) = \big[\Pi_{v \in E_1} q\big]\big[ \Pi_{v \in E_2} (1 - q)\big]\big[\Pi_{v \in E_3} (1 - (1 - p) ^ {nei_{i_{k-1}}(v)})\big]\big[\Pi_{v \in E_4} (1 - p)^{nei_{i_{k-1}}(v)} \big].
\end{equation*}

\subsection{Modelling the importance distribution}
With our sequential sampling procedure we will try to estimate the number of realizations at time $T$ that are equal to $s*$. The importance density will be biased towards that goal. Since we are building the final densities sequentially, our biased sampling must sample reasonably enough at each step (it must not be to "slow" or too "fast"), especially since it is not certain what samples at mid steps are valuable too us as we might perform some sort of  resampling or reduction.

It is certain, however, we do not want to infect the nodes that were never infected in the snapshot $s*$ and we can safely use $SIR(p = 1, q)$ at the last \textbf{SIS} step. That leads us to the biased density similar to $f$ where only the nodes in $s*$ are eligible for events $E_3, E_4$ and it holds $p = 1$ when $k = T$.

It may be reasonble to increase $p$ at each step of \textbf{SIS} procedure but it is not clear when this should be done. Additionally, one might want to use a resampling or a rejection technique based on $vc^2$ for simulations with many  \textbf{SIS} steps. This has to be done carefully too since our target event is rare and weights $w$ are naturally small.



\chapter{Benchmark dataset}
\label{Bench}

\citet{Nino} provided a dataset of SIR realizations along with their estimations obtained with Direct Monte Carlo for $4$ classes of SIR parameters: $A = (p = 0.3, q = 0.3, T =0 .5), B = (p = 0.3, q = 0.7, T = 0.5), C = (p = 07, q = 0.3, T = 5)$ and  $D = (p = 0.7, q = 0.7, T = 5)$. The benchmark dataset contains $160$ such realizations on the grid of size $30x30$.
 Their estimations obtained with Direct Monte Carlo were held under convergence condition $|P_{ML}^{2n} - P_{ML}^{n}| / P_{ML}^{2n}| \leq 0.05$  and $|P_i^x - P_i^{2x}| \leq 0.05$ for all other nodes. %TODO where P_{ML} je...

\section{Correctness of the Direct Monte Carlo implementation}

\section{Correctness of the Soft Margin implementation}
For the Soft Margin estimator we use the following convergence condition: 
\begin{equation*}
|\hat{P}_a^{n}(\Theta = \theta_{MAP} | \vec{R} = \vec{r}_*) - \hat{P}_a^{2n}(\Theta = \theta_{MAP} | \vec{R} = \vec{r}_*)| / \hat{P}_a^{2n}(\Theta = \theta_{MAP} | \vec{R} = \vec{r}_*) \leq 0.05 
\end{equation*}
and
\begin{equation*}
|\hat{P}_a^{n}(\Theta = \theta | \vec{R} = \vec{r}_*) - \hat{P}_a^{2n}(\Theta = \theta | \vec{R} = \vec{r}_*)| \leq 0.05.
\end{equation*}

\section{Sequential Importance Sampling results}
 
Sequential importance sampling is done under importance sampling distribution with the following properties:
\begin{itemize}
\item{parameter $p$ is fixed in steps $t<5$ and $p=1$ in the last step $t= T = 5$,}
\item{parameter $q$ is fixed,}
\item{at each step, only nodes that are in the given final simulation may be infected with probability $p$,}
\item{nodes that are infected may be recovered with probability $q$.}
\end{itemize}

The simulations are done under the same convergence condition as Direct Monte Carlo simulation from the benchmark dataset, starting from $n = 10000$ samples.

\begin{figure}[H]
\includegraphics[width=\textwidth]{/home/iva/dipl/res/seq_benchmark/accuracy_true_node.png}
\caption{}
\label{accuracy_true}
\end{figure}
Figure \ref{accuracy_true} represents accuracies of estimations obtained by Direct Monte Carlo and Sequential Importance Sampling estimators w.r.t. the realizations true source node. In other words, they represent the portion of MAP estimations that correctly estimated the source node of the realization. When we observe low accuracy for Direct Monte Carlo estimator on average, we shouldn't expect such accuracy to be higher for the "inferior" Sequential Monte Carlo estimator. Accuracies for Direct Monte Carlo and Sequential Importance Sampling estimators follow similar pattern overall and for all the SIR parameter classes. For classes A and B they are low, and for classes C and D they are high. 

\begin{figure}[H]
\includegraphics[width=\textwidth]{/home/iva/dipl/res/seq_benchmark/MAP_MAP_Acc.png}
\caption{}
\label{map_map_acc}
\end{figure}
Figure \ref{map_map_acc} represents the accuracy used in \cite{Nino} to compare range of estimators. This accuracy refers to the portion of MAP estimations that are equal to corresponding MAP estimations of Direct Monte Carlo estimator provided in the benchmark dataset. Soft Margin accuracies presented here are taken from \cite{Nino}.Those were calculate with fixed $a=0.031$ and under the same convergence conditions as the benchmark Direct Monte Carlo solutions. Sequential Importance Sampling estimator for classes A and B outperforms SoftMargin. This only means its MAP estimations are  more similar to Direct Monte Carlo estimations. Note that these classes also have low true source node accuracy and belong to low  to medium detectability zone of parameters.

The similarity between estimations obtained with Sequential Importance Sampling and Direct Monte Carlo also presents itself as a low relative MAP error estimation w. r. t. Direct Monte Carlo probability across all classes of parameters, as presented in Figure \ref{rel_err}.
\begin{figure}[H]
\includegraphics[width=\textwidth]{/home/iva/dipl/res/seq_benchmark/MAP-rel_err.png}
\caption{}
\label{rel_err}
\end{figure}

\begin{figure}[H]
\includegraphics[width=\textwidth]{/home/iva/dipl/res/seq_benchmark/bench_sim_no.png}
\caption{}
\label{bench_sim_no}
\end{figure}
In Figure \ref{bench_sim_no} distributions of number of simulations (samples) for which the estimators converged are presented. For Sequential Importance Sampling estimator we observe $10^5$ samples are needed for SIR parameters in classes C and D in more than $80\%$ of benchmark realizations. However, some simulations, observably mostly those in classes A and B, require more than $10^6$ samples for convergence. The impact on the accuracies and the results presented here when the number of samples is capped by $10^6$ is yet to be analysed. 

\begin{figure}[H]
\includegraphics[width=\textwidth]{/home/iva/dipl/res/seq_benchmark/bench_sim_acc_DMC.png}
\caption{}
\label{bench_sim_accDMC}
\end{figure}
Figure \ref{bench_sim_accDMC} presents accuracy w.r.t. true source node of Direct Monte Carlo and Sequential Importance Sampling based estimations grouped by number of simulations required to obtain Direct Monte Carlo estimation for the coresponding benchmark sample. 
	
\begin{figure}[H]
\includegraphics[width=\textwidth]{/home/iva/dipl/res/seq_benchmark/bench_sim_acc_SIS.png}
\caption{}
\label{bench_sim_accSIS}
\end{figure}
Figure \ref{bench_sim_accSIS} presents accuracies for Sequential Importance Sampling. Note the benchmark samples that required more than $10^7$ simulations are the ones Direct Monte Carlo estimator also failed to estimate correctly.  

\begin{figure}[H]
\includegraphics[width=\textwidth]{/home/iva/dipl/res/seq_benchmark/bench_acc_sim_DMC.png}
\includegraphics[width=\textwidth]{/home/iva/dipl/res/seq_benchmark/bench_acc_sim_SIS.png}
\caption{}
\label{bench_acc_simDMC}
\end{figure}


\chapter{Detectability of patient zero} 
\label{Det}

The source detectability $D(\vec{r_*}) = 1 - H(\vec{r_*})$ is characterized via Shannon entropy H (normalized by entropy of uniform distribution) of the estimated source probability distribution $P(\Theta = \theta_i |\vec{R} = \vec{r_*}).$

\section{Detectability based on parameters of the SIR model}
\begin{figure}[H]
\includegraphics[width=\textwidth]{/home/iva/dipl/res/supfig12/SIR_grid3_org.png}
\caption{Box plots of estimated entropy density for entropy of source probability distributions of 
single source candidates on the $4$-connected lattice of different sizes estimated with Soft Margin method with $10^4 - 10^6$ simulations with adaptive $a$ chosen from $\{1/2^3, 1/2^4, \ldots, 1/2^{15}\}$. Estimation is done under $SIR$ model with different parameters $p$ in range $0.1 - 0.9$, fixed $q = 0.5$ and $T = 5$. The source node in each experiment is the central node of lattice. Each entropy density is estimated with $50$ experiments containing realizations with more than $1$ node.}
\label{network_size}
\end{figure}

\begin{figure}[H]
\includegraphics[width=\textwidth]{/home/iva/dipl/res/supfig12/SIR_big_grid_org.png}
\caption{Box plots of estimated entropy density for entropy of source probability distributions of 
single source candidates on the $4$-connected lattice $30 \times 30$ estimated with Soft Margin method with $10^4 - 10^6$ simulations with adaptive $a$ chosen from $\{1/2^3, 1/2^4, \ldots, 1/2^{15}\}$. Estimation is done under $SIR$ model with different parameters $p$ in range $0.1 - 0.9$, and $q = \{0, 0.5, 1\}$ with $T = 5$. The source node in each experiment is the central node of lattice. Each entropy density is estimated with $50$ experiments containing realizations with more than $1$ node.}
\label{entropy_zones}
\end{figure}

In Figures \ref{network_size} and \ref{entropy_zones} the results of \cite{Nino} are reproduced. The existence of different detectability regimes is shown in Figure \ref{entropy_zones} as well as a similar detectability behaviour for SIR models with the same parameter $p$ across different values of parameter $q$.  Three entropy regions are observed: low detectability-high entropy region $(p < 0.2)$, intermediate detectability - intermediate entropy region $(0.2 < p < 0.7)$ and high detectability-low entropy region $(p > 0.7)$.

In a regime where network size restricts the epidemic spreading but not the epidemic itself, the entropy is high as the realizations from different sources are almost identical (Figure \ref{network_size}).

\section{Detectability based on parameters of the ISS model}
\begin{figure}[H]
\includegraphics[width=\textwidth]{/home/iva/dipl/res/iss_grid/ISS_big_grid_org.png}
\caption{Box plots of estimated entropy density for entropy of source probability distributions of 
single source candidates on the $4$-connected lattice $30 \times 30$ estimated with Soft Margin method with $10^4 - 10^6$ simulations with adaptive $a$ chosen from $\{1/2^3, 1/2^4, \ldots, 1/2^{15}\}$. Estimation is done under $ISS$ model with different parameters $a$ in range $0.1 - 0.9$, and $b = \{0, 0.5, 1\}$ with $T = 5$. The source node in each experiment is the central node of lattice. Each entropy density is estimated with $50$ experiments containing realizations with more than $1$ node.}
\end{figure}

\section{Detectability based on network topology for the SIR model}
\subsection{Barabassi graph}
\begin{figure}[H]
\includegraphics[width=\textwidth]{/home/iva/dipl/res/bara/bara_big.png}
\caption{Kepsn}
\end{figure}


\subsection{Erdos-Renyi graph}
\begin{figure}[H]
\includegraphics[width=\textwidth]{/home/iva/dipl/res/bara/bara_big.png}
\caption{Kepsn}
\end{figure}

\subsection{Topological properties of the nodes}



\chapter{Conclusion}
Zaključak.

\bibliography{Bibliography}
%\bibliographystyle{plainnat}
\bibliographystyle{unsrtnat}

\begin{sazetak}
Sažetak na hrvatskom jeziku.

\kljucnerijeci{Ključne riječi, odvojene zarezima.}
\end{sazetak}

% TODO: Navedite naslov na engleskom jeziku.
\engtitle{Title}
\begin{abstract}
Abstract.

\keywords{Keywords.}
\end{abstract}

\end{document}
S}
