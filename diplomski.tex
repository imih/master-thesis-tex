\documentclass[times, utf8, diplomski]{fer}
\usepackage{booktabs}
\usepackage[ruled, linesnumbered, noend]{algorithm2e}
\usepackage{float}
\usepackage{mathtools}
\usepackage{amsmath}
\newcommand\tab[1][1cm]{\hspace*{#1}}
\usepackage{graphicx}
\usepackage[final]{pdfpages}
\usepackage{pdflscape}

\begin{document}
\thesisnumber{1286}
\title{Detectability of Patient Zero Depending on its Position in the Network}
\author{Iva Miholić}
\maketitle
\izvornik
\zahvala{}
\tableofcontents
%\listoffigures
\listofalgorithms
\addcontentsline{toc}{chapter}{List of Algorithms}
\listoftables

\chapter{Introduction}

A \emph{network} is a set of items with connections between them. The Internet, the World Wide Web, social networks like genealogical trees, networks of friends or co-workers, biological networks like epidemiological networks, networks of citations between papers, distribution systems like postal delivery routes: they all take a form of networks. Most social, biological and technological networks have specific structural properties. Such networks are referred to as \emph{complex networks}.  An example of a complex network of scientific collaborations is presented in Figure \ref{net}.

A network structure or a topology can be mathematically modelled as a graph with set of vertices (or nodes) representing the items of the network. The network structure can then be analysed using graph theory. An edge between two nodes represents a connection between the two corresponding items. Edges can be directed or undirected, depending  on the nature of the connection.  
\begin{figure}[h]
\centering
\includegraphics[scale = 0.3]{/home/iva/dipl/img/alters2.png}
\caption{A network graph of Paul Erd{\H{o}}s and his collaborators, courtesy of \citet{krebs}. The nodes represent mathematicians and the edges represent the relationship "wrote a paper with".}
\label{net}
\end{figure}

To better mimic the real-world (complex) network structure, it is common to add attributes to nodes and/or edges or to have both directed and undirected edges on the same graph.

For large-scaled complex networks that have millions or billions of vertices, the study in the form of traditional graph theory is not sufficient or sometimes possible. When this is the case, the statistical methods  for quantifying large complex networks are used. 

The ultimate goal of the study of complex network structure is to understand and explain the workings of systems built upon the network such as spreading of disease or information propagation.

After statistical properties analysis, the model of the system or a process  is created. The model can help us understand the meaning of statistical properties - how they came to be as they are and how they relate to the behaviour of a networked system. Based on statistical properties and using the right model, the behaviour of networked systems can be determined and predicted.

The basis of the complex network theory, the structure  analysis and the process modelling can be found in \citet{Newman03thestructure}.

\section{Epidemic processes in complex networks}

The models for stochastic processes such as disease spreading are categorized as homogeneous or heterogeneous mixing frameworks. The former assume that all individuals in a population have an equal probability of contact and different equations can be applied to understand epidemic dynamics. Since such models fail to describe the realistic scenario of disease spreading,  heterogeneity is introduced by using a network structure.

There is an extremely close relationship between epidemiology and network theory since the connections between individuals (or group of individuals) allowing an infectious disease to propagate naturally define a contact network. Simplest epidemic dynamics consider a system with fixed total population consisting of $N$ individuals modelled with undirected contacting network. We define the contact network as an undirected and non-weighted graph $G(N, L)$ with fixed set of nodes $N$ and fixed set of links $L$. A link $(u, v)$ between two nodes exists if the two corresponding members were in contact during the epidemic time.

The structure of the network has profound impact on the contagnion dynamics but in order to understand the evolution of the epidemic over time we have to define the basic individual-level processes that govern the epidemic spreading. Complementary to the network, epidemic modelling describes the dynamical evolution of the contagion process within a population. The state of the art results on epidemic modelling in complex networks can be found in \citet{revmod}.

 Classic epidemic models generally  assume the network is static during epidemic process while the population can be divided into different classes or compartments depending on the stage of the disease, such as susceptible (those who can contract the infection), infectious (those who contracted the infection and are contagious), recovered,  removed or immune. The model defines the basic processes that govern the transition of individuals from one compartment to another.
 Each member of population can be a part of exactly one compartment at once. 
 
 Understanding the structure of the transmission network along with choosing the right epidemic model allows us to predict the distribution of infection and to simulate the full dynamics in order to control disease or plan immunization. In this thesis we will focus on SIR model for epidemic spreading and its modification, the ISS model for modelling rumour diffusion. 

\section{Finding patient zero}  
The inverse problem of estimating the initial epidemic conditions like localizing the source of an epidemic commonly known as the patient zero problem has only recently been formulated.

In the patient zero problem the source(s) of an epidemic or information diffusion propagation are determined based on limited knowledge of network structure or partial history of the propagation. The survey of methods for identifying the propagation source in networks can be found in \citet{soa_source}.

In the case of the SIR model there are three different approaches. \citet{Zhu} proposed a simple path counting approach and prove that the source node minimizes the maximum distance (Jordan centrality) to the infected nodes on infinite trees. \citet{Lohkov} used a dynamic message-passing algorithm and estimate the probability that a given node produces the observed snapshot using a mean-field approch and an assumption of a tree-like contact network.

\citet{Nino} introduce analytical combinatoric, as well as Monte-Carlo based methods for epidemic source detection problem. These methods produce exact and approximate source probability distribution for any network topology based on a snapshot of the epidemic at known discrete time $T$. The provided benchmark results show Monte-Carlo based MAP estimators outperform previous results on a lattice network for the SIR model. 

Additionally, these methods are applicable to many heterogeneous mixing models (SIR, IS, ISS) and are able to introduce uncertainty in the epidemic starting time, as well as uncertainty of temporal ordering of interactions. 
Even though the introduced Monte Carlo methods assume the epidemic started from a single source, one can also discriminate such hypothesis using Kolmogorov-Smirnov test \cite{Nino}.

\section{Effects of network topology on epidemic spreading and detectability of patient zero}
Complex networks show various levels of correlation in their topology which can have an impact on dynamical processes running on top of them.

 
Real-world networks of relevance for epidemic spreading are different from regular lattices. Networks are hierarchically organized with a few nodes that may act as hubs and where the vast majority of nodes have few direct connections. 

Although randomness in the connection process of nodes is always present, organizing principles and correlations in the connectivity patterns define network structures that are deeply affecting the evolution and behavior of epidemic and contagion process. These  complex features often find their signature in statistical distributions which are generally heavy tailed and skewed. 

\citet{Nino} have  introduced a metric for source detectability based on the entropy of estimated source probability distribution. The detectability of source node differs based on models parameters concerning the rate of disease spreading. Since topological properties of the network  have profound impact on epidemic dynamics, the detectability of source node in relation to its topological properties is an interesting analytical problem.

%\section{Structure of the thesis}
%%T give overview of the chapters
%In Chapter \ref{structure} ... 
%
%% [x] Implement Monte Carlo and Soft margin algorithms for the identification of patient zero for SIR (susceptible-infected-recovered) and ISS (ignorant-spreader-stifler) dynamic models in networks. 
%In Chapter \ref{EPM} the SIR epidemic model as well as ISS model for rumour diffusion are introduced and the algorithms for their simulation in discrete time are presented. 
%
%Chapter \ref{PZ} introduces Direct Monte Carlo and Soft Margin Monte Carlo algorithms \cite{Nino} for indetification of patient zero. 
%
%% [x] Check whether it is possible to implement importance sampling in the Monte Carlo algorithm. 
%In Chapter \ref{IS} the technique of importance sampling is  used to create the Sequential Importance Sampling (SIS) algorithm - a new algorithm for single source detection problem. A few reduction based optimization techniques are also introduced. 
%
%% [x] The solution should be appropriated for parallel architectures and implemented in c++. % [] The code is to be documented using comments and should follow the Google C++ Style Guide. 
%% [x]The complete application should be hosted on Github under an OSI-approved licence.\textbf{q}
%Chapter \ref{Bench} uses benchmark data from \cite{Nino} to assert accuracy of Direct Monte Carlo and Soft Margin implementations. Detailed analysis of SIS algorithm performance is also given.
%
%% [x] For several networks explore the detectability of the source node when source nodes are selected using different centrality measures such as degree, k-core, betweennes centraliy and eigenvector centrality. 
%In Chapter \ref{Det} detectability of source node based on parameters of SIR and ISS model is revisited. The detectability analysis for $1-$Barab\'{a}si and Erdos Renyi sample graphs is given, as well as the detectability breakdown based on nodes attributes: degree, k-core, betweenness centrality and eigenvector centrality.

\chapter{Complex network structure}
\label{structure}
Most of real networks in social and biological systems are characterized by  similar topological properties: small average path length, high clustering coefficients, fat tailed scale-free degree distributions and  local network structure observable in the presence of communities.

\section{Measures and metrics}
Since larger networks can be difficult to envision and describe only by the graph $G$, we observe more detailed insights of the structure of these networks with various metrics. 

\subsubsection{Degree distribution}
Degree distribution $P(k)$ defines the probability that a vertex in the network interacts with exactly $k$ other vertices. That is, $P(k)$ is the fraction of nodes in the network with degree equal to $k$.  

\emph{Scale-free} power-law degree distribution of the form $P(k) = Ak^{-\gamma}$ where $2<\gamma<3$ appears in wide variety of complex networks. The networks with such property are referred to as \emph{scale-free networks}. This feature is a consequence of two generic mechanisms: networks expand continuously by the addition of new vertices and new vertices attach preferentially to sites that are already well connected \cite{Barabasi99emergenceof}. It is often said the scale-free distributions have "fat tails" since there tends to be many more nodes with higher degree compared to a Poisson degree distribution in a network with links formed completely independently.


\subsubsection{Geodesic path}

A path in a network is defined as an arbitrary sequence of vertices in which each pair of adjacent vertices is  directly connected in the graph. 
%Number of different paths between two vertices $i$ and $j$ can be computed from the $0-1$ adjacency matrix as $A^k_{ij}$. The number of different cycles of length $k$ can thus be computed as the sum $\sum_{m \in V} A^k_{mm}$ - exactly the trace of the matrix $A^k$. 

A geodesic path is the shortest path between two vertices.
The small world network property observable in complex networks is considered to be present when average shortest path length is comparable to the logarithm of the network size. 

\subsubsection{Centrality}
Centrality measures compare  nodes and say something about how a given node relates to the overall network. 

\textbf{Degree centrality} describes how connected a node is in terms of direct connections. For a vertex $v$ in a network with $n$ vertices it is defined as $\frac{deg(v)}{n - 1}$. Since the degree centrality captures only centrality in terms of direct connections, it doesn't measure node's marginal contribution to the network when the node has relatively few links  but lies in a critical location in the network which can be the case.

\textbf{Closeness centrality} describes how close a given vertex is to any other vertex.  Let $d_{ij}$ denote the length of geodesic path from vertex $i$ to vertex $j$. 
For  vertex $v$ closeness centrality $C_v$ is defined as harmonic mean between the distances of geodesic paths from vertex $v$ to all others:
\begin{equation}
C_v = \frac{1}{n - 1} \sum_{j \neq v} \frac{1}{d_{vj}}. 
\label{clos}
\end{equation}

\textbf{Betweenness centrality} describes how well situated a vertex is in terms of the paths it lies on. Let $\sigma_{st}$ be the number of geodesic paths between pairs of vertices $v_s$ and $v_t$ and let $\sigma_{st}(v_i)$ be the number of the geodesic paths $\sigma_{st}$  which pass via vertex $v_i$. The betweenness centrality is than defined as 
\begin{equation}
C(v_i) = \sum_{s, t} \frac{\sigma_{st}(v_i)}{\sigma_{st}}.
\label{betw}
\end{equation}

Neighbours characteristics like  eigenvector centrality measure how important, central or influential nodes neighbours are and  capture a concept the vertex is more important if it has more important neighbours.  

Let's define the adjacency matrix $A$ of network $G$ with $N$ nodes as a matrix of size $N \times N$ that contains non-zero element $A_{ij}$ if there exist an edge between vertices $i$ and $j$. For an unweighed network all non-zero elements of $A$ are equal to one.  Note the adjacency matrix is symmetric for undirected graphs and generally asymmetric for  directed graphs. 

For the given vertex $v$,  \textbf{ eigenvector centrality } $C_v$ \cite{bonacich1987power} is proportional to the sum of  centralities of its neighbours: 
\begin{equation}
\lambda C_v = \sum_{k} A_{vk} C_k.
\label{eig}
\end{equation}
 Consequently, $C_v$ is eigenvector of adjacency matrix $A$ corresponding to eigenvalue $\lambda$.  The standard convention is to use the eigenvector associated with the largest eigenvalue for the eigenvector centrality. 

\subsubsection{K-core}
A $k$-core of undirected graph $G$ is a maximal connected subgraph of $G$ in which all vertices have degree at least $k$. The $k$-core is a measure of how sparse the graph is. 
Additionally, a vertex $u$ has coreness $c$ if it belongs to a $c$-core but not to  $(c+1)$-core. 

The $k$-core can be obtained in $O(|L|)$ time by iteratively removing all vertices of degree less than $k$ from the graph. 

 The $k$-core decomposition refers to a process of determining the coreness of each node and grouping the nodes according to their coreness. The concept of $k$-core (decomposition) was introduced to study the clustering structure of social networks and to describe the evolution of random graphs. $K$-core decomposition of complex networks reveals rich $k$-core architectures as presented in Figure \ref{kcore}. 

\begin{figure}
\includegraphics[width=\textwidth]{/home/iva/dipl/img/kcore_rep.png}
\caption{Graphical representation of a fraction of the .fr domain of Web, courtesy of \citet{Alvarez-hamelin_k-coredecomposition:}. The vertices of the same coreness are represented with the same color.}
\label{kcore}
\end{figure}

\section{Modelling global network structure}
\subsection{Erd{\H{o}}s R{\'{e}}nyi graph model}

Traditionally, networks of complex topology have been described with the random graph theory of Erd{\H{o}}s and R{\'{e}}nyi \cite{Erdos1959}, but in the absence of data on large networks, the predictions of the ER theory were rarely tested in the real world.

This random graph model assumes we start with $N$ vertices and connect each pair of vertices with probability $p$. The formation is independent across links so the probability of generating a network with exactly $m$ links is equal to $p^m(1 - p)^{\frac{N(N - 1)}{2} - m}$ and the expected number of links is $\langle d \rangle = pN(N - 1) / 2$. 

The degree distribution of the generated random network is 
\begin{equation}
P(d) = {{N - 1}\choose{d}}p^d (1 - p)^{N - 1 - d}.
\end{equation}
For large $n$, the degree distribution follows a Poisson distribution $P(d) = e ^{-\lambda}\lambda^d / d!$, where $$\lambda = N {{N - 1}\choose{d}} p^d (1 - p)^{N - 1 - d}.$$
% Hence, the probability of finding a highly connected vertex (that is, of large $d$) decreases exponentially with $k$ for large $N$.

\citet{Erdos1959} have described the behaviour of degree distribution for various values of $p$. The important result talks about the  emergence of one large graph component for higher values of $p$. In more detail,

\let\labelitemi\labelitemii
\begin{itemize}
\item{if $Np < 1$ a generated graph will almost surely have no connected components of size larger than $O(\log(N))$, } 
\item{if $Np = 1$ a graph will almost surely have a largest component whose size is of order $N^{2/3}$,}
\item{if $Np > 1$ a graph will almost surely have a unique giant component, i.e. no other component will contain more than $O(\log(N))$ vertices.}
\end{itemize}

In Erd{\H{o}}s-R{\'{e}}nyi model, maximum coreness is related to the average degree $\langle d \rangle$. Since the topology is very homogeneus, it is also expected most vertices will belong  to the same $k$-core that is also the highest. 

While random network can observe features like diameters small relative to the network size, they lack certain features that are prevalent among complex networks, such as  high clustering and presence of communities. 

\subsection{Barab\'{a}si-Albert graph model}

Barab\'{a}si-Albert model is the model of evolving a scale-free network  which uses a preferential attachment property thus creating a heterogeneous topology.  The preferential attachment mechanism is one of two generating mechanisms of scale-free networks \cite{Barabasi99emergenceof} and refers to building the network gradually where each new vertex tends to connect to old vertices that are already well connected within the old network.

The Barab\'{a}si-Albert graph is generated starting from $m_0$ isolated vertices. At each time step new vertices with $m$ edges are added to the network $m < m_0$. The new vertex will create an edge to the existing node $v_i$  with probability proportional to its degree $k_i$.

 The Barab\'{a}si-Albert graph model produces a power law degree distribution $P(k) \approx k ^{-3}$  in the limit of growth time, i.e. number of vertices. The average geodesic path  increases logarithmically with the size of the network.

By repetitively connecting each new node to the previous graph with exactly $m$ edges, we obtain a graph where any subgraph has a vertex of degree at most $m$ and the $k$-core of the graph is $m$. 

\chapter{Epidemic process modelling}
\label{EPM}
In the focus of this thesis are heterogeneous epidemic models on the contact network formed by connections between single contacting individuals with transitions of individuals between compartments happening in discrete time steps.

\section{SIR model}
Wide range of diseases that provide immunity to the host can be successfully modelled on a network whose members take 	one of three possible roles at a time: susceptible $(S)$, infected $(I)$ or recovered $(R)$ \cite{Kermack1991}. 

The diffusion of disease takes place between infected nodes and their susceptible neighbours. An infectious node may also recover from the disease. The recovery grants permanent immunity effectively erasing the member from the contacting network.  The possible events can be represented as 
\begin{equation}
S + I \xrightarrow{p} 2I,\hspace{5mm}  I \xrightarrow{q} R.
\label{transitionsSIR}
\end{equation}

In the SIR model infection and recovery process completely determine the epidemic evolution. The transitions (\ref{transitionsSIR}) occur spontaneously and independently in each time step. In discrete-time formulation an infected individual when meeting susceptible will infect the neighbouring susceptible with probability $p$ at each time step. The recovery probability $q$ is the probability the infected individual will recover for each time step. 

The transition probabilities $p$ and $q$ are often assumed constant and equal for all nodes in the same epidemic process. 

\subsection{Simulating the discrete SIR epidemic}

For the contacting network represented by graph $G(N, L)$ and SIR parameters $p$ and $q$, we are able to simulate one time step of discrete SIR process. Let $s_t$, $i_t$ and $r_t$ denote sets of nodes that are respectively susceptible, infected and recovered after time step $t$. 

At time step $t$ all previously infected nodes $i_{t - 1}$ will try to infect their susceptible neighbours independently of each other and at the same time. Afterwards the passive recovery process will try to turn them to recovering nodes, each with probability $q$.

This process can be simulated with NaiveSIR algorithm  \cite{NaiveSIR} by putting all the initially infected nodes in the queue. While traversing the nodes, we try to infect each neighbouring node. When the new node gets infected, it gets pushed to the queue. 

SIR simulation of one time step $t$ is described in Algorithm \ref{SIRStep}.

\begin{algorithm}[h]
 \caption{One time step of NaiveSIR simulation on graph $\mathbf{G}$.}
 \label{SIRStep}
 \KwData{$\mathbf{G}$ - network, $(p, q)$ -parameters of the SIR model, 
  \textbf{$Iq$} - queue of infected nodes,    
  \textbf{$I$} - bitset of infected nodes, 
  \textbf{$S$} - bitset of susceptible nodes,
  \textbf{$R$} - bitset of recovered nodes}
 infected\_size = \textbf{size}($Iq$)\\
 \For{$k = 1$ to infected\_size} {
 \If{$Iq$ is empty} {
 \textbf{break}\\
 }
 \textbf{dequeue}(u, Iq)\\
  \ForEach {$v \in nei(u)$}{
    \uIf{$v \in S$}{
    let transmission $u \rightarrow v$ occur with probability  $p$\\
    \If{$u \rightarrow v$ \textbf{occured}}{ 
     \textbf{update} $I(v)$ and $S(v)$\\
     }}
  }
  let transmission $u \rightarrow v$ occur with probability  $q$\\
             \uIf{$u \rightarrow v$ \textbf{occured}}{ 
      \textbf{update} $I(u)$ and $R(u)$\\
                }\Else{ 
                \textbf{enqueue}(u, Iq)\\
  }
  }
  \Return \{S, I, R\}
\end{algorithm}

\subsubsection{Time and space complexity of NaiveSIR algorithm}
For algorithm complexity analysis standard big-$O$ notation is used (asymptotic upper bound within a constant factor) \cite{Graham:1994:CMF:562056}. 
In a single SIR step,  simulation tries to infect all neighbours of infected nodes that are susceptible, i.e.  $O(\langle d \rangle)$ nodes where $\langle d \rangle$ denotes the average node degree. Since after each SIR step each infected node is recovered with probability $q$, the average number of time steps the node spends in infected state is a sample from geometric distribution $P(\Delta T= \Delta t) = (1 - q)^{\Delta t - 1}  q$ with expectation $\frac{1}{q}$. Total time complexity for one infected node is thus $O(\frac{\langle d \rangle}{q})$.   
Finally, the average  case running time of the NaiveSIR algorithm is equal to $O(\frac{E[X]\langle d \rangle}{q})$ where $E[X]$ denotes total expected number of infected nodes  \cite{NaiveSIR}.
 
The space complexity of NaiveSIR algorithm with respect to the number of links $L$ of the graph $G$ is equal to $O(L)$ since the memory holds a contact network $G$ in a form of adjacency list ($O(L)$), queue of infected nodes ($O(N)$) and indicators of each compartment that are best implemented as a bitset data structure ($O(1)$). 


\subsection{Probability of compartment transitions in one time step of SIR simulation}
Probability of compartment transitions in one time step of SIR simulation can be easily evaluated. Let $nei(v)$ indicate a set of all neighbours of node $v$,  $nei(V)$ a set of all neighbours of all nodes in set $V$ and  $nei_V(v) = nei(v) \cap V$  a set of all neighbours of $v$ that are also in $V$. After $k$-th time step of the SIR process the resulting $i_k$ and $r_k$ were given. At time step $k$, only initially active nodes $i_{k - 1}$ and their neighbours $nei(i_{k - 1})$ actively participate in the epidemic process. For each node $v$ in $i_{k - 1} \cup nei(i_{k - 1})$, one of four independent events may happen during time step $k$ and they are easily detectable based on $i_{k - 1}, r_{k - 1}, i_{k}$ and $r_{k}$:
\let\labelitemi\labelitemii
\begin{itemize}
\label{listE}
\item{$E_1:$ $\mathbf{if}$ $v \not\in i_{k - 1}$ and $v \not\in r_{k - 1}$ and $v \in i_{k}$\\ \tab node $v$ was infected with probability $1 - (1 - p) ^ {nei_{i_{k-1}}(v)}$}
\item{$E_2:$ $\mathbf{if}$ $v \not\in i_{k - 1}$ and $v \not\in r_{k - 1}$ and $v \not \in i_{k}$\\ \tab node $v$ was not infected with probability $(1 - p)^{nei_{i_{k-1}}(v)}$}
\item{$E_3:$ $\mathbf{if}$ $v \in i_{k - 1}$ and $v \in r_{k}$\\ \tab  node $v$ was recovered with probability $q$}
\item{$E_4:$ $\mathbf{if}$ $v \in i_{k - 1}$ and $v \not\in r_{k}$ \\ \tab node $v$ was not recovered with probability $1 - q$}
\end{itemize}

Since all events $E_1 - E_4$ are independent and  sets of nodes corresponding to each event are disjoint while completely covering the set of active nodes $i_{k - 1} \cup nei(i_{k - 1})$, the  conditional probability of  one time step SIR transition $P(i_k, r_k | i_{k - 1}, r_{k - 1})$ can be calculated as
\begin{equation}
\begin{split}
P(i_k, r_k | i_{k - 1}, r_{k - 1}) =  &\big[\Pi_{v \in E_1} (1 - (1 - p) ^ {nei_{i_{k-1}}(v)})\big]\big[\Pi_{v \in E_2} (1 - p)^{nei_{i_{k-1}}(v)} \big] \\& \cdot
\big[\Pi_{v \in E_3} q\big]\big[ \Pi_{v \in E_4} (1 - q)\big].
\end{split}
\label{fISS}
\end{equation}
Set  $nei_{i_{k-1}}(v)$ denotes the set of all neighbours of $v$ that were  infected  at the beginning of time step $k$, i.e. the set $nei(v) \cap i_{k - 1}$. 

\section{Epidemic models as social contagion processes}
Even though infectious diseases represent the central focus of epidemic modelling, the model where an individual is strongly influenced by the interaction with its peers is present in several other domains, especially in social context in the diffusion of information, the propagation of rumour and adoption of innovation or behaviours.  Since the social contacts can in these domains generate epidemic-like outbreaks, simple models for information diffusion are epidemic models modified to specific features of social contagion. The crucial difference to pathogen spreading is that transmission of information involves intentional acts by both the sender and the receiver and it is often beneficial for both participants.

\subsection{Rumour spreading with ISS model}
The need to study rumour spreading presents itself in a number of important technological and commercial applications where it is desirable to spread the "epidemic" as fast and as efficient as possible.  In examples such as rumour based protocols for resource discovery and marketing campaigns that use rumour like strategies (viral marketing)  the problem translates to design of an epidemic algorithm in such a way that the given information reaches as much nodes as possible, similarly to a rumour.

Models for rumour spreading are variants of the SIR model in which the recovery process does not occur spontaneously, but rather is a consequence of interactions. The modification mimics the idea it is worth spreading the rumour as long as it is novel for the recipient.

 This process can be formalized as a model where each of  $N$ members of the contacting network can be a part of one of three compartments: ignorant (S), spreader (I) and stifler (R). Ignorants have not heard the rumour and are susceptible to being informed. Spreaders are actively spreading the rumour, while stiflers know about the rumour but they're not  spreading it.
 
The spreading process evolves by direct contacts of spreaders with others in the population. 
When a spreader meets an ignorant, the latter turns into a new spreader with probability $a$. When a spreader meets another spreader or a stifler, the former spreader turns into stifler with probability $b$ and the latter remains unchanged. This model is known as  ISS (Ignorant-Spreader-Stifler) model \cite{Moreno2004}. The possible events can be represented as 
\begin{equation}
S + I \xrightarrow{a} 2I,\hspace{5mm}  R + I \xrightarrow{b} 2R,  \hspace{5mm} 2I \xrightarrow{b} R + I.
\end{equation}

Since we are examining the spreading process in discrete time, at each time step the current spreaders try to interact with their neighbours. A modification of the NaiveSIR algorithm for rumour spreading simulation of one time step $t$ is described by Algorithm \ref{ISSStep}. 
 
\begin{algorithm}
\caption{One time step of ISS simulation with modified NaiveSIR algorithm on graph $\mathbf{G}$.}
 \label{ISSStep}
 \KwData{$\mathbf{G}$ - network, $(a, b)$ - parameters of the ISS model, 
  \textbf{$Iq$} - priority queue of spreader nodes,    
  \textbf{$I$} - bitset of spreader nodes, 
  \textbf{$S$} - bitset of ignorant nodes,
  \textbf{$R$} - bitset of stifler nodes }
 stifler\_size = \textbf{size}($Iq$)\\
 \For{$k = 1$ to stifler\_size} {
 \If{$Iq$ is empty} {
 \textbf{break}\\
 }
 \textbf{dequeue}(u, Iq)\\
  \ForEach {$v \in nei(u)$}{
    \uIf{$v \in S$}{
    let transmission $u \rightarrow v$ occur with probability  $a$\\
    \If{$u \rightarrow v$ \textbf{occured}}{ 
     \textbf{update} $I(v)$ and $S(v)$\\
     }}\uElse{
         let transmission $v \rightarrow u$ occur with probability  $b$\\
             \If{$v \rightarrow u$ \textbf{occured}}{ 
      \textbf{update} $I(u)$ and $R(u)$\\
                }}
  }
  \If{$u \in$ I}{
  \textbf{enqueue}(u, Iq)\\
  }
  }
  \Return \{S, I, R\}
\end{algorithm}

\chapter{Patient zero -- single source epidemic detection}
\label{PZ}

In accordance with \citet{Nino}, we will focus on a patient zero problem given snapshot of population at time $T$ and complete knowledge of underlying contacting network modelled by graph $G(N, L)$ with  assumption the epidemic has started from a single source node and that it is governed by the SIR process with known $p$ and $q$. 

 The estimators proposed by \citet{Nino} will be presented in this chapter, while the newly proposed estimators based on importance sampling  technique will be presented in the next chapter.

\section{Problem definition}
Let random vector $\vec S = (S(1), \ldots, S(N))$ indicate the nodes that got infected up to a predefined temporal threshold $T$ with SIR$(p, q)$ epidemic process on network $G$ with $N$ nodes. $S(i)$ is a Bernoulli random variable with the value $1$ if the node $i$ got infected before time $T$ from the start of the epidemic process. 

We observe one realization $\vec s_*$ of $\vec S$ -- the epidemic snapshot at time $T$ and want to infer which nodes from the set of infected or recovered nodes $\Theta = \{\theta_1, \theta_2, \ldots, \theta_m \}$ are most likely to be the source of observed epidemic process.  
 The finite set of possible source nodes $\Theta$ is determined by realization $\vec s_*$.

A maximum aposteriori probability estimate (MAP) is the node with the highest probability for being the source of the epidemic spread for  given target realization $\vec s_*$: 
\begin{equation}
\hat{\theta}_{MAP} = \text{arg max}_{\theta_i \in \Theta} P(\Theta = \theta_i | \vec{S} = \vec s_*)
\label{MAP}
\end{equation}
By applying the Bayes theorem with equal apriori probabilities $P(\Theta = \theta_i)$, probability in (\ref{MAP}) can be expressed as 
\begin{equation}
\begin{aligned}
P(\Theta = \theta_i | \vec{S} = \vec s_*) &= \frac{P(\vec S = \vec s_* | \Theta = \theta_i) P(\Theta = \theta_i)}{\sum_{\theta_k \in \Theta} P(\vec S = \vec s_* | \Theta = \theta_k) P(\Theta = \theta_k)} \\ &= \frac{P(\vec S = \vec s_* | \Theta = \theta_i)}{\sum_{\theta_k \in \Theta} P(\vec S = \vec s_* | \Theta = \theta_k)}.
\end{aligned}
\label{MAP_pravi}
\end{equation}

\section{Direct Monte Carlo epidemic source detector}

The integration problem
\begin{equation}
\mathbf{E_f[\textit{h(X)}]} = \int_{X} h(x) f(x) dx
\label{exp}
\end{equation} of computing the expectation of the function $h(X) : X \rightarrow \mathbb{R}$ of random variable $X$ with density $f(X)$ can be estimated using Monte Carlo technique with $n$ samples $X_1, \ldots, X_n$ generated from density $f$ as the empirical average 
\begin{equation}
h_n = \frac{1}{n} \sum_{j = 1}^{n} h(X_j).
\end{equation}
The convergence of $h_n$ towards $\mathbf{E_f[\textit{h(X)}]}$ is assured by the Strong Law of Large Numbers.

Inferring the probability  $P(\vec S = \vec s_* | \Theta = \theta_i)$ up to multiplicative constant is an integration problem equivalent to expectation of Kronecker delta function
 $\delta(\vec S) = 1\{\vec S = \vec s_{*}\}$ where $\vec S$ is a random variable governed by probability distribution $P(\vec S  | \Theta = \theta_i)$. 
 
 
 Let $m_i$ denote number of realizations out of $n$ that completely correspond to  $\vec s_*$ for a fixed source $\theta_i$ estimated using Monte Carlo technique:
\begin{equation}
 m_i = \sum_{j = 1}^{n} 1\{\vec S_i = s_{*}\}
\label{mi}
\end{equation}
where $\vec S_i$ are drawn from $P(\vec S | \Theta = \theta_i)$. 

The estimate $m_i$ is obtained using Direct Monte Carlo technique by simulating epidemic process up to time $T$ starting from a single infected node $\theta_i$ and checking whether the generated realization of $\vec S_i$ coincides with $\vec s_*$. Since $m_i$ is estimation of $P(\vec S = \vec s_* | \Theta = \theta_i)$ up to multiplicative constant $1/n$ for all $\theta_i \in \Theta$, we derive Direct Monte Carlo MAP detector based on the estimation of probability $P(\Theta = \theta_i | \vec S = \vec s_*)$ by combining (\ref{mi}) with Bayes rule  (\ref{MAP_pravi}): 
\begin{equation}
\hat{P_i}^n = \hat{P}(\Theta = \theta_i | \vec S = \vec s_*) = \frac{m_i}{m}
\end{equation}
where $m = \sum_{j = 1}^{n} m_j$ .

The accuracy of Direct Monte Carlo estimation is controlled by convergence conditions. Upon estimating two source PDF's $\hat{P}_i^n$ and $\hat{P}_i^{2n}$ with $n$ and $2n$ independent simulations respectively, the estimated distribution is said to converge when the following conditions are satisfied:
\begin{equation}
| \hat{P}_{MAP}^{2n} - \hat{P}_{MAP}^{n} | / \hat{P}_{MAP}^{2n} \leq c \hspace{2mm} \text{and} \hspace{2mm} | \hat{P}_i^{2n} - \hat{P}_i^{n} | \leq c \hspace{5mm} \forall \theta_i \in \Theta.
\label{DMCConv}
\end{equation}

The term $\hat{P}_{MAP}$ corresponds to MAP probability of estimated distribution $\hat{P}$. 

If the size of realization $\vec s_*$ is big, the number of simulations required to obtain reliable estimations can be large. This makes it is crucial to optimise the simulation procedure. 

 Since the estimations for different source node candidates are independent, the computations can be parallelised.  

Additionally, a prunning mechanism can be incorporated. If a simulation infects a node that was not infected during the target epidemic  represented by  realization $\vec s_*$, it is safe to stop the simulation prior to ending time $T$ and call the partial sample unequal to target realization $\vec s_*$.

\vspace{5mm}
\begin{algorithm}[H]
\label{DMC_lag}
 \caption{Direct Monte Carlo estimation of number of realizations out of $n$ simulations completely corresponding to $\vec s_*$ after $T$ time steps  for a fixed source node $\theta_i$.}
\ \KwData{$\mathbf{G}$ - network, $(p, q)$ - parameters of the SIR process, 
  $\vec s_*$ - target realization,    
  $T$ - temporal threshold, 
  $\theta_i$ - proposed source node,
  $n$ - number of simulations}
% \KwResult{ $m_i$ - estimated expected number of realizations started from $\theta_i$ and completely corresponding to $s_*$ }
 $m_i = 0$\\ 
 \For{$d = 1$ to $n$} {
   \For{$t = 1$ to $T$} {
    Continue SIR simulation $(d, p, q, \theta_i)$ for time step $t$ and obtain
    $\vec S_t^{(d)}$ \\
    \If{$\hspace{3mm} \exists j \in N: (S_t^{(d)}(j) == 1$ \textbf{and} $s_*(j) == 0)$} {
   \textbf{break}
    }
   }
    \If{$\vec S_T^{(d)}$ \textbf{equals} $\vec s_*$} {
  $m_i = m_i + 1$\\
 }
 }
  \Return $m_i$
\end{algorithm}

\section{Soft Margin epidemic source detector}
Let $\vec S^{(j)}_{\theta}$ denote $j$-th sample (outcome) obtained by Monte Carlo simulation of contagion process with source node $\theta$ and duration of $T$ time steps. $\vec S^{(j)}_{\theta}$ is one realization of random binary vector $\vec S_{\theta}$ that describes the outcome of an epidemic process. A similarity measure $\varphi : (\vec S_{\theta} \times \vec S_{\theta}) \rightarrow [0, 1 ]$  can  be defined between any two  realizations of $\vec S_{\theta}$. For example,  $\varphi$ can be defined as the Jaccard similarity function:
\begin{equation}
\varphi(\vec S_1, \vec S_2) = \frac{\vec S_1 \cap \vec S_2}{\vec S_1 \cup \vec S_2 } = \frac{\sum_{j = 1}^{N} (S_1(j) = 1 \hspace{2mm} \text{and} \hspace{2mm} S_2(j) = 1) }{\sum_{j = 1}^{N} (S_1(j) = 1 \hspace{2mm} \text{or} \hspace{2mm} S_2(j) = 1) }.
\label{JaccardPHI}
\end{equation}

Moreover, we  can define a discrete random variable $\varphi(\vec s_*, \vec S_{\theta})$ that measures the similarity between fixed realization $\vec s_*$ and random realization from $\vec S_{\theta}$. Let PDF of that random variable be $f_\theta(x)$ where $x = \varphi(\vec s_*, \vec S_{\theta})$. Since $\varphi(\vec s_*, \vec S)$ takes  discrete values, the probability density function is an integral of a range of Dirac delta functions, each positioned at one value of $\varphi(\vec s_*, \vec S)$ and weighted by corresponding probability. 

By using Monte Carlo method we can take PDF definition as an integration problem (\ref{exp}) and sample from this discrete distribution $\{p_1, \ldots, p_d\}$ of  $\varphi(\vec s_*, \vec S)$  to obtain the PDF estimate:
\begin{equation}
f_{\theta}(x) = \int_0^1 p_k \delta(x - x_k)dx \approx \frac{1}{n} \sum_{i = 1}^{n} \delta(x - \varphi(\vec s_*, \vec S_{\theta} ^{(i)}))
\label{Fapprox}
\end{equation}
where $\delta(x)$ denotes the Dirac delta function.

\subsection{Soft Margin estimator}

The Soft Margin estimator is defined as 
\begin{equation}
\hat{P}_a(\vec S = \vec s_* | \Theta = \theta) = \int_0^1 w_a(x)\hat{f}_\theta(x)dx
\label{SMF1}
\end{equation}
where $w_a(x)$ is a weighting function and $\hat{f}_\theta(x)$ is the estimated PDF of the random variable $\varphi(\vec s_*, \vec S_\theta)$. For $w_a(x)$ \citet{Nino} proposed a Gaussian weighting form $w_a(x) = e^{-(x -1)^2 / a ^2}$. 

With Soft Margin approximation the problem definition is altered to estimating the number of realizations with similarity to $\vec s_*$ in the interval around $\varphi = 1$  defined by Gaussian weighting function $w_a(x)$, as opposed to estimating the number of realizations with similarity strictly equal to $\varphi = 1$ like with Direct Monte Carlo method. 

Appropriate values of parameter $a$ can be deduced from contour plot in Figure \ref{wax}. For $a$ close to $1$, Soft Margin approximation includes more asimilar samples.
In the limit where $a \rightarrow 0$ Soft Margin approximation converges to Direct Monte Carlo estimate.

\begin{figure}[H]
\center
\includegraphics[scale=0.6]{/home/iva/dipl/res/sm_benchmark/fi_contour_brew.pdf}
\caption{Contour plot of Gaussian weighting function $w_a(x) = e^{-(x -1)^2 / a ^2}$.}
\label{wax}
\end{figure}

The Soft Margin formula \ref{SMF1} can be further simplified by combining with \ref{Fapprox}:
\begin{equation}
\begin{aligned}
\hat{P}_a(\vec S = \vec s_* | \Theta = \theta) &= \int_0^1 w_a(x)\hat{f}_\theta(x)dx \\ &=
\int_0^1 w_a(x) \frac{1}{n} \sum_{i = 1}^{n}\delta(x - \varphi(\vec s_*, \vec S_{\theta}^{(i)})) dx,
\end{aligned}
\end{equation}
and further by using the property of Dirac delta function $\int_{-\infty}^{\infty} f(x)\delta(x - b)dx = f(b)$:
\begin{equation}
\begin{aligned}
\hat{P}_a(\vec S = \vec s_* | \Theta = \theta) &= \frac{1}{n} \sum_{i = 1}^{n} \int_0^1 w_a(x) \delta(x - \varphi(\vec s_*, \vec S_{\theta}^{(i)})) dx  \\ &= \frac{1}{n} \sum_{i = 1}^{n}  w_a(\varphi(\vec  s_*, \vec S_\theta^{(i)})) \\ &= \frac{1}{n} \sum_{i = 1}^{n} e ^{\frac{(\varphi_i -1)^2}{a ^2}}.
\end{aligned}
\label{sm_est}
\end{equation}


Since our final goal is estimation of probability distribution $P(\Theta = \theta_i | \vec S = \vec s_*) $, for numerical reasons it is wise to use likelihood $n\hat{P}(\Theta = \theta_i \vec S = \vec s_*)$ in the calculation of \ref{MAP_pravi} instead of the estimated probability $\hat{P}(\Theta = \theta_i | \vec S = \vec s_*)$ when the number of simulations $n$ used to estimate $\hat {P}$ is the same for all potential source nodes. 

Note that it's not needed to determine constant $a$ in advance. The parameter $a$ can be chosen as the infinum of the set of parameters for which the source probability distribution estimate $\hat{P_a}(\Theta = \theta_i | \vec S = \vec s_*)$ has converged under the convergence property \ref{DMCConv}. 

Additionally, for a fixed number of simulations $n$, PDF's  based on different parameters $a$ can be estimated with one set of samples by evaluating \ref{sm_est} for different values of parameter $a$. 

\vspace{5mm}
\begin{algorithm}[H]
\label{SMC_alg}
\caption{Soft Margin approximation of $P(\vec S = \vec s_* | \Theta = \theta_i)$ for a fixed source node $\theta_i$.}
 \KwData{$\mathbf{G}$ - network, $(p, q)$ - parameters of the SIR process, 
  $\vec s_*$ - target realization,    
  $T$ - temporal threshold, 
  $\theta_i$ - proposed source node,
  $n$ - number of simulations,
  $a$ - Soft Margin parameter}
 %\KwResult{ $\hat{P_a}(\vec R = \vec r_* | \Theta = \theta_i)$}
 \For{$i = 1$ to $n$} {
    Run SIR simulation $(p, q, \theta_i)$ for $T$ time steps and obtain  
    $\vec S_T^{(i)}$\\
    Calculate and save $\varphi_i = \varphi(\vec s_*, \vec S_T^{(i)})$\\
 }
  Calculate $\hat{P}(\vec S = \vec s_* | \Theta = \theta_i) = \frac{1}{n} \sum_{i = 1}^{n} e ^ {\frac{-(\varphi_i -1)^2}{a^2}}$\\
  \Return $\hat{P}(\vec S = \vec s_* | \Theta = \theta_i)$
\end{algorithm}


\section{Time complexity of Direct Monte Carlo and Soft Margin epidemic source detectors}
The average run time complexity of  Monte Carlo epidemic source detectors Direct Monte Carlo and Soft Margin is $ m n \overline{RT}_{M}$, where $m$ denotes the number of potential sources in the observed  realization, $n$ the number of samples of the random variable $\vec S_{\theta}$  and $\overline{RT}_M$ denotes the average run-time complexity of sampling one realization from contagion process $M$ \cite{Nino}.

Note that in the worst-case scenario the number of potential sources is proportional to the network size, but in reality we are mostly interested in epidemic source detection problems in which the number of potential sources is much smaller than the network size. 

Additionally, different Monte Carlo estimators have different convergence properties with respect to  number of samples $n$. Under convergence conditions \ref{DMCConv} and $c=0.05$ the Soft Margin estimator converges for $n \in [10^4, 10^6]$ on the benchmark lattice dataset on which Direct Monte Carlo requires $n\in[10^6, 10^8]$ simulations for each potential source node, as presented in Figure \ref{bench_sim_no}.

\chapter{Importance sampling based epidemic single source detection}
\label{IS}

\section{Importance sampling}

Importance sampling is a technique for estimating properties of a particular distribution with samples generated from a different distribution than the one of interest. The technique is used with Monte Carlo method as an estimator variance reduction technique since  the new sampling distribution of choice is usually biased towards realizations that have more impact on the parameters being estimated.

Suppose we want to estimate area under $f(x)$ plotted in Figure \ref{densities}. With Monte Carlo technique we sample uniformly at random from $x$ and add each sampled value $f(x)$ to the estimate. The importance of each sample in the estimate depends on the value of the function in that point. To gain better estimate with fewer number of samples one might want to sample $x$ from a density similar to $f(x)$. By sampling from $g(x)$ to estimate area under $f(x)$, values of $x$ that are more included in the estimation  will be sampled more frequently. 
\begin{figure}[h]
\center
\includegraphics[width=0.68\textwidth]{/home/iva/dipl/img/densities.pdf}
\caption{Target density $y = f(x)$ and biased importance density $y = g(x)$.}
\label{densities}
\end{figure}

The method of importance sampling is estimation of the integration problem (\ref{exp}) based on generating a sample $X^{(1)}, \ldots, X^{(n)}$ from a given biased distribution $\textit{g}$ when in fact the samples $X^{(i)}$ come from the target distribution $\textit{f}$:
\begin{equation}
\mathbf{E_f[\textit{h(X)}]} = \int_{X} h(x) f(x) dx = \int_{X} h(x) \frac{f(x)}{g(x)} g(x) dx \approx \frac{1}{n} \sum_{j = 1}^{n} \frac{f(X^{(j)})}{g(X^{(j)})} h(X^{(j)}).
\end{equation}
By choosing to sample from the biased distribution $g$, we are left with the extra weight $w^{(j)} = \frac{f(X^{(j)})}{g(X^{(j)})}$ from the integral. The weight $w^{(j)}$ corrects the bias of the sampling procedure. 

The new estimator converges whatever the choice of distribution $g$, as long as $supp(g) \supset supp(f)$\footnote{$supp(g) = \{x | g(x) \neq 0\}$},  i.e. for each generated sample the weight $w^{(j)}$ has to be finite \cite{RobertMCS}. 

Note the estimation can be done with unbiased estimate
\begin{equation}
\frac{1}{n}\sum_{i=1}^{n}w^{(i)}h(X^{(i)}),
\end{equation}
or with a weighted estimate
\begin{equation}
\frac{\sum_{i=1}^{n}w^{(i)}h(X^{(i)})}{ \sum_{i=1}^{n}w^{(i)}}.
\label{wei_est}
\end{equation}
When using the weighted estimate, we only need to know the ratio $f(x)/g(x)$ up to a multiplicative constant. Although inducing a small bias, the weighted estimate often has a smaller mean squared error than the unbiased one \cite{RobertMCS}.

\subsection{Measuring the quality of importance distribution}
\label{sec_vc2}
By properly choosing $g(\cdot)$ one can reduce the estimator variance substantially. In order to make the estimation error small, one wants to choose $g(x)$ as close in shape to $f(x)h(x)$ as possible. The efficiency of importance distribution is difficult to measure. 

Effective sample size (ESS) is commonly used to measure how different the importance distribution is from the target  distribution. ESS will give the size of an iid sample set with the same variance as the current sample set.
Suppose we have $n$ independent samples generated from $g(\mathbf{x})$. The ESS is then defined as 
\begin{equation}
\text{ESS}(n) = \frac{n}{1 + var_g[w(x)]}.
\end{equation}  
The variance here is estimated as a square of the coefficient of variation of the weights:
\begin{equation}
cv^2 = \frac{\sum_{j=1}^{n} (w^{(j)} - \bar{w})^2}{(n - 1)\bar{w}^2}
\label{cv2}
\end{equation}
where $\bar{w}$ is sample average of the weights $w^{(j)}$. Effective sample as a measure of efficiency can be partially justified by the delta method \cite{Liu}.

\section{Sequential importance sampling}

Since it is not trivial to design a good importance sampling density, especially for high dimensional problems, one may build up the importance density sequentially \cite{Liu}. Suppose we can decompose $x$ as $\mathbf{x} = (x_1, \ldots, x_d)$ where each of the $x_j$ may be multidimensional. That is especially helpful when the state space of $x_{t+1}$ is an augmentation of state $x_{t}$. The importance distribution can then be constructed as 
\begin{equation*}
g(\mathbf{x}) = g_1(x_1) g_2(x_2 | x_1) g_3(x_3 | x_1, x_2) \ldots g_d(x_d | x_1, \ldots, x_{d - 1})
\end{equation*}
With recursive form we hope to obtain some guidance from the target density while building up the  importance density. We can then rewrite the target density as 
\begin{equation*}
f(\mathbf{x}) = f_1(x_1) f_2(x_2 | x_1) f_3(x_3 | x_1, x_2) \ldots f_d(x_d | x_1, \ldots, x_{d - 1})
\end{equation*}
and the weights as 
\begin{equation}
w(\mathbf{x}) = \frac{f_1(x_1) f_2(x_2 | x_1) f_3(x_3 | x_1, x_2) \ldots f_d(x_d | x_1, \ldots, x_{d - 1})}{g_1(x_1) g_2(x_2 | x_1) g_3(x_3 | x_1, x_2) \ldots g_d(x_d | x_1, \ldots, x_{d - 1})}
\end{equation}
which suggests a recursive monitoring and computing of importance weight
\begin{equation}
w_t(\mathbf{x}_t) = w_{t - 1}(\mathbf{x}_{t - 1})\frac{f(x_t | \mathbf{x}_{t - 1})}{g(x_t | \mathbf{x}_{t - 1})}. 
\label{wei_rec}   
\end{equation}

In other words, we build samples and importance weights sequentially. Each partial sample $\mathbf{x}_{t - 1}$ is extended using generator based on transitional importance density $g(x_t | \mathbf{x}_{t - 1})$. For the generated sample $x_1, \ldots x_t$ the target transitional density $f(x_t | \mathbf{x}_{t - 1})$ is calculated. Final importance weight $w_d(\textbf x_t)$ can be calculated using the series of transitional target and importance densities for the particular sample.  
At the end, $w_d(\mathbf x_d)$ is equal to $w(\mathbf{x})$. 

By using the recursive process we can stop generating further components of $\mathbf{x}$ if the partial weight derived from the sequentially generated partial samples is too small in relation to other weights and we can take advantage of current set of samples $x_{t - 1}$ in design of  $g_t(x_t | \mathbf{x}_{t - 1})$.

Finally, the sequential importance sampling method can be defined as $d$ Sequential importance sampling (SIS) steps as presented in Algorithm \ref{SIS_step}. 

\vspace{5mm}
\begin{algorithm}[h]
\caption{Sequential importance sampling (SIS) procedure}
\label{SIS_step}
\KwData{$n$ - number of samples}
\For{$i = 1$ to $n$}{
Initialize $X_{0}^{(i)}.$ 
}
\For{$t = 1$ to $d$}{
\For{$i = 1$ to $n$} {
Draw $X^{(i)}_t$ from $g_t(x_t | \mathbf{X^{(i)}_{t - 1}})$, and let $\mathbf{X}^{(i)}_t = (\mathbf{X}^{(i)}_{t - 1}, X_t^{(i)})$.\\
Compute 
$$w_t(\mathbf{X}^{(i)}_t) = w_{t - 1}(\mathbf{X}^{(i)}_{t - 1})\frac{f(X_t^{(i)} | \mathbf{X}^{(i)}_{t - 1})}{g(X^{(i)}_t | \mathbf{X}^{(i)}_{t - 1})}.$$
}
}
\vspace{5mm}
\end{algorithm}
\vspace{5mm}

\subsection{Improving the SIS procedure with resampling}
When the system grows, the variance of the importance weights $w_t$ increases since the process becomes martingale \cite{Kong94}. After a certain number of steps, many of the weights become very small and a few very large.
In that situation one may want to use a resampling strategy. 

The role of resampling   is to prune away "bad" samples and to split the good ones by rearranging the samples in existing sample set and modifying their weights accordingly. The new set of samples is also properly weighted by new weights with respect to $g$.

The resampling step is done on the existing partial sample set before expanding it with the SIS step before inner loop in Algorithm \ref{SIS_step}.

Two classic resampling techniques - simple random sampling and residual sampling - are presented as Algorithms \ref{SRS} and \ref{ResS}. 
Residual sampling dominates the simple random sampling in having smaller estimator variance \cite{Liu}.

\begin{algorithm}
\caption{Simple random sampling for the SIS procedure}
\label{SRS}
\KwData{$S_t = \{ \mathbf{X}_{t}^{(j)}, j = 1, \ldots, n \}$ - collection of $n$ partial samples of length $t$, which are properly weighted by the collection of weights $W_t = \{w_t^{(j)}, j = 1, \ldots, n\}$ with respect to the density $g$}
Sample a new set of partial samples, $S'_t$ from $S_t$ according to  weights $w_t^{(j)}$.\\
Assign equal weights $W_t / n$, to samples in $S_t'$ where $W_t = \sum_{i = 1}^{n} w_t^{(i)}$.\\
\end{algorithm}

\begin{algorithm}
\caption{Residual resampling for the SIS procedure}
\label{ResS}
\KwData{$S_t = \{ \mathbf{X}_{t}^{(j)}, j = 1, \ldots, n \}$ - collection of $n$ partial samples of length $t$, which are properly weighted by the collection of weights $W_t = \{w_t^{(j)}, j = 1, \ldots, n\}$ with respect to the density $g$}
\For{$j =1 $ to $n$}{
Retain $k_j = \lfloor nw_t^{(*j)}\rfloor$ copies of $\mathbf{X}_t^{(j)}$ where $w_t^{(*j)} = w_t^{(j)} / W_t.$\\}
Let $n_r = n - \sum_{j = 1}^{n} k_j$.\\
Obtain $n_r$ draws from $S_t$ with probabilities proportional to $nw_t^{(*j)}- k_j, \forall j = 1, \ldots n$.\\
Assign equal weights $W_t / n$, to  samples in $S_t'$ where $W_t = \sum_{i = 1}^{n} w_t^{(i)}$.\\
\end{algorithm}

\subsection{Resampling schedule}
The resampling step tends to result in a better group of ancestors so  as to produce better descendants. The success of resampling, however, relies heavily on the Markovian structure among the state variables $x_1, x_2, \ldots x_d$. If resampling from set $\{ \mathbf{x}_{t - 1}^{(j)}, j = 1, \ldots n\}$ is not equivalent to resampling from $\{ x_{t - 1}^{(j)}, j = 1, \ldots, n\}$ -- the set of the "current state", frequent resampling will rapidly impoverish diversity of the partial samples produced earlier. When no simple Markovian structure is present, frequent resampling generally gives bad results.

For this reason, it is desirable to prescribe scheduling for resampling to take place. The resampling schedule can be either deterministic or dynamic. When the schedule is dynamic, some small bias may be introduced.

With a deterministic schedule, we conduct resampling at time $t_0, 2t_0, \ldots,$ where $t_0$ is given in advance. In a dynamic schedule, a sequence of thresholds $c_1, c_2, \ldots, c_d$ is given in advance. We monitor the coefficient of variation of the weights $cv_t^2$ and invoke the resampling step when event $cv_t^2 > c_t$ occurs. A typical sequence of $c_t$ can be $c_t = a + bt^\alpha$.

Increasing $c_t$ after each SIS step makes sense since, as the system evolves, $cv_t^2$ increases stochastically while the variance of importance weights increases. 

\section{Sequential importance sampling epidemic source detector}
Given snapshot $\vec s_*$ that holds all infected nodes up to time $T$ we want to determine the probability $P(\theta_i | \vec S = \vec s_*)$ of an epidemic starting in node $\theta_i$. Since all apriori probabilities $P(\theta_i)$ are the same, we can approximate aposteriori probabilities $P(\vec S = \vec s_* | \theta_i)$ and use them to determine $P(\theta_i | \vec S = \vec s_*)$ as we did in \ref{MAP_pravi}. These aposteriori probabilities were estimated with Direct Monte Carlo and Soft Margin method up to a multiplicative constant which can also be done using Sequential importance sampling technique.

First note the SIS step as defined in Algorithm \ref{SIS_step} is based on densities of a complete history of the process or, at time $t$, all the process steps up to time $t$. The target density is thus the joint probability of all steps taken in the process. Since we are only interested  in the final realization, it makes sense to use target and importance probability distributions  of the form
\begin{equation}
\begin{aligned}
f(s_T) = f_t(i_1, r_1) f_t(i_2, r_2 | i_1, r_1)  f_t(i_3, r_3 | i_2, r_2)  \ldots  f_t(i_T, r_T | i_{T - 1}, r_{T - 1})\\
g(s_T) = g_1(i_1, r_1) g_2(i_2, r_2 | i_1, r_1) g_3(i_3, r_3 | i_2, r_2) \ldots  g_t(i_T, r_T | i_{T - 1}, r_{T - 1})
\end{aligned}
\end{equation}
where $i_k$ denotes a vector of infected nodes after SIS step $k$, and $r_k$ denotes a vector of recovered nodes after step $k$. Note that $i_k \cup r_k = s_k$ and $i_k \cap r_k = \emptyset$.
Each SIS step corresponds to one time step of discrete SIR simulation and 
realization $s_T$ belongs to discrete random variable $\vec S_\theta$.  

Each adjacent element of the sequence $(i_1, r_1), (i_2, r_2), (i_3, r_3), \ldots, (i_T, r_T)$ is connected with one SIR step. These SIR steps build up the target distribution $f(s_T)$ and since each discrete SIR step is described by the same process \ref{transitionsSIR}, the target transitional distributions $f_t(i_k, r_k | i_{k - 1}, r_{k - 1})$ for each SIS step are the same. On the other hand, the SIS procedure let us change the transitional importance distributions $g_k(i_k, r_k | i_{k - 1}, r_{k - 1}), k \in \{1..T\}$ based on the current set of partial samples as long as we are able to calculate target transitional probability $f_t(i_k, r_k | i_{k - 1}, r_{k - 1})$ for each sample.

\subsection{Modelling the target distribution}
We can evaluate the partial target distribution $f_t(i_k, r_k | i_{k - 1}, r_{k - 1})$ in closed form. This is exactly the probability of one time step SIR transition given with Formula \ref{fISS}:
\begin{equation*}
\begin{split}
P(i_k, r_k | i_{k - 1}, r_{k - 1}) =  &\big[\Pi_{v \in E_1} (1 - (1 - p) ^ {nei_{i_{k-1}}(v)})\big]\big[\Pi_{v \in E_2} (1 - p)^{nei_{i_{k-1}}(v)} \big] \\& \cdot
\big[\Pi_{v \in E_3} q\big]\big[ \Pi_{v \in E_4} (1 - q)\big],
\end{split}
%\label{fISS}
\end{equation*}
with events $E_1 - E_4$ as defined in \ref{listE}. 

\subsection{Modelling the importance distribution}
With our sequential sampling procedure we will try to estimate the ratio of realizations at time $T$ that are equal to $\vec s_*$ for some fixed starting node $\theta_i$ up to a multiplicative constant. The ideal importance distribution is biased towards that goal. Since we are building the final source distribution sequentially from partial samples, our biased sampling must sample reasonably well at each step -- it must not be to "slow" or too "fast", especially since it is not clear what samples at mid steps are valuable to us.

However, it is certain we do not want to infect the nodes that were never infected or recovered in the snapshot $\vec s_*$. By excluding those nodes from events $E_1 - E_4$ while retaining fixed $p$ we implicitly make the probability of infection per time step for the nodes in the target realization higher. 

Modified transitions and their corresponding transitional probabilities at each time step can be determined from $i_{k - 1}$, $r_{k - 1}$ and $\vec s_*$. For the SIR model, the transitions are similar to \ref{transitionsSIR}:

\begin{itemize}
\item{$T'_1:$ $\mathbf{if} v \in s_*$ and $v \not\in i_{k - 1}$ and $v \not\in r_{k - 1}$\\  %$v$ is susceptible and a member of target realization\\
 \tab -- infect $v$ with probability $1 - (1 - p) ^ {nei_{i_{k-1}}(v)}$} 
\item{$T'_2:$ $\mathbf{if} v \not\in s_*$ and $v \not\in i_{k - 1}$ and $v \not\in r_{k - 1}$%\\ $=$ $v$ is susceptible not not a member of target realization \\
\\\tab -- infect $v$ with probability $0$}
\item{$T'_3:$ $\mathbf{if}$ $v \in i_{k - 1}$ \\ \tab -- recover node $v$  with probability $q$.}
\end{itemize}

This biased transitional distribution $g_k(i_k, r_k | i_{k - 1}, r_{k - 1})$ defined by one-time-step long transitions $T_1' - T_3'$ can be sampled without conducting any SIR simulations. The sample is generated by traversing the nodes in the target realization $\vec s_*$. If the node is not infected yet, it may be infected if its neighbours are infected with probability  $1 - (1 - p)^{nei_{i_{k - 1}}(v)}$, as in transition $T'_1$.  Otherwise, the recovery process is simulated with probability $q$ for recovery. Since all  transitions are independent, the biased transitional probability can be computed easily. The generator of partial samples governed by  biased transitional distribution $g_k(i_k, r_k | i_{k - 1}, r_{k - 1})$ is presented in Algorithm \ref{biased_trans}. 
 
\vspace{5mm}
\begin{algorithm}[H]
\caption{Partial sample generator based on importance distribution $g_k(i_k, r_k | i_{k - 1}, r_{k - 1})$}
\label{biased_trans}
\KwData{$(p, q)$ - parameters of the SIR process, $i_{k - 1}$ - infected nodes at the beginning of time step $k$, $r_{k - 1}$ recovered nodes at the beginning of time step $k$, $s_*$ - target realization}
\KwResult{$i_k$ - infected nodes after time step $k$, $r_k$ - recovered nodes after time step $k$, $g = g(i_k, r_k | i_{k - 1}, r_{k - 1})$ - probability of generating realizations $i_k$, $r_k$ based on $i_{k - 1}, r_{k - 1}$ from the importance generator}
$i_{k} = i_{k - 1},r_{k} = r_{k - 1}, g = 1$\\
\ForEach{$v \in nei(i_{k - 1}) \cap s_*$}{
\If{$v \not\in i_{k - 1}$ \textbf{and} $v \not\in r_{k - 1}$}{
 $D = |nei_{i_{k - 1}}(v)|$\\
 let \textit{infection} occur with probability $p_v = 1 - (1 - p)^{D}$\\
 \If{infection \textbf{occured}}{
  \textbf{update} $i_{k}(v)$\\
  $ g= g \cdot p_v$
 }\uElse{
  $g = g \cdot (1 - p_v)$
 }
}
}
\ForEach{$v \in i_{k - 1} \cap s_*$}{
 let \textit{recovery} occur with probability $q$\\
 \If{recovery \textbf{occured}}{
  \textbf{update} $i_{k}(v)$ and $r_{k}(v)$\\
  $g = g \cdot q$
 }\uElse{
  $g = g \cdot (1 - q)$
 }
}
\Return{$(i_k, r_k, g)$}
\end{algorithm}
\vspace{5mm} 

In order to calculate importance weights $w_k(i_k, r_k)$ recursively like in \ref{wei_rec}, one needs to be able to compute target transitional probability $f_t(i_k, r_k | i_{k - 1}, r_{k - 1})$ based on known partial sample $i_k, r_k$ and its base $i_{k - 1}, r_{k - 1}$. This computation can be done by traversing potential active nodes in the SIR process: infected nodes at the beginning 
of time step $k$ defined by set $i_{k - 1}$ for recovery, and their susceptible
 neighbours for 
 infection. Computation of target transitional probability $f_t(i_k, r_k|i_{k - 1}, r_{k - 1})$ is described by Algorithm \ref{target_trans}.

Note the biased generator in Algorithm \ref{biased_trans} can be run with arbitrary parameters $p$ and $q$. The samples generated with arbitrary $p$ and $q$ can be included in the set of samples as long as the transitional target distribution  can be calculated. 

\vspace{5mm}
\begin{algorithm}[H]
\caption{Computation of  $f(i_k, r_k | i_{k - 1}, r_{k - 1})$ - target transitional probability of generated partial sample}
\label{target_trans}
\KwData{$(p, q)$ - parameters of the SIR process, $i_{k - 1}$ - infected nodes at the beginning of step $k$, $r_{k - 1}$ recovered nodes at the beginning of step $k$, $i_k$ - infected nodes after step $k$, $r_k$ - recovered nodes after step $k$, $s_*$ - target realization}
$f = 1$\\
\ForEach{$v \in nei(i_{k - 1})$}{
\If{$v \not\in i_{k - 1}$ \textbf{and} $v \not \in r_{k - 1}$} {
$D = |nei_{ i_{k - 1}}(v)|$\\
\If{$v \in i_{k}$}{
$f = f \cdot [1 - (1 - p)^D]$\\ 
}\Else{
 $f = f \cdot (1 - p)^D$\\
}
}
}
\ForEach{$v \in i_{k - 1}$}{
\If{$v \in r_{k}$}{
 $f = f \cdot q$
}
\If{$v \not\in r_{k}$}{
 $f = f \cdot (1 - q)$ 
}
}
\Return{f}
\end{algorithm}
\vspace{5mm}

\subsection{Building the epidemic source detector}

To estimate ratio of realizations that are equal to the target realization $\vec s_*$ up to multiplicative constant at time $T$ for a fixed source $\theta$, SIS step (Algorithm \ref{SIS_step}) will be conducted $T$ times. In each step, the samples are generated using importance distribution $g(i_{k}, r_{k} | i_{k - 1}, r_{k - 1})$ and the transitional target probability $f(i_{k}, r_{k} | i_{k - 1}, r_{k - 1})$ as well as transitional importance probability $g(i_{k}, r_{k} | i_{k - 1}, r_{k - 1})$ for the generated sample are calculated. Based on the data one can obtain the transitional partial weight $w_k$ for said partial sample.

 After expanding the partial samples $T$ times, one obtains the estimate on number of realizations equal to the target realization $\vec s_*$ by taking weighted average of all weights corresponding to the realization hits in the final sample set,
\begin{equation}
m_i = \frac{1}{n} \sum_{j = 1}^{n} w_T^{(j)} 1\{s_T^{(j)} = \vec s_*\},
\label{seq_est}
\end{equation}
where $s_T^{(j)} = i_T^{(j)} \cup r_T^{(j)}$ corresponds to one realization of discrete random variable $\vec S_\theta$. Finally, the pseudocode for the source detector based on sequential importance sampling is presented in Algorithm \ref{sis_dec}.
\begin{algorithm}[h]
\caption{Sequential importance sampling estimation of expected number of realizations completely corresponding to $\vec s_*$ after $T$ time steps for a fixed source node $\theta_i$}
\label{sis_dec}
\KwData{$\mathbf{G}$ - network, $(p, q)$ - parameters of the SIR process, 
  $\theta_i$ - proposed source node,  
  $\vec s_*$ - target realization,    
  $T$ - temporal threshold, 
  $n$ - number of samples
  }
  \For{$j = 1$ to $n$}{
Initialize $i^{(j)}_0 = \{\theta_i\}, r^{(j)}_0 = \emptyset, w_0^{(j)} = 1$ 
}
\For{$t = 1$ to $T$}{
\textbf{resample}$(\{(i_{t - 1}^{(j)}, r_{t - 1}^{(j)}) \hspace{2mm}\forall j \in [1..n]\})$\\
\For{$j = 1$ to $n$} {
Draw $i^{(j)}_t, r^{(j)}_t$ from $g_t(i_t, r_t | i^{(j)}_{t - 1}, r^{(j)}_{t -1})$ with Algorithm \ref{biased_trans}.\\
Compute $f_t(i^{(j)}_{t}, r^{(j)}_{t} | i^{(j)}_{t - 1}, r^{(j)}_{t - 1})$ with Algorithm \ref{target_trans}. \\
Compute 
$$w_t(i_{t}^{(j)}, r_{t}^{(j)}) = w_{t - 1}(i_{t - 1}^{(j)}, r_{t - 1}^{(j)})\frac{f_t(i_t^{(j)}, r_t^{(j)} | i_{t - 1}^{(j)}, r_{t - 1}^{(j)})}{g_t(i_t^{(j)}, r_t^{(j)} | i_{t - 1}^{(j)}, r_{t -1}^{(j)})}.$$
}}

$m_i = 0$ \\
\For{$j = 1$ to $n$}{
\If{$i_T^{(j)} \cup r_T^{(j)}$ \textbf{equals} $\vec s_*$}{
$m_i = m_i + w_T(i_{T}^{(j)}, r_{T}^{(j)})$
}
}
\Return{$(m_i)$}
\end{algorithm}

It is obvious the accuracy of the estimate \ref{seq_est} depends on number of realizations in the sample set that are equal to $\vec s_*$ after final SIS step. To increase this number, one can alter the generation of samples in the final SIS step to maximize the number of realization hits. This can safely be done by altering the importance generator and making all eligible nodes infected with probability $p=1$ at the final step of the SIS procedure. 

It may also be reasonable to increase $p$ at each step of the SIS procedure but it is not clear when or in what volume this should be done. By using too high or too low $p$ at earlier SIS steps, obtained samples will in general have really small final weight $w_T$ since its target transitional probability will be close to $0$.

Additionally, one might want to use a resampling technique for simulations with many  SIS steps or to fix said small weights. This has to be done carefully too since our target event is rare and weights $w$ are naturally small. By using the resampling schedule based on  ESS or $vc^2$, the decision on when to resample is not governed by absolute value of the weights, but rather on their coefficient of variation. For example, resampling can be invoked in SIS step $t$ when  $vc^2(\mathbf{w}_{t - 1}) > 2^t$.

\subsection{Soft Margin SIS source detector}
Incorporating Soft Margin ideas to sequential importance sampling based epidemic source detector allows us to use all generated samples in the estimation of $m_i$, as opposed to using only the ones completely corresponding to the target realization $\vec s_*$ as presented in Algorithm \ref{sis_dec}.

Using the set of generated biased samples $\{(i_{T}^{(j)}, r_{T}^{(j)}) \hspace{2mm}\forall j \in [1..n]\}$, we approximate $m_i$ with  
\begin{equation}
m_i \approx \sum_{j = 1}^{n} w_T(i_{T}^{(j)}, r_{T}^{(j)})  e^{\frac{(\varphi_j - 1)^2}{a^2}}
\label{SIS_sm}
\end{equation}
 
where $e^{\frac{(\varphi_j - 1)^2}{a^2}}$ corresponds to Gaussian weighting function  with $\varphi_i = \varphi(s_T^{(j)}, \vec s_*)$ defined as Jaccard similarity \ref{JaccardPHI}.

Compared to Soft Margin detector described in Algorithm \ref{SMC_alg}, Soft Margin SIS detector uses the set of biased samples and  will have  smaller estimator variance for the same parameter $a$ and the same number of samples. Additionally, since all biased samples are realizations equal to some subset  of $\vec s_*$ , the realizations containing nodes not in $\vec s_*$ are excluded from approximation \ref{SIS_sm}, as opposed to Soft Margin approximation where these realizations are also being used in the approximation.

\subsection{Time and space complexity of SIS source detector}

To generate estimation of source probability distribution $P(\Theta = \theta_i | \vec S = \vec s_*)$ one needs to run Sequential importance sampling procedure as described in Algorithm \ref{sis_dec} $m$ times where $m$ denotes the number of potential sources in the observed realization, i.e. number of realizations of random variable $\Theta$.

Within each Sequential importance sampling procedure, partial sampling step is conducted $T$ times, with each time $n$ partial samples being extended with the biased generator Algorithm $\ref{biased_trans}$.

Time complexity of generating one full sample (running Algorithm \ref{biased_trans} $T$ times) is $O(m \text{min}(T, \frac{1}{q}))$ 
since one tries to infect or recover at most once no more than each node in the  target realization $\vec s_*$ which is  at most $m$ nodes and the expected number of time steps an infected node remains active is $\text{min}(T, \frac{1}{q})$. Infecting or recovering process is simulated by regular uneven dice throw. 

Each extension of a partial sample is followed by calculation  of target transitional probability based on generated partial sample presented in Algorithm \ref{target_trans}. The time complexity of running this algorithm  for each extension of a partial sample is $O(m\langle d \rangle \text{min}(T, \frac{1}{q}))$ where $\langle d \rangle$ denotes average node degree. The term $m \langle d \rangle$ corresponds to  average upper bound on the size of set $nei_{i_{k - 1}} \cup i_{k - 1}$ of infected nodes and their neighbours. This is exactly the set being traversed in the calculation of target transitional probability. Each infected node is expected to be in that set for min$(T, \frac{1}{q})$ time steps.

Finally, generating estimation $P(\Theta = \theta_i | \vec S = \vec s_*)$ for all nodes in $\Theta$  has time complexity $O(nm^2\langle d \rangle \text{min}(T, \frac{1}{q}))$ since Algorithm \ref{biased_trans}  and Algorithm \ref{target_trans} are being run $Tn$ times where $n$ is the number of samples being used.

While the average time complexity of Direct Monte Carlo and Soft Margin Monte Carlo is $O(nm E[X]\langle d \rangle \text{min}(T, \frac{1}{q}))$ for the SIR model, comparing time complexity of the SIS procedure to Direct Monte Carlo and Soft Margin Monte Carlo is reduced down to where the size of the realization $m$ lays in the distribution of epidemic size for the particular epidemic environment whose expectation is $E[X]$.

Introducing resampling to SIS detector does not influence the upper bound time complexity. 
Introducing Soft Margin estimation to SIS detector does not influence the upper bound time complexity either.

When talking about time complexity, one has to mention the range of number of samples for which the estimators typically converge. Breakdown on the converging number of simulations for epidemic dynamics  on  a lattice network is presented in Figure \ref{bench_sim_no} and analysed in the next chapter. 

The SIS detector can also be parallelised by assigning each worker a set of samples to expand. To enable resampling one needs to aggregate the partial weights for all partial samples corresponding to the same potential source node at each SIS step. In that situation a worker may be assigned a set of potential source nodes for which it must estimate the expected number of target realizations $m_i$. 

Regarding space complexity, sequential importance sampling requires a stored graph in a form of adjacency list and $n$ stored partial samples represented by $3$ bitsets. The  overall space complexity is $O(\text{max}(L, n)),$ where $L$ is number of links in the network and $n$ is number of samples.
 
\chapter{Analysis of epidemic source detectors on the benchmark dataset}
\label{Bench}

Before analysing performance of MAP-based epidemic source detectors it is important to note that the inverse problem of finding the epidemic source is ill-posed \cite{Parker:1994:MDS:174769} while there might not exist a unique solution and the solution may change drastically with  small change in  initial conditions. For this reason a high accuracy in terms of determining the true source node  cannot be expected.

 The main goal in designing new epidemic source detectors is obtaining the results in terms of accuracy as similar to Direct Monte Carlo as possible with better convergence properties and shorter time of execution. 

\section{Benchmark dataset}

\citet{Nino} provided a dataset of SIR realizations along with their estimations obtained with Direct Monte Carlo for $4$ classes of detection problems based on SIR parameters $(p, q, T)$: $A = (p=0.3, q=0.3, T=5), B = (p=0.3, q=0.7, T=5), C = (p=0.7, q=0.3, T=5)$ and  $D = (p=0.7, q=0.7, T=5)$. The benchmark dataset contains $160$ such realizations on the lattice of size $30 \times 30$. 


  Different classes of SIR parameters are used in the dataset since each class yields epidemics of different expected size. Moreover, the source nodes of epidemics on the grid network for classes of SIR parameters $A$ and $B$  present lower detectability than source nodes of epidemics on the grid network for classes of SIR parameters $C$ and $D$ since they belong to low to medium and medium to high detectability zone, respectively, as presented in Figure \ref{entropy_zones}. Source nodes of epidemics governed by SIR parameters in classes $A$ and $B$ thus produce lower accuracy on average.

 The benchmark estimations of \citet{Nino} were obtained with Direct Monte Carlo   under convergence conditions \ref{DMCConv} with $c = 0.05$.
 These estimations provided  along with the benchmark dataset can be used as a benchmark detector to measure similarity of other  MAP based epidemic source detectors with Direct Monte Carlo. 
 
 The similarity and performance of source detectors will be compared using accuracy, accuracy w.r.t. MAP estimation of Direct Monte Carlo benchmark detector and distribution of required number of samples for which the estimations converge. 
 
Direct Monte Carlo, Soft Margin and Sequential importance sampling based epidemic source detectors were implemented in C++ and parallelized using  OpenMPI \cite{open_mpi} library. The complete source code is hosted on GitHub \footnote{https://github.com/imih/cmplx}.


\section{Correctness of  Direct Monte Carlo implementation}

 Direct Monte Carlo implementation was ran on the benchmark dataset with convergence conditions \ref{DMCConv}, $c = 0.05$. 
The correctness of Direct Monte Carlo implementation is confirmed with coinciding accuracies of Direct Monte Carlo implementation and the benchmark detector as presented in Figure \ref{DM_impl}.

 The number of simulations required to fulfil convergence conditions was typically in the interval $n \in [10^6, 10^9]$ as presented in Figure \ref{bench_sim_no_mali} and reported by \citet{Nino} for their Direct Monte Carlo implementation referred in the plot as the benchmark detector.

\begin{figure}[H]
\begin{minipage}{\textwidth}
\center
\includegraphics[width=0.65\linewidth]{/home/iva/dipl/res/sm_benchmark/dm_sm_acc.pdf}
\captionof{figure}{Accuracy of Direct Monte Carlo and Soft Margin implementations on the benchmark dataset with convergence conditions \ref{DMCConv}, $c = 0.05$. For Soft Margin fixed parameter $a=1/2^5$ was used. Classes A, B, C, D correspond to classes of SIR parameters $A = (p=0.3, q=0.3, T=5), B = (p=0.3, q=0.7, T=5), C = (p=0.7, q=0.3, T=5), D = (p=0.7, q=0.7, T=5)$.}
\label{DM_impl}
\end{minipage}
\end{figure}

\begin{figure}[h]
\begin{minipage}{\textwidth}
\center
\includegraphics[width=\textwidth]
{/home/iva/dipl/res/seq_benchmark/bench_sim_no_mali.pdf}
\caption{Distribution of number of samples needed for detectors to converge on the benchmark dataset under convergence conditions \ref{DMCConv}, $c=0.05$  for Direct Monte Carlo and Soft Margin implementations and Direct Monte Carlo benchmark detector.}
\label{bench_sim_no_mali}
\end{minipage}
\end{figure}
 
\section{Correctness of Soft Margin implementation}
To show correctness of the Soft Margin implementation the detector was ran on the benchmark dataset. To make the error in approximation the same for all benchmark instances fixed parameter $a = \frac{1}{2^5}$ was used. 

Since Soft Margin provides an approximation of Direct Monte Carlo method it is expected that  Soft Margin will yield results with accuracy (governed by parameter $a$) not higher  than the accuracy of Direct Monte Carlo, as presented in Figure \ref{DM_impl}.

 The Soft Margin estimations follow the estimations of Direct Monte Carlo fairly well, especially for classes of SIR parameters $C$ and $D$ on which the source node is highly detectable, as presented in Figure \ref{SM_MAP}. This is in accordance to results obtained by Soft Margin implementation of \citet{Nino} referred in the plot as the Soft Margin benchmark detector. The Soft Margin benchmark detector was reported to be ran under the same convergence conditions as the implemented Soft Margin detector.

%Furthermore, the Soft Margin implementation is  comparable to benchmark Soft Margin implementation of \citet{Nino} in terms of its similarity to Direct Monte Carlo benchmark results, as presented in Figure \ref{SM_MAP}. 
\begin{figure}[h]
\begin{minipage}{\textwidth}
\center
\includegraphics[width=0.65\linewidth]{/home/iva/dipl/res/sm_benchmark/sm_dm_map_acc.pdf}
\captionof{figure}{MAP accuracy of Soft Margin detector of \citet{Nino}, Direct Monte Carlo and Soft Margin implementations. MAP accuracy is accuracy relative to MAP estimation of the benchmark detector. The simulations were held with convergence conditions \ref{DMCConv}, $c = 0.05$ and fixed parameter $a=1/2^5$ for Soft Margin. Classes A, B, C, D correspond to classes of SIR parameters $A = (p=0.3, q=0.3, T=5), B = (p=0.3, q=0.7, T=5), C = (p=0.7, q=0.3, T=5), D = (p=0.7, q=0.7, T=5)$.}
\label{SM_MAP}
\end{minipage}
\end{figure}

Soft Margin requires fewer number of simulations for estimations to converge.
Converging number of simulations $n$ was selected based on  convergence conditions \ref{DMCConv}, $c=0.05$ and they were tipically in the range $n \in [10^4, 10^6]$ as presented in Figure \ref{bench_sim_no_mali} and reported by \citet{Nino} for their Soft Margin implementation. 

\section{Sequential importance sampling epidemic source detector}

Sequential importance sampling was run under importance sampling distribution defined by biased generator presented in Algorithm \ref{biased_trans} with the following properties:
\begin{itemize}
\item{parameter $p$ is fixed in steps $t<T$ and $p=1$ in the final step $t = T$,}
\item{parameter $q$ is fixed,}
\item{at each step, only nodes that are in the given final realization may be infected,}
\item{nodes that are infected may be recovered with probability $q$.}
\item{the simulations are held under  Direct Monte Carlo convergence conditions \ref{DMCConv} with $c = 0.05$, starting from $n = 10^4$ samples.}
\end{itemize}

To show correctness of Sequential importance sampling detector accuracy of estimations was compared with estimations of the benchmark detector. 
Accuracies for benchmark detector and Sequential importance sampling detector follow similar pattern overall and for each of the four classes of SIR parameters, as presented in Figure \ref{accuracy_true_sis}. For classes $A$ and $B$ they are low, and for classes  $C$ and $D$ they are high. 

Compared to accuracy of the Soft Margin detector, Sequential importance sampling outperforms the Soft Margin in terms of accuracy which is especially  highlighted for epidemics on low detectability SIR parameters of classes $A$ and $B$. For classes $C$ and $D$ that belong to medium to high detectability zone of parameters the source can also be successfully estimated by the Soft Margin detector.

\begin{figure}[H]
\begin{minipage}{\linewidth}
\center
\includegraphics[width=0.70\linewidth]{/home/iva/dipl/res/seq_benchmark/acc_true_sve_sis.pdf}
\captionof{figure}{Accuracy of Sequential importance sampling detector  on the benchmark dataset with convergence conditions \ref{DMCConv}, $c=0.05$ for classes of SIR parameters  $A = (p=0.3, q=0.3, T=5), B = (p=0.3, q=0.7, T=5), C = (p=0.7, q=0.3, T=5), D = (p=0.7, q=0.7, T=5)$.}
\label{accuracy_true_sis}
\end{minipage}
\end{figure}

To present the overall similarity of estimations of Sequential importance sampling detector to the ones of the benchmark detector, MAP accuracy is used. This accuracy refers to the portion of MAP estimations that are equal to corresponding MAP estimations of the benchmark detector provided in the benchmark dataset. 

\citet{Nino} provided MAP accuracies for range of epidemic source detectors out of which their Soft Margin implementation yields highest MAP accuracy on the benchmark dataset. 

Comparison of MAP accuracy for Soft Margin detector and SIS detector presented in Figure \ref{map_map_acc_sis} shows how Sequential importance sampling detector outperforms Soft Margin in having higher MAP accuracy.

In other words, the estimations of Sequential importance sampling detector are  more similar to Direct Monte Carlo estimations than Soft Margin approximations are. This makes sense since Soft Margin method is the approximation of Direct Monte Carlo. Sequential importance sampling detector differs from Direct Monte Carlo only by smaller estimator variance while still being a valid Monte Carlo method.

\begin{figure}[H]
\begin{minipage}{\linewidth}
\center
\includegraphics[width=0.85\linewidth]{/home/iva/dipl/res/seq_benchmark/MAP_MAP_Acc_sis.pdf}
\captionof{figure}{MAP accuracy of Sequential importance sampling detector on the benchmark dataset with convergence conditions \ref{DMCConv}, $c=0.05$ for classes of SIR parameters  $A = (p=0.3, q=0.3, T=5), B = (p=0.3, q=0.7, T=5), C = (p=0.7, q=0.3, T=5), D = (p=0.7, q=0.7, T=5)$.}
\label{map_map_acc_sis}
\end{minipage}
\end{figure}
\begin{figure}[H]
\begin{minipage}{0.5\textwidth}
\center
\includegraphics[width=\textwidth]{/home/iva/dipl/res/seq_benchmark/MAP_sve_A.pdf}
\end{minipage}
\begin{minipage}{0.5\textwidth}
\center
\includegraphics[width=\textwidth]{/home/iva/dipl/res/seq_benchmark/MAP_sve_B.pdf}
\end{minipage}
\begin{minipage}{0.5\textwidth}
\center
\includegraphics[width=\textwidth]{/home/iva/dipl/res/seq_benchmark/MAP_sve_C.pdf}
\end{minipage}
\begin{minipage}{0.5\textwidth}
\center
\includegraphics[width=\textwidth]{/home/iva/dipl/res/seq_benchmark/MAP_sve_D.pdf}
\end{minipage}
\caption{Relative error of estimated MAP source probability w.r.t the estimated MAP probability of the benchmark detector for a range of Sequential importance sampling detectors grouped by classes of SIR parameters  $A = (p=0.3, q=0.3, T=5), B = (p=0.3, q=0.7, T=5), C = (p=0.7, q=0.3, T=5), D = (p=0.7, q=0.7, T=5)$.}
\label{rel_err}
\end{figure}

The similarity between estimations obtained with Sequential importance sampling and Direct Monte Carlo detector also presents itself as a low relative error of MAP probability estimation w. r. t. benchmark Direct Monte Carlo estimated MAP probability across all classes of SIR parameters as presented in Figure \ref{rel_err}. The relative error of Sequential importance sampling is substantially smaller than the same error for Soft Margin detector across all classes of SIR parameters.


One crucial metric of time complexity for Monte Carlo based techniques is the distribution of number of samples required for detector to converge under convergence conditions \ref{DMCConv}, $c=0.05$.  The distribution of converging number of samples for estimations on the benchmark dataset reveals Sequential importance sampling detector requires between $10^4$ and $10^6$ samples to converge on average for  benchmark source detection problems on the lattice network as presented in Figure \ref{bench_sim_no}, while Direct Monte Carlo detector requires at least $10^6$ simulations for estimations to converge on the same dataset. 

\begin{figure}[h]
\begin{minipage}{\textwidth}
\includegraphics[width=\textwidth]
{/home/iva/dipl/res/seq_benchmark/bench_sim_no_svi_lines.pdf}
\caption{Distribution of number of samples required for detectors to converge on the benchmark dataset under convergence conditions \ref{DMCConv}, $c=0.05$  for a  range of Sequential importance sampling detectors on the benchmark dataset.}
\label{bench_sim_no}
\end{minipage}
\end{figure}

Let's examine the form of distribution of required converging number of samples for Direct Monte Carlo benchmark detector, Soft Margin and Sequential importance sampling detectors in more detail.  The distributions of required converging number of samples for Direct Monte Carlo and Soft Margin take similar form. Introduction of the Soft Margin approximation shifted left  the distribution of converging number of samples for one order of size. Around $50 \%$ percent detection problem estimations on the benchmark dataset still require $10^6$ samples for estimation to converge.

On the other hand,  using sequential importance sampling  significantly alters the distribution of required converging number of samples that now takes exponentially decreasing form in which between $60 \%$ and $80\%$ percent of the detection problems required less than  $10^5$ samples to converge.

\section{Sequential importance sampling detector with resampling}

To examine the impact of resampling techniques on Sequential importance sampling detector, simple random sampling and residual sampling were incorporated. The resampling schedule was based on coefficient of variation of the weights $cv^2$ defined as a measure of effectiveness of importance distribution in \ref{cv2}.

Resampling is invoked before the generation of new partial samples at $t$-th SIS step as marked in Algorithm \ref{sis_dec} if it holds $vc^2(\textbf{w}_{t - 1}) \geq 2^t$ where $\textbf{w}_{t - 1}$ is set of weights corresponding to partial samples generated at the previous SIS step.
The variation of weights $cv^2$ increases drastically after each SIS step on the benchmark dataset and resampling limits the increase of variation, as presented in Figure \ref{vc2_distr}.

\begin{figure}[H]
\center
\begin{minipage}{\textwidth}
\center
\includegraphics[width = 0.6\textwidth]{/home/iva/dipl/res/seq_benchmark/none_vc2.pdf}
\end{minipage}
\begin{minipage}{\textwidth}
\center
\includegraphics[width = 0.6\textwidth]{/home/iva/dipl/res/seq_benchmark/srs_vc2.pdf}
\end{minipage}
\caption{Distribution of squared variation of weights $cv^2$ across discrete time steps $T$ for detection problems in the benchmark dataset estimated with Sequential importance sampling detector  without resampling and with simple random sampling using $n=10^5$ samples. 
%Resampling is triggered if it holds $vc^2(\textbf{w}_{t - 1}) \geq 2^t$.
} 
\label{vc2_distr}
\end{figure}


%Using resampling tends to result in a better group of ancestors so as to produce better descendants. 

Incorporating resampling step to Sequential importance sampling detector does not effect accuracy nor MAP accuracy on average for the benchmark dataset as presented in Figure \ref{accuracy_true} and  \ref{map_map_acc}, respectively. 

\begin{figure}[H]
\begin{minipage}{\linewidth}
\includegraphics[width=\linewidth]{/home/iva/dipl/res/seq_benchmark/accuracy_true_sve.pdf}
\captionof{figure}{Accuracy of Sequential importance sampling detectors  on the benchmark dataset with convergence conditions \ref{DMCConv}, $c=0.05$ for classes of SIR parameters  $A = (p=0.3, q=0.3, T=5), B = (p=0.3, q=0.7, T=5), C = (p=0.7, q=0.3, T=5), D = (p=0.7, q=0.7, T=5)$.}
\label{accuracy_true}
\end{minipage}
\end{figure}

\begin{figure}[H]
\begin{minipage}{\linewidth}
\includegraphics[width=\linewidth]{/home/iva/dipl/res/seq_benchmark/MAP_MAP_Acc_svi.pdf}
\captionof{figure}{MAP accuracy of Sequential importance sampling detectors on the benchmark dataset with convergence conditions \ref{DMCConv}, $c=0.05$ for classes of SIR parameters  $A = (p=0.3, q=0.3, T=5), B = (p=0.3, q=0.7, T=5), C = (p=0.7, q=0.3, T=5), D = (p=0.7, q=0.7, T=5)$.}
\label{map_map_acc}
\end{minipage}
\end{figure}

However, incorporating resampling technique introduces a small bias in the estimator visible as a change in distribution of relative error of MAP probability as presented in Figure \ref{rel_err}. The relative MAP probability error also reveals smaller estimator variance of residual sampling compared to simple random sampling on estimations for classes of SIR parameters $B$ and $D$ with high recovery rate.

The impact of resampling on the source detector is also visible in higher ratio of benchmark instances requiring smaller number of drawn samples to converge, as presented in Figure  \ref{bench_sim_no}.

It is interesting to observe accuracy  in the group of detection problems converging with the same number of samples. 
This group accuracy presented in Figure \ref{bench_sim_acc} shows the SIS detector with incorporated resampling methods observes higher accuracy for problems requiring less than $10^5$ samples to converge compared to Soft Margin detector on corresponding group of detection problems even though this group of detection problems for SIS detectors is bigger than the corresponding group of detection problems requiring the same number of simulations to converge with Soft Margin detector as presented in Figure \ref{bench_sim_no}. 

\begin{figure}[h]
\center
\includegraphics[width=\textwidth]
{/home/iva/dipl/res/seq_benchmark/sim_acc_svi.pdf}
\caption{Accuracy of range of Sequential importance sampling detectors on the benchmark dataset calculated seperately for each group of benchmark detection problems corresponding to the same number of samples their estimation required to converge under convergence conditions \ref{DMCConv}, $c=0.05$.}
\label{bench_sim_acc}
\end{figure}

Additionally, with group accuracy minor difference in variance of estimators incorporating simple random sampling and residual sampling is confirmed.

\section{Sequential importance sampling and Soft Margin} 

Adding Soft Margin approximation to Sequential importance sampling detector means evaluating final samples with Gaussian weighting as presented in \ref{SIS_sm}. We will observe two detectors, both with fixed parameters $a = \frac{1}{2^5}$ and $a=\frac{1}{2^3}$. 

Even though Soft Margin SIS detector with $a=\frac{1}{2^3}$ converges faster as presented in Figure \ref{bench_sim_no}, its overall accuracy is lower  than the accuracy of other SIS detectors, as presented in Figures \ref{accuracy_true}, \ref{map_map_acc}, \ref{rel_err}, \ref{bench_sim_acc}.

On the other hand, the Soft Margin SIS detector with $a = \frac{1}{2^5}$ outperforms benchmark detector and other SIS detectors with higher accuracy as presented in Figure \ref{accuracy_true}. This detector also converges more quickly  on the benchmark dataset as presented in Figure \ref{bench_sim_no}. 

Moreover, the group accuracy of detection problems requiring less than $10^5$ samples to converge for Soft Margin SIS detector is higher than the same accuracy for other SIS detectors and Soft Margin as presented in Figure \ref{bench_sim_acc}.

\section{Experimental execution time}

For Soft Margin, Sequential importance sampling and Soft Margin SIS detector time execution was measured and compared.  Execution time was measured on $12$ cpu cores on  Intel(R) Xeon CPU  E5645 processor, 2.40GHz each. Detection was executed on the benchmark dataset with $n = 10^4$ samples using $16$ MPI processes. The results are presented in Table \ref{execution}

\begin{table}[H]
\center
\begin{tabular}{|c|c|c|c|}
\hline 
  & Soft Margin & Sequential importance sampling & Soft Margin SIS \\ 
\hline 
Min &  0.1170  s & 2.460 s & 2.457  s\\ 
\hline 
Median &  0.9556 s & 4.154  s & 4.247 s  \\ 
\hline 
Mean &  91.1847  s & 4.708 s    & 4.703  s \\ 
\hline 
Max & 666.6846 s & 9.225  s& 9.239  s
 \\ 
\hline 
\end{tabular} 
\caption{Execution times in seconds for Soft Margin and Soft Margin SIS detectors on the benchmark dataset estimated with $n = 10^4$ samples per potential source node.}
\label{execution}
\end{table} 

Execution time of Soft Margin variates depending on expected epidemic size, while Sequential importance sampling detectors sample epidemics only from the subset of infected nodes in the given snapshot $\vec s_*$ making variation in execution time smaller.

The execution times of Sequential importance sampling and Soft Margin SIS detectors do not differ substantially.

\chapter{Detectability of patient zero} 
\label{Det}

The source detectability $D(\vec s_*)$ can be defined via Shannon entropy $H(\vec s_*)$  of the estimated source probability distribution $P(\Theta = \theta_i |\vec{S} = \vec{s_*})$ normalized by entropy of the uniform distribution  as  $D(\vec{s_*}) = 1 - H(\vec{s_*})$ \cite{Nino}.

 When entropy $H(\vec s_*)$ of $P(\Theta = \theta_i | \vec{S} = \vec{s_*})$ is close to $1$, the detector has estimated that all potential nodes have the same probability of being the source node of observed epidemic, while low entropy $H$ corresponds to the case where  detector filtered out a single node or a few nodes as  potential epidemic sources. 

Apart from entropical detectability,  when talking about ability the node can be detected by the source detector as an epidemic source, accuracy of the MAP estimator and number of  samples that are on average required for  estimation to converge play an important role.

Accuracy of  estimation is crucial since it measures the overall successfulness of the estimation  for the particular epidemic environment -- network topology, source node, epidemic model, its parameters and, consequently, average epidemic size. At the same time, the number of samples a detector requires for its estimations to converge gives a picture of the detection complexity in terms of state space size for the particular epidemic environment.

\section{Detectability based on parameters of the SIR model}

To show how detectability presents itself in the parametric space of SIR parameters $p$ and $q$, several simulations were ran using Soft Margin SIS detector with fixed $a=\frac{1}{2^5}$. The Soft Margin SIS detector was chosen since it produced the highest accuracy on the benchmark dataset as presented in Figure \ref{accuracy_true}. 

For each set of parameters $(p, q)$ and various $4$-connected lattice networks of different size, $50$ epidemic simulations were conducted  for $T=5$ time steps starting from the central node in the lattice network. All epidemic simulations of size $1$ were excluded. Soft Margin SIS detector was ran on each epidemic simulation with converging conditions \ref{DMCConv}, $c=0.05$ and number of samples in range $n \in [10^4, 10^6]$.

Influence of parameters $p$ and $q$ on entropical detectability for the SIR model is presented in Figure \ref{entropy_zones}. For simulations on lattice of size $30 \times 30$  we observe the existence of different detectability regimes (or entropy regions), as reported by \citet{Nino}.  Three entropy regions are observed: low detectability-high entropy region $(p < 0.2)$, intermediate detectability - intermediate entropy region $(0.2 < p < 0.7)$ and high detectability-low entropy region $(p > 0.7)$.  The entropy regions are similarly distributed for different  values of $q \in \{0, 0.5, 1\}$.

\begin{figure}[H]
\center
\includegraphics[width=\textwidth]{/home/iva/dipl/res/supfig12/SIRGridBig.pdf}
\caption{Violin plots of estimated entropy distribution for source probability distributions on  $4$-connected lattice $30 \times 30$ estimated with Soft Margin SIS method with $10^4 - 10^6$ samples and fixed $a=\frac{1}{2^5}$ under  SIR model with different parameters $p$ in range $0.1 - 0.9$, $q = \{0, 0.5, 1\}$ and $T = 5$. }
\label{entropy_zones}
\end{figure}

\newpage
Accuracy of source detections, as presented in Figure \ref{accuracy_SIR}, follows the behaviour of entropical detectability -- the accuracy grows while entropy gets lower. Additionally, accuracy does not differ significantly for the same value of parameter $p$ across different values of parameter $q \in \{0, 0.5, 1\}$. 

\begin{figure}[H]
\center
\includegraphics[width=\textwidth]{/home/iva/dipl/res/supfig12/accuracy_SIR.pdf}
\caption{Accuracy of source MAP estimation  on $4$-connected lattice $30 \times 30$ estimated with Soft Margin SIS method with $10^4 - 10^6$ samples and fixed $a=\frac{1}{2^5}$ under SIR  model with different parameters $p$ in range $0.1 - 0.9$,  $q = \{0, 0.5, 1\}$ and $T = 5$.}
\label{accuracy_SIR}
\end{figure}

While observing the distribution of samples for which estimations converge under convergence conditions \ref{DMCConv}, $c=0.05$ one can also detect the same three detectability regions, as presented in Figure \ref{simulations_SIR}. The region requiring the most samples is the intermediate entropy region while the high entropy and low entropy regions require minimal number of samples for estimations to converge. Across values of recovery parameter $q$, the estimations on intermediate entropy region  and median value of $q$ require the most samples to converge.

\begin{figure}[H]
\begin{minipage}{\textwidth}
\center
\includegraphics[width=\textwidth]{/home/iva/dipl/res/supfig12/simulations_SIR.pdf}
\caption{Box plots of converging samples distribution on  $4$-connected lattice $30 \times 30$ estimated with Soft Margin SIS method with $10^4 - 10^6$ samples and fixed $a=\frac{1}{2^5}$ under SIR  model with different parameters $p$ in range $0.1 - 0.9$,  $q = \{0, 0.5, 1\}$ and $T = 5$.}
\label{simulations_SIR}
\end{minipage}
\end{figure}

It is interesting to see how  detectability gets restricted in a regime where network topology restricts the epidemic spreading. By simulating epidemic on smaller lattices, the epidemic spreading gets restricted by the network size for smaller lattices and for higher value of infection parameter $p$, as presented in Figure \ref{epidemic_size}. 

\begin{figure}
\begin{minipage}{\textwidth}
\center
\includegraphics[width=\textwidth]{/home/iva/dipl/res/supfig12/epidemic_share_size.pdf}
\caption{Box plots of size of epidemics simulated on  $4$-connected lattices of different sizes under SIR model with different parameters $p$ in range $0.1 - 0.9$, fixed $q = 0.5$ and $T = 5$.}
\label{epidemic_size}
\end{minipage}
\end{figure}

For simulations in which the network size restricts epidemic spreading, the entropy is high as the realizations from different sources are almost identical. As the lattice grows and epidemic is less restricted by the network size, entropy distribution takes expected mean-decreasing form, as presented in Figure \ref{network_size}.

\begin{figure}
\center
\includegraphics[width=\textwidth]{/home/iva/dipl/res/supfig12/SIR_network_size.pdf}
\caption{Violin plots of estimated entropy distribution for source probability distributions  on  $4$-connected lattices of different sizes estimated with Soft Margin SIS method with $10^4 - 10^6$ samples and fixed $a=\frac{1}{2^5}$ under SIR model with different parameters $p$ in range $0.1 - 0.9$, fixed $q = 0.5$ and $T = 5$.}
\label{network_size}
\end{figure}

On the other hand, the accuracy is lower for epidemic-restricting network sizes and parameters, as presented in Figure \ref{epidemic_ac}. The biggest impact on accuracy reveals itself in the intermediate entropy region. The lower accuracy for the lower values of $p$ -- the ones  for which the epidemic is still not restricted by the lattice size -- can be explained by increasing symmetry of epidemic snapshots for smaller lattices, i.e. there tends to be serveral potential source nodes with the same MAP probability.

\begin{figure}
\begin{minipage}{\textwidth}
\center
\includegraphics[width=\textwidth]{/home/iva/dipl/res/supfig12/SIR_network_acc.pdf}
\caption{Accuracy of source MAP estimation  on  $4$-connected lattice of different sizes estimated with Soft Margin SIS method with $10^4 - 10^6$ samples and fixed $a=\frac{1}{2^5}$ under SIR model with different parameters $p$ in range $0.1 - 0.9$, fixed $q = 0.5$ and $T = 5$. }
\label{epidemic_ac}
\end{minipage}
\begin{minipage}{\textwidth}
\center
\includegraphics[width=\textwidth]{/home/iva/dipl/res/supfig12/SIR_network_sim.pdf}
\caption{Box plots of converging samples distribution on  $4$-connected lattice of different sizes estimated with Soft Margin SIS method with $10^4 - 10^6$ samples and fixed $a=\frac{1}{2^5}$ under SIR model with different parameters $p$ in range $0.1 - 0.9$, fixed $q = 0.5$ and $T = 5$.}
\label{epidemic_sim}
\end{minipage}
\end{figure}

\section{Detectability based on parameters of the ISS model}

To show how detectability presents itself in the parametric space of ISS parameters $a$ and $b$ several simulations were ran using Soft Margin detector with adaptive Soft Margin parameter $a$ chosen from $\{\frac{1}{2^3}, \frac{1}{2^4}, \ldots, \frac{1}{2^9} \}$.

For each set of ISS parameter pairs $(a, b)$  $50$ simulations of ISS rumour spreading were conducted for $T = 5$ time steps starting from the central node in $4-$connected lattice network of $30 \times 30$ nodes. All simulations of size $1$ were excluded from analysis. Soft Margin detector was ran on each rumour spreading simulation with converging conditions \ref{DMCConv}, $c=0.05$. Estimations were conducted based on  number of ISS simulations in range $n \in [10^4, 10^6]$. 

\begin{figure}[H]
\center
\includegraphics[width=\textwidth]{/home/iva/dipl/res/iss_grid/ISSBigGrid.pdf}
\caption{Violin plots of estimated entropy distribution for source probability distributions  on $4$-connected lattice $30 \times 30$ estimated with Soft Margin method with $10^4 - 10^6$ simulations and adaptive $a$ chosen from $\{1/2^3, 1/2^4, \ldots, 1/2^{9}\}$ under ISS model with different parameters $a$ in range $0.1-0.9, b = \{0, 0,5, 1 \}$ and $T = 5$.}
\label{ISS_entr}
\end{figure}

Entropical detectability for the rumour spreading ISS model presented in Figure \ref{ISS_entr}  shows behaviour similar to entropical detectability under epidemic SIR model presented in Figure \ref{entropy_zones}. In the model with no rumour decay where $b = 0$, ISS model is equivalent to SIR model, as are their entropy distributions. The $3$ entropy-detectability regions distinguishable in the entropy distribution over parameter $p$ for the SIR model can be distinguished in the entropy distribution over parameter $a$ for the ISS model. 

However, as the value of parameter $b$ in the ISS model compared to the same value of parameter $q$ in the SIR model represents more aggressive form of recovery, the low detectability - high entropy region grows more rapidly with higher value of $b$, taking over the range of parameters $a < 0.4$ and $a < 0.5$ for $b = 0.5$ and $b = 1.0$, respectively. 

On the other hand, accuracy of source detection for the ISS model remains stable for the same value of parameter $a$ and different values of parameter $b$ as presented in Figure \ref{ISS_acc}, similarly to the same accuracy for source detection under SIR model presented in Figure \ref{accuracy_SIR}. For the same value of parameter $b$, accuracy grows with higher value of parameter $a$.

\begin{figure}[H]
%\begin{minipage}{0.8\textwidth}
\center
\includegraphics[width=\textwidth]{/home/iva/dipl/res/iss_grid/ISS_accuracy.pdf}
\caption{Accuracy of source MAP estimation on $4$-connected lattice $30 \times 30$ estimated with Soft Margin method with $10^4 - 10^6$ simulations and adaptive $a$ chosen from $\{1/2^3, 1/2^4, \ldots, 1/2^{9}\}$ under ISS model with different parameters $a$ in range $0.1 - 0.9, b = \{0, 0.5, 1\}$ and $T = 5$. }
%\end{minipage}
\label{ISS_acc}
\end{figure}

\section{Detectability of patient zero based on its position in the network}

To analyse  how detectability changes across node features of the source node we will examine patient zero problem on two different network topologies: Erd{\H{o}}s-R{\'{e}}nyi and Barab\'{a}si-Albert graphs. These network topologies differ in distribution of centrality measures of its nodes so  epidemic spreads differently across each  network.

Two graph datasets were generated for this purpose. The Erd{\H{o}}s-R{\'{e}}nyi dataset consists of $50$ graphs with $N = 100$ nodes generated with $p=0.01$. This probability is the threshold of emergence of the giant component for a graph with $N = 100$ nodes  and the dataset contains only connected graphs. The second dataset consists of $50$ graphs with $N = 100$ nodes generated as Barab\'{a}si-Albert graphs with $m = 2$ attaching edges.

On the generated network impact of centrality measures of the source node  on source detectability will be analysed. Centrality measures include degree, closeness, betweenness, eigenvector centrality and coreness.

\subsection{Erd{\H{o}}s-R{\'{e}}nyi graph}

Before analysing how detectability changes for source nodes with different topological properties, let's examine the distribution of these properties on generated Erd{\H{o}}s-R{\'{e}}nyi $N = 100$, $p = 0.01$ graphs.

In Table \ref{sum_stat_erd} summary of statistics for each centrality measure is presented. The distribution of frequencies for each measure is presented in Figure \ref{erdos_corel}. Degree distribution takes the form of binomial distribution as expected. The centralities are positively correlated. 

In generated  Erd{\H{o}}s-R{\'{e}}nyi graphs, expected degree and betweenness of a node are relatively small compared to the network size. 

\begin{table}[H]
\begin{tabular}{|c|c|c|c|c|c|}
\hline 
 & \textbf{Degree} & \textbf{Closeness} & \textbf{Betweenness} & \textbf{Eigenvector centrality} & \textbf{Coreness} \\ 
\hline 
\textbf{Min} & $1$ & $0.1713$ & $0.00$ & $0.001575$ & $1$ \\ 
\hline 
\textbf{Median} & $5$ & $0.3246$ & $83.33$ & $0.334041$ & $3$ \\ 
\hline 
\textbf{Mean} & $4.657$ & $0.3222$ & $106.14$ & $0.367168$ & $2.831$ \\ 
\hline 
\textbf{Max} & $14$ & $0.4304$ & $698.57$ & $1.000000$ & $4$ \\ 
\hline 
\end{tabular} 
\caption{Summary of cumulative statistics of distributions for degree, closeness, betweenness, eigenvector centrality and coreness of the nodes in $50$ generated Erd{\H{o}}s-R{\'{e}}nyi connected graphs with $p=0.01$ and $N=100$ nodes. }
\label{sum_stat_erd}
\end{table}

%\begin{landscape}
\begin{figure}[H]
\begin{minipage}{0.5\textwidth}
\center
\includegraphics[width=\textwidth]
{/home/iva/dipl/res/erdos/erdos_degree_distr.pdf}
\end{minipage}
\begin{minipage}{0.5\textwidth}
\center
\includegraphics[width=\textwidth]
{/home/iva/dipl/res/erdos/erdos_closeness_distr.pdf}
\end{minipage}

\begin{minipage}{0.5\textwidth}
\center
\includegraphics[width=\textwidth]
{/home/iva/dipl/res/erdos/erdos_betw_distr.pdf}
\end{minipage}
\begin{minipage}{0.5\textwidth}
\center
\includegraphics[width=\textwidth]
{/home/iva/dipl/res/erdos/erdos_eig_distr.pdf}
\end{minipage}

\begin{minipage}{0.5\textwidth}
\center
\includegraphics[width=\textwidth]
{/home/iva/dipl/res/erdos/erdos_coreness_distr.pdf}
\end{minipage}
\caption{Frequencies of degree, closeness, betweenness, eigenvector and coreness centralities on $50$ generated Erd{\H{o}}s-R{\'{e}}nyi graphs with $N=100$ and $p= 0.01$.}
\label{erdos_corel}
\end{figure}
%\end{landscape}

\subsubsection{Degree centrality}

The epidemics simulated on Erd{\H{o}}s-R{\'{e}}nyi graphs infect more nodes when started from a node with higher degree as presented in Figure \ref{erdos_deg_cov}. Epidemics simulated with high infection rate $p = 0.7$  in classes of SIR parameters $C$ and $D$ were highly restricted by the network size and  we can already predict that will have impact on detectability. As expected, high recovery rate $q = 0.7$ in class $B$ limits the size of epidemic compared to epidemics with SIR parameters in class $A$ and low recovery rate $q=0.3$.


\begin{figure}[H]
\begin{minipage}{\textwidth}
\begin{minipage}{0.47\textwidth}
\center
\includegraphics[width=\textwidth]{/home/iva/dipl/res/erdos/epidemic_cov_deg_A.pdf}
\end{minipage}
\begin{minipage}{0.47\textwidth}
\center
\includegraphics[width=\textwidth]{/home/iva/dipl/res/erdos/epidemic_cov_deg_B.pdf}
\end{minipage}

\begin{minipage}{0.47\textwidth}
\center
\includegraphics[width=\textwidth]{/home/iva/dipl/res/erdos/epidemic_cov_deg_C.pdf}
\end{minipage}
\begin{minipage}{0.47\textwidth}
\center
\includegraphics[width=\textwidth]{/home/iva/dipl/res/erdos/epidemic_cov_deg_D.pdf}
\end{minipage}
\caption{Box plots of size of epidemics simulated on on  Erd{\H{o}}s-R{\'{e}}nyi connected graphs with $p=0.01$ and $N=100$ nodes under SIR model with parameters $A = (p=0.3, q=0.3, T=5), B = (p=0.3, q=0.7, T=5), C = (p=0.7, q=0.3, T=5), D = (p=0.7, q=0.7, T=5)$ grouped by degree of the source node.}
\label{erdos_deg_cov}
\end{minipage}
\end{figure}

The impact of restricting network size  for parameter classes $C$ and $D$ is presented in entropy distributions with high expectation as presented in Figure \ref{deg_ent}. For classes of parameters $A$ and $B$, the entropy is also high and difference in degree of the source node does not significantly alter its expected value. Additionally, the epidemics simulated with higher recovery rate (classes $B$ and $D$) exibit higher expected entropy compared to their low recovery counterparts (classes $A$ and $C$, respectively). 

The source detection accuracy is lower for source nodes that have higher degree, as presented Figure \ref{deg_acc}.  The degree of the source node is more restrictive for accuracy on epidemics simulated from classes of parameters $C$ and $D$. 

\begin{figure}[H]
\begin{minipage}{\textwidth}
\begin{minipage}{0.5\textwidth}
\includegraphics[width=\textwidth]{/home/iva/dipl/res/erdos/deg_ent_A.pdf}
\end{minipage}
\begin{minipage}{0.5\textwidth}
\includegraphics[width=\textwidth]{/home/iva/dipl/res/erdos/deg_ent_B.pdf}
\end{minipage}
\begin{minipage}{0.5\textwidth}
\includegraphics[width=\textwidth]{/home/iva/dipl/res/erdos/deg_ent_C.pdf}
\end{minipage}
\begin{minipage}{0.5\textwidth}
\includegraphics[width=\textwidth]{/home/iva/dipl/res/erdos/deg_ent_D.pdf}
\end{minipage}
\caption{Violin plots of estimated entropy distribution for estimated source probability distribution on  Erd{\H{o}}s-R{\'{e}}nyi connected graphs with $p=0.01$ and $N=100$ nodes estimated with Soft Margin SIS detector with $10^4 - 10^6$ simulations and fixed $a = \frac{1}{2^5}$ under SIR model with parameters $A = (p=0.3, q=0.3, T=5), B = (p=0.3, q=0.7, T=5), C = (p=0.7, q=0.3, T=5), D = (p=0.7, q=0.7, T=5)$ grouped by degree of the source node.}
\label{deg_ent}
%\end{figure}
\end{minipage}
%\begin{figure}[H]
\begin{minipage}{\textwidth}
\center
\includegraphics[width=0.77\textwidth]{/home/iva/dipl/res/erdos/accuracy_deg.pdf}
\caption{Source detection accuracy on  Erd{\H{o}}s-R{\'{e}}nyi connected graphs with $p=0.01$ and $N=100$ nodes estimated with Soft Margin SIS detector with $10^4 - 10^6$ simulations and fixed $a = \frac{1}{2^5}$ under SIR model with parameters $A = (p=0.3, q=0.3, T=5), B = (p=0.3, q=0.7, T=5), C = (p=0.7, q=0.3, T=5), D = (p=0.7, q=0.7, T=5)$ grouped by degree of the source node.}
\label{deg_acc}
\end{minipage}
\end{figure}

The converging number of samples required for estimations to converge on  Erd{\H{o}}s-R{\'{e}}nyi graphs is on average in range  $n \in [4 \cdot  10^4, 4 \cdot 10^5]$.  For classes of SIR parameters $A$ and $B$ the expected number of simulations does not  differ significantly  for different degree of the source node, as presented in Figure \ref{erdos_sim_deg}. For classes of SIR parameters $C$ and $D$, required number of samples is in range $n \in [10^4, 10^5 ] $.  For largest values of degree of the source node, the required number of converging samples for classes of parameters $C$ and $D$ is minimal. 

\begin{figure}[H]
\begin{minipage}{0.5\textwidth}
\includegraphics[width=\textwidth]{/home/iva/dipl/res/erdos/deg_sim_A.pdf}
\end{minipage}
\begin{minipage}{0.5\textwidth}
\includegraphics[width=\textwidth]{/home/iva/dipl/res/erdos/deg_sim_B.pdf}
\end{minipage}
\begin{minipage}{0.5\textwidth}
\includegraphics[width=\textwidth]{/home/iva/dipl/res/erdos/deg_sim_C.pdf}
\end{minipage}
\begin{minipage}{0.5\textwidth}
\includegraphics[width=\textwidth]{/home/iva/dipl/res/erdos/deg_sim_D.pdf}
\end{minipage}
\caption{Box plots of converging samples distribution for source MAP estimations on  Erd{\H{o}}s-R{\'{e}}nyi connected graphs with $p=0.01$ and $N=100$ nodes estimated with Soft Margin SIS detector with $10^4 - 10^6$ simulations and fixed $a = \frac{1}{2^5}$ under SIR model with parameters $A = (p=0.3, q=0.3, T=5), B = (p=0.3, q=0.7, T=5), C = (p=0.7, q=0.3, T=5), D = (p=0.7, q=0.7, T=5)$ grouped by degree of the source node.}
\label{erdos_sim_deg} 
\end{figure}

\subsubsection{Closeness centrality}

Closeness centrality corresponds to how close a given node is to any other node, as defined in \ref{clos}. For  Erd{\H{o}}s-R{\'{e}}nyi graph, the nodes with higher closeness usually have a higher degree compared to the nodes of lower closeness so the results for degree and closeness will be similar.

Size of simulated epidemics grows with higher closeness, as presented in Figure \ref{cov_clos}. 

\begin{figure}[H]
\begin{minipage}{\textwidth}
\begin{minipage}{0.50\textwidth}
\includegraphics[width=\textwidth]{/home/iva/dipl/res/erdos/epidemic_cov_clos_A.pdf}
\end{minipage}
\begin{minipage}{0.50\textwidth}
\includegraphics[width=\textwidth]{/home/iva/dipl/res/erdos/epidemic_cov_clos_B.pdf}
\end{minipage}

\begin{minipage}{0.50\textwidth}
\includegraphics[width=\textwidth]{/home/iva/dipl/res/erdos/epidemic_cov_clos_C.pdf}
\end{minipage}
\begin{minipage}{0.50\textwidth}
\includegraphics[width=\textwidth]{/home/iva/dipl/res/erdos/epidemic_cov_clos_D.pdf}
\end{minipage}
\caption{Box plots of size of epidemics simulated on on  Erd{\H{o}}s-R{\'{e}}nyi connected graphs with $p=0.01$ and $N=100$ nodes under SIR model with parameters $A = (p=0.3, q=0.3, T=5), B = (p=0.3, q=0.7, T=5), C = (p=0.7, q=0.3, T=5), D = (p=0.7, q=0.7, T=5)$ grouped by closeness of the source node.}
\label{cov_clos}
%\end{figure}
\end{minipage}
\end{figure}

Entropy distribution over different values of closeness of the source node stays constant and high in expectation for classes of SIR parameters $A$ and $B$. For classes of SIR parameters $C$ i $D$ that correspond to high detectability - low entropy region for detection on the grid, entropy is low for  low values of closeness and grows higher when closeness is high, as presented in Figure \ref{clos_ent}. 

Accuracy is also high for lower values of closeness, as presented in Figure \ref{accuracy_clos}. Note the closeness in range $[0.15, 0.3)$ mostly corresponds to nodes with degree  equal to $1$ and $2$ (Figure \ref{erdos_corel}) on which we've already seen high accuracy of source detection (Figure \ref{deg_acc}).  Accuracy is more restricted by closeness for classes of SIR parameters $C$ and $D$. 

\begin{figure}[H]
\begin{minipage}{\textwidth}
\begin{minipage}{0.5\textwidth}
\includegraphics[width=\textwidth]{/home/iva/dipl/res/erdos/clos_ent_A.pdf}
\end{minipage}
\begin{minipage}{0.5\textwidth}
\includegraphics[width=\textwidth]{/home/iva/dipl/res/erdos/clos_ent_B.pdf}
\end{minipage}
\begin{minipage}{0.5\textwidth}
\includegraphics[width=\textwidth]{/home/iva/dipl/res/erdos/clos_ent_C.pdf}
\end{minipage}
\begin{minipage}{0.5\textwidth}
\includegraphics[width=\textwidth]{/home/iva/dipl/res/erdos/clos_ent_D.pdf}
\end{minipage}
\caption{Violin plots of estimated entropy distribution for estimated source probability distribution on  Erd{\H{o}}s-R{\'{e}}nyi connected graphs with $p=0.01$ and $N=100$ nodes estimated with Soft Margin SIS detector with $10^4 - 10^6$ simulations and fixed $a = \frac{1}{2^5}$ under SIR model with parameters $A = (p=0.3, q=0.3, T=5), B = (p=0.3, q=0.7, T=5), C = (p=0.7, q=0.3, T=5), D = (p=0.7, q=0.7, T=5)$ grouped by closeness of the source node.}
\label{clos_ent}
%\end{figure}
\end{minipage}
%\begin{figure}[H]
\begin{minipage}{\textwidth}
%\begin{figure}[H]
\center
\includegraphics[width=0.70\textwidth]{/home/iva/dipl/res/erdos/accuracy_clos.pdf}
\caption{Source detection accuracy on  Erd{\H{o}}s-R{\'{e}}nyi connected graphs with $p=0.01$ and $N=100$ nodes estimated with Soft Margin SIS detector with $10^4 - 10^6$ simulations and fixed $a = \frac{1}{2^5}$ under SIR model with parameters $A = (p=0.3, q=0.3, T=5), B = (p=0.3, q=0.7, T=5), C = (p=0.7, q=0.3, T=5), D = (p=0.7, q=0.7, T=5)$ grouped by closeness of the source node.}
\label{accuracy_clos}
\end{minipage}
\end{figure}


\begin{figure}[H]
\begin{minipage}{0.5\textwidth}
\includegraphics[width=\textwidth]{/home/iva/dipl/res/erdos/clos_sim_A.pdf}
\end{minipage}
\begin{minipage}{0.5\textwidth}
\includegraphics[width=\textwidth]{/home/iva/dipl/res/erdos/clos_sim_B.pdf}
\end{minipage}
\begin{minipage}{0.5\textwidth}
\includegraphics[width=\textwidth]{/home/iva/dipl/res/erdos/clos_sim_C.pdf}
\end{minipage}
\begin{minipage}{0.5\textwidth}
\includegraphics[width=\textwidth]{/home/iva/dipl/res/erdos/clos_sim_D.pdf}
\end{minipage}
\caption{Box plots of converging samples distribution for source MAP estimations on  Erd{\H{o}}s-R{\'{e}}nyi connected graphs with $p=0.01$ and $N=100$ nodes estimated with Soft Margin SIS detector with $10^4 - 10^6$ simulations and fixed $a = \frac{1}{2^5}$ under SIR model with parameters $A = (p=0.3, q=0.3, T=5), B = (p=0.3, q=0.7, T=5), C = (p=0.7, q=0.3, T=5), D = (p=0.7, q=0.7, T=5)$ grouped by closeness of the source node.}
\end{figure}

\subsubsection{Betweenness centrality}

Betweenness centrality describes how well situated a vertex is in terms of the paths it lies on, as defined in \ref{betw}. Betweenness positively correlates with degree and closeness  for Erd{\H{o}}s-R{\'{e}}nyi graph, so detectability  results grouped by betweenness of the source node are similar to the ones for degree and closeness. Entropy distribution, accuracy, and distribution of converging number of samples grouped by betweenness of the source node are presented in Figure \ref{betw_ent}, \ref{accuracy_betw} and \ref{betw_sim}, respectively.

Detection accuracy is lower for source nodes with higher betweenness and it gets more restrictive for epidemics based on SIR parameters in classes $C$ and $D$.

\begin{figure}[H]
\begin{minipage}{\textwidth}
\begin{minipage}{0.48\textwidth}
\includegraphics[width=\textwidth]{/home/iva/dipl/res/erdos/betw_ent_A.pdf}
\end{minipage}
\begin{minipage}{0.48\textwidth}
\includegraphics[width=\textwidth]{/home/iva/dipl/res/erdos/betw_ent_B.pdf}
\end{minipage}

\begin{minipage}{0.48\textwidth}
\includegraphics[width=\textwidth]{/home/iva/dipl/res/erdos/betw_ent_C.pdf}
\end{minipage}
\begin{minipage}{0.48\textwidth}
\includegraphics[width=\textwidth]{/home/iva/dipl/res/erdos/betw_ent_D.pdf}
\end{minipage}
\caption{Violin plots of estimated entropy distribution for estimated source probability distribution on  Erd{\H{o}}s-R{\'{e}}nyi connected graphs with $p=0.01$ and $N=100$ nodes estimated with Soft Margin SIS detector with $10^4 - 10^6$ simulations and fixed $a = \frac{1}{2^5}$ under SIR model with parameters $A = (p=0.3, q=0.3, T=5), B = (p=0.3, q=0.7, T=5), C = (p=0.7, q=0.3, T=5), D = (p=0.7, q=0.7, T=5)$ grouped by betwenness of the source node.}
\label{betw_ent}
\end{minipage}
%\end{figure}

%\begin{figure}[H]
\begin{minipage}{\textwidth}
\center
\includegraphics[width=0.75\textwidth]{/home/iva/dipl/res/erdos/accuracy_betw.pdf}
\caption{Source detection accuracy on  Erd{\H{o}}s-R{\'{e}}nyi connected graphs with $p=0.01$ and $N=
100$ nodes estimated with Soft Margin SIS detector with $10^4 - 10^6$ simulations and fixed $a = \frac{1}{2^5}$ under SIR model with parameters $A = (p=0.3, q=0.3, T=5), B = (p=0.3, q=0.7, T=5), C = (p=0.7, q=0.3, T=5), D = (p=0.7, q=0.7, T=5)$ grouped by betweenness of the source node.}.
\label{accuracy_betw}
\end{minipage}
\end{figure}

\begin{figure}[H]
\begin{minipage}{\textwidth}
\begin{minipage}{0.49\textwidth}
\includegraphics[width=\textwidth]{/home/iva/dipl/res/erdos/betw_sim_A.pdf}
\end{minipage}
\begin{minipage}{0.49\textwidth}
\includegraphics[width=\textwidth]{/home/iva/dipl/res/erdos/betw_sim_B.pdf}
\end{minipage}
\begin{minipage}{0.49\textwidth}
\includegraphics[width=\textwidth]{/home/iva/dipl/res/erdos/betw_sim_C.pdf}
\end{minipage}
\begin{minipage}{0.49\textwidth}
\includegraphics[width=\textwidth]{/home/iva/dipl/res/erdos/betw_sim_D.pdf}
\end{minipage}
\caption{Box plots of converging samples distribution for source MAP estimations on  Erd{\H{o}}s-R{\'{e}}nyi connected graphs with $p=0.01$ and $N=100$ nodes estimated with Soft Margin SIS detector with $10^4 - 10^6$ simulations and fixed $a = \frac{1}{2^5}$ under SIR model with parameters $A = (p=0.3, q=0.3, T=5), B = (p=0.3, q=0.7, T=5), C = (p=0.7, q=0.3, T=5), D = (p=0.7, q=0.7, T=5)$ grouped by betweenness of the source node.}
\label{betw_sim}
\end{minipage}
\end{figure}

\subsubsection{Eigenvector centrality}

Eigenvector centrality describes centrality of the source node in terms of node centrality of its neighbours, as defined in \ref{eig}. Eigenvector centrality positively correlates with degree, closeness and betweenness for  Erd{\H{o}}s-R{\'{e}}nyi graph as presented in Figure \ref{erdos_corel}, so detectability  results grouped by eigenvector centrality of the source node are similar to the ones for degree, closeness and betweenness. Entropy distribution, accuracy, and distribution of converging number of samples grouped by eigenvector centrality of the source node are presented in Figure \ref{eig_ent}, \ref{accuracy_ent} and \ref{erdos_eig_sim}, respectively.

\begin{figure}[H]
\begin{minipage}{\textwidth}
\begin{minipage}{0.49\textwidth}
\includegraphics[width=\textwidth]{/home/iva/dipl/res/erdos/ev_ent_A.pdf}
\end{minipage}
\begin{minipage}{0.49\textwidth}
\includegraphics[width=\textwidth]{/home/iva/dipl/res/erdos/ev_ent_B.pdf}
\end{minipage}
\begin{minipage}{0.49\textwidth}
\includegraphics[width=\textwidth]{/home/iva/dipl/res/erdos/ev_ent_C.pdf}
\end{minipage}
\begin{minipage}{0.49\textwidth}
\includegraphics[width=\textwidth]{/home/iva/dipl/res/erdos/ev_ent_D.pdf}
\end{minipage}
\caption{Violin plots of estimated entropy distribution for estimated source probability distribution on  Erd{\H{o}}s-R{\'{e}}nyi connected graphs with $p=0.01$ and $N=100$ nodes estimated with Soft Margin SIS detector with $10^4 - 10^6$ simulations and fixed $a = \frac{1}{2^5}$ under SIR model with parameters $A = (p=0.3, q=0.3, T=5), B = (p=0.3, q=0.7, T=5), C = (p=0.7, q=0.3, T=5), D = (p=0.7, q=0.7, T=5)$ grouped by eigenvector centrality of the source node.}
\label{eig_ent}
\end{minipage}
%\end{figure}

%\begin{figure}[H]
\begin{minipage}{\textwidth}
\center
\includegraphics[width=0.74\textwidth]{/home/iva/dipl/res/erdos/accuracy_eig.pdf}
\caption{Source detection accuracy on  Erd{\H{o}}s-R{\'{e}}nyi connected graphs with $p=0.01$ and $N=100$ nodes estimated with Soft Margin SIS detector with $10^4 - 10^6$ simulations and fixed $a = \frac{1}{2^5}$ under SIR model with parameters $A = (p=0.3, q=0.3, T=5), B = (p=0.3, q=0.7, T=5), C = (p=0.7, q=0.3, T=5), D = (p=0.7, q=0.7, T=5)$ grouped by eigenvector centrality of the source node.}
\label{accuracy_ent}
\end{minipage}
\end{figure}
%
%
%
\begin{figure}[H]
\begin{minipage}{\textwidth}
\begin{minipage}{0.5\textwidth}
\includegraphics[width=\textwidth]{/home/iva/dipl/res/erdos/eig_sim_A.pdf}
\end{minipage}
\begin{minipage}{0.5\textwidth}
\includegraphics[width=\textwidth]{/home/iva/dipl/res/erdos/eig_sim_B.pdf}
\end{minipage}
\begin{minipage}{0.5\textwidth}
\includegraphics[width=\textwidth]{/home/iva/dipl/res/erdos/eig_sim_C.pdf}
\end{minipage}
\begin{minipage}{0.5\textwidth}
\includegraphics[width=\textwidth]{/home/iva/dipl/res/erdos/eig_sim_D.pdf}
\end{minipage}
\caption{Box plots of converging samples distribution for source MAP estimations on  Erd{\H{o}}s-R{\'{e}}nyi connected graphs with $p=0.01$ and $N=100$ nodes estimated with Soft Margin SIS detector with $10^4 - 10^6$ simulations and fixed $a = \frac{1}{2^5}$ under SIR model with parameters $A = (p=0.3, q=0.3, T=5), B = (p=0.3, q=0.7, T=5), C = (p=0.7, q=0.3, T=5), D = (p=0.7, q=0.7, T=5)$ grouped by eigenvector centrality of the source node.}
\label{erdos_eig_sim}
\end{minipage}
\end{figure}

\subsubsection{Coreness}
Coreness of the source node also positively correlates with degree, closenes, betweenness and eigenvector centrality for  Erd{\H{o}}s-R{\'{e}}nyi graph as presented in Figure \ref{erdos_corel}, so detectability  results grouped by coreness of the source node are similar to the ones for other measures. Entropy distribution, accuracy, and distribution of converging number of samples grouped by eigenvector centrality of the source node are presented in Figure \ref{core_ent}, \ref{accuracy_core} and \ref{core_sim}, respectively.

\begin{figure}[H]
\begin{minipage}{\textwidth}
\begin{minipage}{0.5\textwidth}
\includegraphics[width=\textwidth]{/home/iva/dipl/res/erdos/core_ent_A.pdf}
\end{minipage}
\begin{minipage}{0.5\textwidth}
\includegraphics[width=\textwidth]{/home/iva/dipl/res/erdos/core_ent_B.pdf}
\end{minipage}
\begin{minipage}{0.5\textwidth}
\includegraphics[width=\textwidth]{/home/iva/dipl/res/erdos/core_ent_C.pdf}
\end{minipage}
\begin{minipage}{0.5\textwidth}
\includegraphics[width=\textwidth]{/home/iva/dipl/res/erdos/core_ent_D.pdf}
\end{minipage}
\caption{Violin plots of estimated entropy distribution for estimated source probability distribution on  Erd{\H{o}}s-R{\'{e}}nyi connected graphs with $p=0.01$ and $N=100$ nodes estimated with Soft Margin SIS detector with $10^4 - 10^6$ simulations and fixed $a = \frac{1}{2^5}$ under SIR model with parameters $A = (p=0.3, q=0.3, T=5), B = (p=0.3, q=0.7, T=5), C = (p=0.7, q=0.3, T=5), D = (p=0.7, q=0.7, T=5)$ grouped by coreness of the source node.}
\label{core_ent}
\end{minipage}
%\end{figure}
%\begin{figure}[H]
\begin{minipage}{\textwidth}
\center
\includegraphics[width=0.75\textwidth]{/home/iva/dipl/res/erdos/accuracy_core.pdf}
\caption{Source detection accuracy on  Erd{\H{o}}s-R{\'{e}}nyi connected graphs with $p=0.01$ and $N=100$ nodes estimated with Soft Margin SIS detector with $10^4 - 10^6$ simulations and fixed $a = \frac{1}{2^5}$ under SIR model with parameters $A = (p=0.3, q=0.3, T=5), B = (p=0.3, q=0.7, T=5), C = (p=0.7, q=0.3, T=5), D = (p=0.7, q=0.7, T=5)$ grouped by coreness of the source node.}
\label{accuracy_core}
\end{minipage}
\end{figure}
%
\begin{figure}[H]
\begin{minipage}{\textwidth}
\begin{minipage}{0.5\textwidth}
\includegraphics[width=\textwidth]{/home/iva/dipl/res/erdos/core_sim_A.pdf}
\end{minipage}
\begin{minipage}{0.5\textwidth}
\includegraphics[width=\textwidth]{/home/iva/dipl/res/erdos/core_sim_B.pdf}
\end{minipage}
\begin{minipage}{0.5\textwidth}
\includegraphics[width=\textwidth]{/home/iva/dipl/res/erdos/core_sim_C.pdf}
\end{minipage}
\begin{minipage}{0.5\textwidth}
\includegraphics[width=\textwidth]{/home/iva/dipl/res/erdos/core_sim_D.pdf}
\end{minipage}
\caption{Box plots of converging samples distribution for source MAP estimations on  Erd{\H{o}}s-R{\'{e}}nyi connected graphs with $p=0.01$ and $N=100$ nodes estimated with Soft Margin SIS detector with $10^4 - 10^6$ simulations and fixed $a = \frac{1}{2^5}$ under SIR model with parameters $A = (p=0.3, q=0.3, T=5), B = (p=0.3, q=0.7, T=5), C = (p=0.7, q=0.3, T=5), D = (p=0.7, q=0.7, T=5)$ grouped by coreness of the source node.}
\label{core_sim}
\end{minipage}
\end{figure}

\subsection{Barab\'{a}si-Albert graph}

Barab\'{a}si-Albert graph is generated using preferential attachment property. We will use a set of  Barab\'{a}si-Albert graphs wtih $n = 100$ nodes that were evolved with  $m = 2$ attaching edges for each added node. For these graphs coreness is constant and equal to $m$ so it will be  omitted from  analysis. 

The summary of statistics for centrality measures on generated $2-$ Barab\'{a}si-Albert graphs is presented in Table \ref{table_bara}. 
Apart from constant coreness and compared to distribution of the same measures on  Erd{\H{o}}s-R{\'{e}}nyi graphs of the same size, degree, closeness and betweenness can take higher values.

\begin{table}[H]
\begin{tabular}{|c|c|c|c|c|c|}
\hline 
 & \textbf{Degree} & {\textbf{Closeness}} & \textbf{Betweenness} & \textbf{Eigenvector centrality} & \textbf{Coreness} \\ 
\hline 
\textbf{Min} & $2$ & $0.2421$ & $0$ & $0.003098$ & $2$ \\ 
\hline 
\textbf{Median} & $2$ & $0.3808$ & $5.924$ & $0.102812$ & $2$ \\ 
\hline 
\textbf{Mean} & $3.96$ & $0.3823$ & $82.952$ & $0.116188$ & $2$ \\ 
\hline 
\textbf{Max} & $108$ & $0.8250$ & $4517.507$ & $1.000000$ & $2$ \\ 
\hline 
\end{tabular} 
\caption{Summary of cummulative statistics of distributions for degree, closeness, betweenness, eigenvector centrality and coreness of the nodes in $50$ generated Barab\'{a}si-Albert graphs with $m=2$ and $N=100$ nodes.}
\label{table_bara}
\end{table}

\begin{figure}[h]
\begin{minipage}{0.5\textwidth}
\center
\includegraphics[width=\textwidth]
{/home/iva/dipl/res/bara/deg_b2.pdf}
\end{minipage}
\begin{minipage}{0.5\textwidth}
\center
\includegraphics[width=\textwidth]
{/home/iva/dipl/res/bara/clos_b2.pdf}
\end{minipage}
\begin{minipage}{0.5\textwidth}
\includegraphics[width=\textwidth]
{/home/iva/dipl/res/bara/betw_b2.pdf}
\end{minipage}
\begin{minipage}{0.5\textwidth}
\includegraphics[width=\textwidth]{/home/iva/dipl/res/bara/eig_b2.pdf}
\end{minipage}
\caption{Distribution of degree, closeness, betweenness and eigenvector centrality in  $2$- Barab\'{a}si-Albert dataset.}
\label{bara_distr}
\end{figure}

Additionally, degree distribution is scale-free, as expected, while  distributions of other centrality measures take similar form as presented in Figure \ref{bara_distr}. Ranges of values for each centrality measure upon which the detectability is analysed has been chosen according to these distributions.  Between each pair of centrality measures the correlation is mostly positive. 

\subsubsection{Degree centrality}

Epidemic simulations on  $2$- Barab\'{a}si-Albert graphs are restricted by the network size and are expected to infect all nodes in the network for classes of SIR parameters $A$, $C$, $D$ as presented in Figure \ref{bara_cov}. The structure of $2$- Barab\'{a}si-Albert graphs, precisely their short average path length, help in epidemic spreading. 

 For SIR parameters $B = (p = 0.3, q = 0.3)$ expected size of epidemic positively correlates with degree of the source node. The epidemics simulated with SIR parameters with lower recovery rate  (classes $A$ and $C$) have higher expected epidemic size  compared to the corresponding classes with higher recovery rate (classes $B$ and $D$, respectively).
  
\begin{figure}[H]
\begin{minipage}{0.47\textwidth}
\center
\includegraphics[width=\textwidth]{/home/iva/dipl/res/bara/epidemic_coverage_deg_A.pdf}
\end{minipage}
\begin{minipage}{0.47\textwidth}
\center
\includegraphics[width=\textwidth]{/home/iva/dipl/res/bara/epidemic_coverage_deg_B.pdf}
\end{minipage}

\begin{minipage}{0.47\textwidth}
\center
\includegraphics[width=\textwidth]{/home/iva/dipl/res/bara/epidemic_coverage_deg_C.pdf}
\end{minipage}
\begin{minipage}{0.47\textwidth}
\center
\includegraphics[width=\textwidth]{/home/iva/dipl/res/bara/epidemic_coverage_deg_D.pdf}
\end{minipage}
\caption{Box plots of size of epidemics simulated on  $2$-Barab\'{a}si-Albert graphs  with $N=100$ nodes under SIR model with parameters $A = (p=0.3, q=0.3, T=5), B = (p=0.3, q=0.7, T=5), C = (p=0.7, q=0.3, T=5), D = (p=0.7, q=0.7, T=5)$ grouped by degree of the source node.}
\label{bara_cov}
\end{figure}

The entropy distributions,  on  $2$-Barab\'{a}si-Albert graphs take high values for all SIR parameter classes, no matter the degree of the source node, as presented in Figure \ref{bara_deg_ent}.

Moreover, for $2$-Barab\'{a}si-Albert graphs detection accuracy is generally low. Except for network size restricting epidemics with parameter class $C$, the accuracy is highest for detection for network non-restricting epidemics with SIR parameter class $B$ and low degree source nodes, as presented in Figure \ref{bara_acc_deg}. 

The emergence of detectability for high degrees of the source node for source detection problems with SIR parameters in class $C$ gives a hint the local network structure plays a role in detectability since usually the high degree nodes are the ones at the core of these locally connected groups. This property isn't noticeable in Erd{\H{o}}s-R{\'{e}}nyi graphs since they do note have with such local  structure. 

\begin{figure}[H]
\begin{minipage}{\textwidth}
\center
\includegraphics[width=0.7\textwidth]{/home/iva/dipl/res/bara/accuracy_deg.pdf}
\caption{Source detection accuracy on  $2$-Barab\'{a}si-Albert graphs  with $N=100$ nodes estimated with Soft Margin SIS detector with $10^4 - 10^6$ simulations and fixed $a = \frac{1}{2^5}$ under SIR model with parameters $A = (p=0.3, q=0.3, T=5), B = (p=0.3, q=0.7, T=5), C = (p=0.7, q=0.3, T=5), D = (p=0.7, q=0.7, T=5)$ grouped by degree of the source node.}
\label{bara_acc_deg}
%\end{figure}
\end{minipage}
\end{figure}

\begin{figure}[H]
%\begin{minipage}{\textwidth}
%\begin{figure}[H]
\begin{minipage}{0.49\textwidth}
\includegraphics[width=\textwidth]{/home/iva/dipl/res/bara/deg_ent_A.pdf}
\end{minipage}
\begin{minipage}{0.49\textwidth}
\includegraphics[width=\textwidth]{/home/iva/dipl/res/bara/deg_ent_B.pdf}
\end{minipage}

\begin{minipage}{0.49\textwidth}
\includegraphics[width=\textwidth]{/home/iva/dipl/res/bara/deg_ent_C.pdf}
\end{minipage}
\begin{minipage}{0.49\textwidth}
\includegraphics[width=\textwidth]{/home/iva/dipl/res/bara/deg_ent_D.pdf}
\end{minipage}
\caption{Violin plots of estimated entropy distribution for estimated source probability distribution on  $2$-Barab\'{a}si-Albert graphs  with $N=100$ nodes estimated with Soft Margin SIS detector with $10^4 - 10^6$ simulations and fixed $a = \frac{1}{2^5}$ under SIR model with parameters $A = (p=0.3, q=0.3, T=5), B = (p=0.3, q=0.7, T=5), C = (p=0.7, q=0.3, T=5), D = (p=0.7, q=0.7, T=5)$ grouped by degree of the source node.}
\label{bara_deg_ent}
%\end{minipage}
\end{figure}

For $2$-Barab\'{a}si-Albert graphs converging number of samples negatively correlates with degree of the source node, as presented for classes of parameters $A$ and $D$ in Figure \ref{bara_sim_deg}. Additionally, epidemics with higher recovery rate require more samples to converge compared to corresponding low recovery rate class pairs. 

The expected size of required samples is in range $[4 \cdot 10^4, 10^6]$ for classes of SIR parameters $A$ and $B$.  For SIR parameter classes $C$ and $D$, the converging number of samples is usually lower, as presented in Figure \ref{bara_sim_deg}. It is worth mentioning the number of  samples was upper bounded  by $10^6$ during detection process and it is expected even more samples are actually required to have higher detection accuracy for $2$-Barab\'{a}si-Albert graphs with classes of SIR parameters $A$ and $B$.

\begin{figure}[H]
\begin{minipage}{0.5\textwidth}
\includegraphics[width=\textwidth]{/home/iva/dipl/res/bara/deg_sim_A.pdf}
\end{minipage}
\begin{minipage}{0.5\textwidth}
\includegraphics[width=\textwidth]{/home/iva/dipl/res/bara/deg_sim_B.pdf}
\end{minipage}
\begin{minipage}{0.5\textwidth}
\includegraphics[width=\textwidth]{/home/iva/dipl/res/bara/deg_sim_C.pdf}
\end{minipage}
\begin{minipage}{0.5\textwidth}
\includegraphics[width=\textwidth]{/home/iva/dipl/res/bara/deg_sim_D.pdf}
\end{minipage}
\caption{Box plots of converging samples distribution for source MAP estimations on  $2$-Barab\'{a}si-Albert graphs  with $N=100$ nodes estimated with Soft Margin SIS detector with $10^4 - 10^6$ simulations and fixed $a = \frac{1}{2^5}$ under SIR model with parameters $A = (p=0.3, q=0.3, T=5), B = (p=0.3, q=0.7, T=5), C = (p=0.7, q=0.3, T=5), D = (p=0.7, q=0.7, T=5)$ grouped by degree of the source node.}
\label{bara_sim_deg} 
\end{figure}

\subsubsection{Closeness centrality}

Closeness corresponds to how close a given node is to any other node in the network, as defined in \ref{clos}. 

For classes of SIR parameters $A$ and $B$, expected epidemic size positively correlates with closeness and the higher recovery rate limits the epidemic coverage, as presented in Figure \ref{bara_cov_clos}. 

Except accuracy on network size restricted simulations with SIR parameter class $C$, accuracy is highest for SIR parameter class $A$ and $B$ $(p = 0.3)$ when closeness of source node is low, as presented in Figure \ref{bara_acc_clos}.  The high accuracy for high values of closeness and SIR parameters in class $C$ can be explained by local network structure. 

Finally, it is interesting to see how the required converging number of samples for source estimation on SIR epidemic negatively correlates with higher closeness of the source nodes, as presented in Figure \ref{bara_sim_clos}. 

\begin{figure}[H]
\begin{minipage}{\textwidth}
\begin{minipage}{0.5\textwidth}
\includegraphics[width=\textwidth]{/home/iva/dipl/res/bara/epidemic_cov_clos_A.pdf}
\end{minipage}
\begin{minipage}{0.5\textwidth}
\includegraphics[width=\textwidth]{/home/iva/dipl/res/bara/epidemic_cov_clos_B.pdf}
\end{minipage}
\begin{minipage}{0.5\textwidth}
\includegraphics[width=\textwidth]{/home/iva/dipl/res/bara/epidemic_cov_clos_C.pdf}
\end{minipage}
\begin{minipage}{0.5\textwidth}
\includegraphics[width=\textwidth]{/home/iva/dipl/res/bara/epidemic_cov_clos_D.pdf}
\end{minipage}
\caption{Box plots of size of epidemics simulated on  $2$-Barab\'{a}si-Albert graphs  with $N=100$ nodes under SIR model with parameters $A = (p=0.3, q=0.3, T=5), B = (p=0.3, q=0.7, T=5), C = (p=0.7, q=0.3, T=5), D = (p=0.7, q=0.7, T=5)$ grouped by closeness of the source node.}
\label{bara_cov_clos}
\end{minipage}
\end{figure}


\begin{figure}[H]
\begin{minipage}{\textwidth}
%\begin{figure}[H]
\center
\includegraphics[width=0.75\textwidth]{/home/iva/dipl/res/bara/accuracy_clos.pdf}
\caption{Source detection accuracy on  $2$-Barab\'{a}si-Albert graphs  with $N=100$ nodes estimated with Soft Margin SIS detector with $10^4 - 10^6$ simulations and fixed $a = \frac{1}{2^5}$ under SIR model with parameters $A = (p=0.3, q=0.3, T=5), B = (p=0.3, q=0.7, T=5), C = (p=0.7, q=0.3, T=5), D = (p=0.7, q=0.7, T=5)$ grouped by closeness of the source node.}
\label{bara_acc_clos}
\end{minipage}
\end{figure}
%
%\begin{figure}[H]
%\begin{minipage}{0.5\textwidth}
%\includegraphics[width=\textwidth]{/home/iva/dipl/res/bara/clos_ent_A.pdf}
%\end{minipage}
%\begin{minipage}{0.5\textwidth}
%\includegraphics[width=\textwidth]{/home/iva/dipl/res/bara/clos_ent_B.pdf}
%\end{minipage}
%\begin{minipage}{0.5\textwidth}
%\includegraphics[width=\textwidth]{/home/iva/dipl/res/bara/clos_ent_C.pdf}
%\end{minipage}
%\begin{minipage}{0.5\textwidth}
%\includegraphics[width=\textwidth]{/home/iva/dipl/res/bara/clos_ent_D.pdf}
%\end{minipage}
%\caption{Violin plots of estimated entropy distribution for estimated source probability distribution on  $2$-Barab\'{a}si-Albert graphs  with $N=100$ nodes estimated with Soft Margin SIS detector with $10^4 - 10^6$ simulations and fixed $a = \frac{1}{2^5}$ under SIR model with parameters $A = (p=0.3, q=0.3, T=5), B = (p=0.3, q=0.7, T=5), C = (p=0.7, q=0.3, T=5), D = (p=0.7, q=0.7, T=5)$ grouped by closeness of the source node.}
%\end{figure}

\begin{figure}[H]
\begin{minipage}{0.5\textwidth}
\includegraphics[width=\textwidth]{/home/iva/dipl/res/bara/clos_sim_A.pdf}
\end{minipage}
\begin{minipage}{0.5\textwidth}
\includegraphics[width=\textwidth]{/home/iva/dipl/res/bara/clos_sim_B.pdf}
\end{minipage}
\begin{minipage}{0.5\textwidth}
\includegraphics[width=\textwidth]{/home/iva/dipl/res/bara/clos_sim_C.pdf}
\end{minipage}
\begin{minipage}{0.5\textwidth}
\includegraphics[width=\textwidth]{/home/iva/dipl/res/bara/clos_sim_D.pdf}
\end{minipage}
\caption{Box plots of converging samples distribution for source MAP estimations  on  $2$-Barab\'{a}si-Albert graphs  with $N=100$ nodes estimated with Soft Margin SIS detector with $10^4 - 10^6$ simulations and fixed $a = \frac{1}{2^5}$ under SIR model with parameters $A = (p=0.3, q=0.3, T=5), B = (p=0.3, q=0.7, T=5), C = (p=0.7, q=0.3, T=5), D = (p=0.7, q=0.7, T=5)$ grouped by closeness of the source node.}
\label{bara_sim_clos}
\end{figure}

\subsubsection{Betweenness centrality}

Betweenness centrality describes how well situated a vertex is in terms of the paths it lies on as defined in \ref{betw}. 

Similarly to closeness, epidemic size grouped by betweenness of the source node shows the epidemics of smaller size are expected to start from the source node with lower betweenness, at least for SIR parameter classes $A$ and $B$ where epidemic is still localized within network size, as presented in Figure \ref{bara_betw_cov}.

For  accuracy, on the other hand, betweenness shows to be the centrality measure that separates the detections so the ones more probable to produce correct results have source node of high betweenness, at least for SIR parameters in class $C$, as presented in Figure \ref{bara_acc_betw}. This can be explain by  local network structure since the centres of  local communities usually have high centrality.  

Similarly to closeness, the converging number of samples for source detection is negatively correlated with betweenness, as presented in Figure \ref{bara_sim_betw}.  Source detection for parameter classes with high recovery rate require more samples to converge than their low recovery counterparts.

\begin{figure}[H]
\begin{minipage}{\textwidth}
\begin{minipage}{0.5\textwidth}
\includegraphics[width=\textwidth]{/home/iva/dipl/res/bara/epidemic_cov_betw_A.pdf}
\end{minipage}
\begin{minipage}{0.5\textwidth}
\includegraphics[width=\textwidth]{/home/iva/dipl/res/bara/epidemic_cov_betw_B.pdf}
\end{minipage}
\begin{minipage}{0.5\textwidth}
\includegraphics[width=\textwidth]{/home/iva/dipl/res/bara/epidemic_cov_betw_C.pdf}
\end{minipage}
\begin{minipage}{0.5\textwidth}
\includegraphics[width=\textwidth]{/home/iva/dipl/res/bara/epidemic_cov_betw_D.pdf}
\end{minipage}
\caption{Box plots of size of epidemics simulated on  $2$-Barab\'{a}si-Albert graphs  with $N=100$ nodes under SIR model with parameters $A = (p=0.3, q=0.3, T=5), B = (p=0.3, q=0.7, T=5), C = (p=0.7, q=0.3, T=5), D = (p=0.7, q=0.7, T=5)$ grouped by betweenness of the source node.}
\label{bara_betw_cov}
\end{minipage}
%\end{figure}
%\begin{figure}[H]
\begin{minipage}{\textwidth}
\center
\includegraphics[width=0.75\textwidth]{/home/iva/dipl/res/bara/accuracy_betw.pdf}
\caption{Source detection accuracy on  $2$-Barab\'{a}si-Albert graphs  with $N=100$ nodes estimated with Soft Margin SIS detector with $10^4 - 10^6$ simulations and fixed $a = \frac{1}{2^5}$ under SIR model with parameters $A = (p=0.3, q=0.3, T=5), B = (p=0.3, q=0.7, T=5), C = (p=0.7, q=0.3, T=5), D = (p=0.7, q=0.7, T=5)$ grouped by betweenness of the source node.}
\label{bara_acc_betw}
\end{minipage}
\end{figure}

\begin{figure}[H]
\begin{minipage}{\textwidth}
\begin{minipage}{0.5\textwidth}
\includegraphics[width=\textwidth]{/home/iva/dipl/res/bara/betw_sim_A.pdf}
\end{minipage}
\begin{minipage}{0.5\textwidth}
\includegraphics[width=\textwidth]{/home/iva/dipl/res/bara/betw_sim_B.pdf}
\end{minipage}
\begin{minipage}{0.5\textwidth}
\includegraphics[width=\textwidth]{/home/iva/dipl/res/bara/betw_sim_C.pdf}
\end{minipage}
\begin{minipage}{0.5\textwidth}
\includegraphics[width=\textwidth]{/home/iva/dipl/res/bara/betw_sim_D.pdf}
\end{minipage}
\caption{Box plots of converging samples distribution for source MAP estimations on  $2$-Barab\'{a}si-Albert graphs  with $N=100$ nodes estimated with Soft Margin SIS detector with $10^4 - 10^6$ simulations and fixed $a = \frac{1}{2^5}$ under SIR model with parameters $A = (p=0.3, q=0.3, T=5), B = (p=0.3, q=0.7, T=5), C = (p=0.7, q=0.3, T=5), D = (p=0.7, q=0.7, T=5)$ grouped by betweenness of the source node.}
\label{bara_sim_betw}
\end{minipage}
\end{figure}

\subsubsection{Eigenvector centrality}

Eigenvector centrality describes  centrality of the source node in terms of node centrality of its neighbours, as defined in \ref{eig}. Epidemic source detectability grouped by eigenvector centrality shows similar results to closeness and betweenness for epidemic coverage, accuracy and  converging number of samples, as presented in Figure \ref{eig_cov}, \ref{eig_acc} and \ref{eig_sim}, respectively since the measures correlate positively.

\begin{figure}[H]
\begin{minipage}{\textwidth}
\begin{minipage}{0.5\textwidth}
\includegraphics[width=\textwidth]{/home/iva/dipl/res/bara/eig_cov_A.pdf}
\end{minipage}
\begin{minipage}{0.5\textwidth}
\includegraphics[width=\textwidth]{/home/iva/dipl/res/bara/eig_cov_B.pdf}
\end{minipage}
\begin{minipage}{0.5\textwidth}
\includegraphics[width=\textwidth]{/home/iva/dipl/res/bara/eig_cov_C.pdf}
\end{minipage}
\begin{minipage}{0.5\textwidth}
\includegraphics[width=\textwidth]{/home/iva/dipl/res/bara/eig_cov_D.pdf}
\end{minipage}
\caption{Box plots of size of epidemics simulated on  $2$-Barab\'{a}si-Albert graphs  with $N=100$ nodes under SIR model with parameters $A = (p=0.3, q=0.3, T=5), B = (p=0.3, q=0.7, T=5), C = (p=0.7, q=0.3, T=5), D = (p=0.7, q=0.7, T=5)$ grouped by eigenvector centrality of the source node.}
\label{eig_cov}
\end{minipage}
%\end{figure}
%\begin{figure}[H]
\begin{minipage}{\textwidth}
\center
\includegraphics[width=0.7\textwidth]{/home/iva/dipl/res/bara/accuracy_eig.pdf}
\caption{Source detection accuracy on  $2$-Barab\'{a}si-Albert graphs  with $N=100$ nodes estimated with Soft Margin SIS detector with $10^4 - 10^6$ simulations and fixed $a = \frac{1}{2^5}$ under SIR model with parameters $A = (p=0.3, q=0.3, T=5), B = (p=0.3, q=0.7, T=5), C = (p=0.7, q=0.3, T=5), D = (p=0.7, q=0.7, T=5)$ grouped by eigenvector centrality of the source node.}
\label{eig_acc}
\end{minipage}
\end{figure}

\begin{figure}[H]
\begin{minipage}{\textwidth}
\begin{minipage}{0.5\textwidth}
\includegraphics[width=\textwidth]{/home/iva/dipl/res/bara/eig_sim_A.pdf}
\end{minipage}
\begin{minipage}{0.5\textwidth}
\includegraphics[width=\textwidth]{/home/iva/dipl/res/bara/eig_sim_B.pdf}
\end{minipage}
\begin{minipage}{0.5\textwidth}
\includegraphics[width=\textwidth]{/home/iva/dipl/res/bara/eig_sim_C.pdf}
\end{minipage}
\begin{minipage}{0.5\textwidth}
\includegraphics[width=\textwidth]{/home/iva/dipl/res/bara/eig_sim_D.pdf}
\end{minipage}
\caption{Box plots of converging samples distribution for source MAP estimations on  $2$-Barab\'{a}si-Albert graphs  with $N=100$ nodes estimated with Soft Margin SIS detector with $10^4 - 10^6$ simulations and fixed $a = \frac{1}{2^5}$ under SIR model with parameters $A = (p=0.3, q=0.3, T=5), B = (p=0.3, q=0.7, T=5), C = (p=0.7, q=0.3, T=5), D = (p=0.7, q=0.7, T=5)$ grouped by eigenvector centrality of the source node.}
\label{eig_sim}
\end{minipage}
\end{figure}

\chapter{Conclusion}

Detectability of patient zero is source detection problem based on partial history of epidemic spreading and underlying network structure.
Assuming the epidemic started from a single source and based on the known infected nodes up to fixed time threshold simulated with discrete SIR epidemic spreading model, it is possible to construct MAP classifier based on estimation of epidemic source distribution. These Direct Monte Carlo and Soft Margin methods  of \citet{Nino} were implemented  and results on provided benchmark dataset were reproduced for SIR and ISS models. 

Importance sampling method and partial generation of epidemic spreading samples with Sequential importance sampling algorithm provide optimization  of Monte Carlo methods in terms of higher accuracy and faster convergence rate while omitting the necessity of direct simulation of  epidemic spreading.

Source detection detectability in terms of entropical detectability and accuracy is presented to be higher for higher values of SIR parameter $p$ and ISS parameter $a$ for rumour spreading model on the grid network. This directly correlates with epidemic size and unique position of each node in the epidemic which pushes towards a unique solution of the ill-posed problem of source detection and, consequently, lower entropy and better accuracy of estimated source probability distribution.

When analysing detectability on non-grid random Erd{\H{o}}s-R{\'{e}}nyi graph with $N=100$ nodes and $p=0.01$,  the role of SIR parameters $p$ and $q$  in source detectability intensifies. The expected entropy of source distribution for  detection on  epidemic with higher recovery rate is higher. With high recovery rate there is more ways to obtain the snapshot upon which we base the detection in terms of what nodes got infected and what nodes got recovered. Additionally, the probability of obtaining our snapshot is smaller than probability of obtaining the same snapshot without recovery and the state space stays at most of the size of state space with corresponding epidemic parameters but without recovery. This means more nodes will have similar source probability distribution and expected entropy will be higher.

Detectability is higher for lower values of centrality in Erd{\H{o}}s-R{\'{e}}nyi graph since the source nodes with lower values of centrality produce smaller epidemics. Accuracy of source detector on Erd{\H{o}}s-R{\'{e}}nyi graph is negatively correlated with centrality metrics accordingly.

Additionally, SIR parameter pairs that on the grid network obtain low detectability (the ones with low infection rate), produce on Erd{\H{o}}s-R{\'{e}}nyi graph accuracy comparable to accuracy of  SIR parameter classes (the ones with high infection rate) that have high detectability on the grid network too. This can be explained by the random structure of Erd{\H{o}}s-R{\'{e}}nyi graph where each node is by default structurally more unique than the node in the grid network and therefore  more distinguishable, even within small epidemics. 

The source node detectability is more restricted by centrality metrics of the source node for the epidemics with high infection rate. 

 Detectability based on centrality measures on  $2$-Barab\'{a}si-Albert graphs with $N = 100$ nodes is presented to be higher for the source node having lower centrality in terms of degree, closeness, betweenness and eigenvector centrality for both graph types. The detectability is similarly positively correlated with centrality measures of the source node. The presence of local network structure in Barab\'{a}si-Albert graphs and importance of nodes of high centrality in local communities plays important role in detectability of source nodes with high centrality. Namely, the detectability of these nodes as the source nodes of epidemic is high. The detectability of nodes with low centrality remains high too, similarly to Erd{\H{o}}s-R{\'{e}}nyi graph.
 
 \bibliography{Bibliography}
%\bibliographystyle{plainnat}
\bibliographystyle{unsrtnat}

\begin{sazetak}
Odredivost nultog pacijenta je problem traženja izvora zaraze na temelju djelomične povijesti epidemiološke dinamike i mrežne strukture. 
 Uz pretpostavku da je epidemija krenula iz jednog čvora i na temelju cjelokupnog znanja o mreži te poznavanja zaraženih čvorova do fiksnog vremenskog trenutka u diskretnom SIR modelu širenja zaraze, moguće je konstruktirati MAP klasifikator na temelju estimacije distribucije vjerojatnosti izvora zaraze pomoću Monte Carlo metoda, a koje se dodatno optimiziraju  uzorkovanjem po važnosti. Izvedeni novi Soft Margin sekvencijalni Monte Carlo detektor usprediv je s detektorom iz \emph{benchmarka} uz bolju konvergenciju i bolju vremensku složenost.
 
Pokazano je da detektabilnost kao entropijska detekatiblnost i točnost detekcije  ovise o vrijednostima parametara SIR modela za širenje epidemije, a slično je pokazano i za detekciju izvora glasine prema  ISS modelu širenja glasine. Za SIR model pokazano je kako je lakše otkriti izvor zaraze ako početni čvor ima manju centralnost. 
 
\kljucnerijeci{kompleksne mreže, odredivost nultog pacijenta, širenje epidemije, širenje glasina, Monte Carlo metode, sekvencijalni Monte Carlo}
\end{sazetak}
\newpage

\engtitle{Detectability of Patient Zero Depending on its Position in the Network}
\begin{abstract}

Detectability of patient zero is source detection problem based on partial history of epidemic spreading and underlying network structure.
Assuming the epidemic started from a single source and based on the known infected nodes up to fixed time threshold simulated with discrete SIR epidemic spreading model, it is possible to construct MAP classifier based on estimation of epidemic source distribution using Monte Carlo methods optimised with importance sampling. The new Soft Margin Sequential Monte Carlo detector is comparable with the benchmark detector with better convergence and better time complexity.

It is presented how detectability in terms of entropical detectability and accuracy depends on parameters of the SIR model  for  epidemic source detection, as well as how parameters of ISS model for rumour spreading govern the rumour source detection. For the SIR model it is presented the source node with lower centrality is expected to be more detectable.

\keywords{complex networks, detectability of patient zero, epidemic spreading, rumour spreading, Monte Carlo methods, Sequential Monte Carlo}
\end{abstract}

\end{document}
